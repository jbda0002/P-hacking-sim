% --------------------------------------------------------------
% Preamble
% --------------------------------------------------------------

\documentclass[english,man,floatsintext,natbib]{apa7}
% \usepackage{arxiv}
\usepackage[english]{babel}
\usepackage[utf8]{inputenc} % allow utf-8 input
\usepackage[T1]{fontenc}    % use 8-bit T1 fonts
\usepackage{url}            % simple URL typesetting
\usepackage{hyperref}       % hyperlinks
\usepackage{xcolor}
\hypersetup{
    colorlinks,
    linkcolor={red!50!black},
    citecolor={blue!50!black},
    urlcolor={blue!80!black}
}
\usepackage{booktabs}       % professional-quality tables
\usepackage{tabularx}
\usepackage{amsfonts}       % blackboard math symbols
\usepackage{nicefrac}       % compact symbols for 1/2, etc.
\usepackage{microtype}      % microtypography
\usepackage{gensymb}        % degree, angle symbols
\usepackage{lscape} 

% for math equations and symbols
\usepackage{amsmath} 
\usepackage{amssymb}
\newcommand{\E}{\mathbb{E}}
\newcommand{\SD}{\mathit{SD}}
\newcommand{\SE}{\mathit{SE}}
\newcommand{\BF}{\mathit{BF}}
\DeclareMathOperator\arctanh{arctanh}

% This prevents placing floats before a section.
\usepackage{placeins}

% bibliography
%\usepackage{natbib}

% allows for floats when doing jou or doc style
\usepackage{graphicx}      
\graphicspath{{./Figures/}}  
\usepackage{threeparttable}
%\usepackage{float}

% for long tables
\usepackage{longtable}
% for multitables
\usepackage{multirow}
% to rotate figures
\usepackage{rotating}
% linenumbers
\usepackage[mathlines]{lineno}
\usepackage{pdflscape}

% in APA man mode captions are too large
% making sure they are reasonable
\usepackage{setspace}
% \usepackage[font=singlespacing]{caption}
% \captionsetup{font=singlespacing}

%%% some support for commenting
\usepackage{color}
\definecolor{Blue}{RGB}{0,0,255}
\definecolor{Red}{RGB}{255,0,0}
\definecolor{Green}{RGB}{0,255,0}
\newcommand{\jdc}[1]{\textcolor{Red}{[Jacob DC: #1]}}  
\newcommand{\jlo}[1]{\textcolor{Blue}{[Jacob O: #1]}} 
\newcommand{\spe}[1]{\textcolor{Green}{[Sonja: #1]}}

% ---------------------------------------------------------
% Title, authors
% ---------------------------------------------------------

\title{Inflated false-positive risk in common regression analyses:\\ A combinatorial analysis of model sets}
\shorttitle{false-positives in common regression analyses}

\authorsnames[{1},{2,3},{2},{1}]{{Jacob D. Christensen*},{Jacob L. Orquin},{Sonja Perkovic},{Carl-Johan Lagerkvist}}

\authorsaffiliations{{Swedish University of Agricultural Sciences},{Aarhus University},{Reykjavik University}}



\authornote{Jacob D. Christensen, *Correspondence concerning this article should be addressed to Jacob D. Christensen, Department of Economics at Swedish University of Agricultural Sciences. E-mail: jacob.dalgaard.christensen@slu.se.}

% ---------------------------------------------------------
% Abstract
% ---------------------------------------------------------

\abstract{Even with a small number of variables researchers can test many possible models of their data thus increasing the risk of false-positive results. Using combinatorics, we show that one key independent variable and three covariates can generate 95 possible models, while six covariates can generate over 2.3 million models. Using simulation, we show that exploring such large model sets nearly guarantees false-positive results. Preregistering a single analysis heavily reduces the problem, but surprisingly, many correctly specified models produce false-positive results at a much higher rate than the expected 5\%. The problem arises when the key independent variable is contained in multiple model terms such as main effects or interactions. We show that a Bonferroni correction of the alpha by the number of terms containing the key independent variable is sufficient to ensure the desired alpha level. 
}

\keywords{type I error, false-positive statistics, replicability, simulation} 

\begin{document}
\maketitle

% introduction
Imagine that you are interested in the effect of a specific variable A (the independent variable) on another variable B (the dependent variable). You are not sure whether A may interact with other variables, so you decide to collect several covariate variables you think could influence your presumed causal chain. After collecting the data, you run multiple models with and without these additional variables as covariates in order to probe the data; you remove some outliers to improve the clarity of the results; and you finally find that your independent variable significantly interacts with one of your covariates. Hopefully by now, most researchers are aware that this research process is likely to result in a false positive finding (i.e., Type I error) and a low chance of replicability. Consequently, it is advisable that researchers plan and conduct their research differently; before collecting any data, they should preregister their study and all variables and hypotheses they aim to test. As everything is preregistered, the researcher may only run one model to test each hypothesis. Additional models or tests can still be performed and published, but are then labelled as exploratory. Following this procedure, one can expect to increase the replicability of research findings by lowering the risk of false positives to any given preset alpha level (such as the conventional level of statistical significance of 5\%). But can researchers rely on this procedure to entirely eliminate false positives? \\

There is no doubt that there are many issues with analyzing the data as in the first example, as running multiple models will increase the probability of false positive results. Simulations suggest that running even a small number of models with different variables, dropping conditions etc. can increase the chance of a false positive result to 81.5\% \citep{Simmons2011}. The Open Science solution to this problem is that the researcher commits to performing only one analysis, with the possibility to label other analyses that go beyond the plan as exploratory \citep{Nosek2018}. This procedure should lower the false positive for the preregistered study to a predetermined alpha level, normally set to 5\% \citep{Moore2016}. However, the 5\% chance of a false positive result is an assumption that relies on running only one single hypothesis test and that all statistics are done correctly. There are several studies showing that misspecifying a  statistical model can increase the false positive to above the alpha level even when only running one single model \citep{Dennis2019,Litiere2007}. Misspecification is not the only challenge; noise in measured independent variables can also increase the risk of false positives \citep{Brunner2009} and so can variables with ceiling effects \citep{Austin2003}. Austin and Brunner also show that the rates of false positive findings increase when having the interaction effects with a naturally continuous covariate which is dichotomized \citep{Austin2004}. The problem with such dichotomization is further aggravated when the sample size increases, which runs counter to the recommendation to use larger samples in order to minimize the risk of Type I error \citep{Simmons2011}. 
\\
It may appear tempting to dismiss these findings as irrelevant as long as we rely on sound scientific and statistical practices. However, it is easy to determine if a model is misspecified in a simulation study since we know the ground truth about the data generating process, but in reality we hardly ever know this. Can we judge the risk of misspecification simply from the model? Perhaps we can, by looking at issues such as estimating interaction terms, but omitting the corresponding main effects. By omitting the main effects, which is the same as setting the main effect to zero, can lead to part of the variance to be captured in the error term which will then be correlated with the interaction term \citep{Branbor2006}. The practice of estimating interactions without the corresponding main effects is widespread \citep{Branbor2006} and some method books still suggest doing this \citep{Cleves2008}. This is justified with arguing that if just the model is based on theory, this is an acceptable practice \citep{aiken1991multiple}. To the best of our knowledge, no studies have examined whether this practice leads to an increase in false positives findings. The same is true for many other common practices. Clearly, running models with and without outlier deletion increases false positives due to multiple testing, but it is also possible that running only one model with outlier deletion could increase false positives if the exclusion leads to range restriction in the dependent or independent variable \citep{Raju2003}. \\        

We believe there is a need for a more nuanced understanding regarding how different specifications and factors in the analysis can increase the risk of false positives. There are already simulation studies examining what the false positive probability (FPP) \citep{Simmons2011}, which can be thought of as the chance of erroneously finding one or more significant models among all the tested ones. However, while the FPP can reveal problems with conducting multiple tests and with flexible research practices in general, it does not clarify whether a particular combination of methods is more likely than others to produce a false positive. To understand this latter aspect, we rather need to examine the false positive ratio (FPR), hereinafter defined as the probability that a selected model produces a false positive out of any given set of models from which a researcher can choose. \\

In the present article, we report the results of a simulation study that examines the FPP and FPR, respectively. A simulation approach is useful since it is not obvious how to calculate the false positive probability and false positive ratio when the different factors interact with each other. By examing both metrics we bring together two strands of literature: one concerned with p-hacking\footnote{This also has other names, such as researchers degrees of freedom \citep{Simmons2011}, data snooping \citep{white2000reality}, data mining \citep{lovell1983}, and data fishing \citep{selvin1966data} .} \citep{simonsohn2014p} and researcher flexibilities and another concerned with type I error rates due to specific conditions such as model misspecification. We believe both perspectives are necessary to advance science and replicability since open science practices may not be a sufficient safeguard against false positive findings – some models and methods could have inflated false positive rates even though they are preregistered. In our simulation study we include some of the factors previous examine by \cite{Simmons2011} and flexibilities identified by \cite{Wicherts2016}. These include expanding the model with more covariates, including interactions both between the variable of interest, but also between the covariates themselves. When including interactions, we examine what happens when the main effect is included or not. We also examine the effect of the most commonly used outlier criteria in the literature \citep{Leyes2013} and the effect of using several correlated dependent variables. Simmons and colleagues recommended to increase the sample size to overcome some of the problem with false positive findings \citep{Simmons2011}, and we therefore also examine how an increase in sample size affects the false positive probability and false positive ratio. 



% alternative method section
%\section{Combinatorial analysis}

When a researcher has to select which model(s) to use for testing a hypothesis, there are a lot of possibilities regarding what variables to collect and how they should enter the model. In many cases a researcher is either interested in how the variable of interest, in this case denoted as $x$, interacts with other variables or there is a need to control for other variables, which we call covariates and denote as $z=(z_1,z_2,..,z_n)$. 
To better understand the construction of model sets, we divided the models into seven sets based on different interaction structures. The first three sets contained the key variable of interest and simple interactions, whereas the last four sets were the combinations of the first three sets. For clarity, the model equations exclude the constant and the coefficient in front of each variable. The first set only includes models of the main effects of the variable of interest and the covariates ($x + z$). The second set includes models with a main effect of the variable of interest and interactions between the variable of interest and covariates ($x \times z$)\footnote{For simplicity, we omit the main effect of the variable of interest, $x$, in the notation for the second, third, and sixth model set.}. The third set includes models with a main effect of the variable of interest and covariate-covariate interactions ($z \times z$).
To obtain the last four model sets, we combine the first three model sets. This gives us the sets $x + z + x \times z$, $x + z + z \times z$, $z \times z + x \times z$, and $x + z + x \times z + z \times z$. We will go through each of the sets and show how to calculate how many models there are in them. There are, however, two ways of populating the seven model sets: with restrictions on main effects and without restrictions on main effects. Restrictions on main effects mean that whenever there is an interaction effect, the corresponding main effects have to be included in the model. Models without restrictions, on the other hand, are allowed to include interaction effects without the corresponding main effects. The calculations for the number of models in each model set with restrictions can be seen in the next paragraph whereas with no restrictions these calculation can be found in the \textit{Appendix} in section \nameref{witout_res}. 

\subsection{Model sets with restrictions}
\label{with_res}
In the case with restrictions, we enforce that every time there is an interaction effect, the corresponding main effects are included in the model as in the model below:
$y=x_1+z_1+x_1z_1$ 

In the previous equation, we have an interaction between our variable of interest $x_1$ and a covariate $z_1$. Therefore, the two corresponding main effects also enter the model. This restriction does not put any restrictions on the $x + z$ set as there are no interactions in it. On the other hand, the restriction implies that the sets $x \times z$ and $z \times z$ are empty sets as the models in these sets never have main effects following interactions. We will therefore not show them here, but they are part of the model sets in the \textit{Appendix} in section \nameref{witout_res}.

\subsubsection{Set $x + z$}
The set $x + z$ contains all the combinations that can be made by the covariates $z$ and the variable of interest $x$ including the model where there is only the variable $x$. Based on the previous terminology the model containing only the variable of interest is therefore the empty set and the binomial coefficient computes how many combinations there can be made by $z$. Let us denote the number of covariates $n$. If we, for example, have three covariates ($n=3$), then the number of combinations where all three covariates are present can be written using the binomial coefficient as


\[\#\ of\ models\ in\ x + z = \left( \begin{array}{c}
3 \\ 
3 \end{array}
\right)+\left( \begin{array}{c}
3 \\ 
2 \end{array}
\right)+\left( \begin{array}{c}
3 \\ 
1 \end{array}
\right)+\left( \begin{array}{c}
3 \\ 
0 \end{array}
\right)=\sum^3_{i=0}{\left( \begin{array}{c}
3 \\ 
i \end{array}
\right)}\] 
Using the rule for adding binomial coefficients, we can rewrite this into
\[\#\ of\ models\ in\ x + z = \sum^3_{i=0}{\left( \begin{array}{c}
3 \\ 
i \end{array}
\right)}=\left(2^3\right)=8\] 
We can generalize this to the case with $n$ covariates  as follows 
\[\#\ of\ models\ in\ x + z = \sum^n_{i=0}{\left( \begin{array}{c}
n \\ 
i \end{array}
\right)}=\left(2^n\right)\] 

\subsubsection{Formally written} This size of this set can also be more formally calculated. This is done by using the notation from set theory to calculate the size of each set.
Let $Z$ be the set of covariates 
\[Z=\{\left.z_1,z_2,..,z_n\right.\}\] 
\[\left|Z\right|=n\] 


\noindent The $x + z$ set is then the power set of \emph{Z} 
\[x + z = \{\left.S:S\subseteq Z\right.\}\] 

Each element $S\in x + z$ is a set of the covariates e.g., $S=\left.z_3,z_7\right.$, or any combination of the covariates, including the empty set. This is just the power set of $Z$ and the size of the set can be written as
\[\left|x + z\right|=|\mathcal{P}\left(Z\right)|=2^n\] 


% section
\subsubsection{Set $x + z + x \times z$}
For the sake of simplicity, we begin with the case where $n=3$. The models in this set can be seen in Table \ref{tab:model2}. Here the models are split into three different parts. These are split such that $x + z + x \times z$(1), $x + z + x \times z$(2), and $x + z + x \times z$(3) contain all the models where there are one, two, or three covariates, respectively.\\

\begin{table}[ht!]
\caption{Example of division of the $x + z + x \times z$ set into three subsets when there are three covariates and one dependent variable. $x + z + x \times z$(1) contains all the models with one covariate as main effect, $x + z + x \times z$(2) all the models with two covariates, and $x + z + x \times z$(3) all the models with three covariates}
\centering\label{tab:model2}
\begin{tabular}{lc}  
\toprule
Set & Models \\
\midrule
\multirow{3}{*}{$x + z + x \times z$(1)} & $y=x_1+z_1+x_1z_1$\\ & $y=x_1+z_2+x_1z_2$\\ & $y=x_1+z_3+x_1z_3$\\ & \\ 
\multirow{9}{*}{$x + z + x \times z$(2)} & $y=x_1+z_1$$+z_2+x_1z_1$\\ & $y=x_1+z_1+z_3+x_1z_1$\\ & $y=x_1+z_3+z_1+x_1z_3$\\ & $y=x_1+z_3+z_2+x_1z_3$\\ & $y=x_1+z_2+z_1+x_1z_2$\\ & $y=x_1+z_2+z_3+x_1z_2$\\ & $y=x_1+z_1+z_2+x_1z_1+x_1z_2$\\ & $y=x_1+z_1+z_3+x_1z_1+x_1z_3$\\ & $y=x_1+z_1+z_2+x_1z_3+x_1z_2$\\  & \\  
\multirow{7}{*}{$x + z + x \times z$(3)} & $y=x_1+z_1+z_2+z_3+x_1z_1$\\ & $y=x_1+z_1+z_2+z_3+x_1z_2$\\ & $y=x_1+z_1+z_2+z_3+x_1z_3$\\ & $y=x_1+z_1+z_2+z_3+x_1z_1+x_1z_2$\\ & $y=x_1+z_1+z_2+z_3+x_1z_1+x_1z_3$\\ & $y=x_1+z_1+z_2+z_3+x_1z_3+x_1z_2$\\ & $y=x_1+z_1+z_2+z_3+x_1z_1+x_1z_2+x_1z_3$\\  
\bottomrule
\end{tabular}
\end{table}

To calculate the size of each of these sets separately, we only need to focus on the number of covariates that enter the model, as per assumption the variable of interest $x_1$ is always present. Let us start with the $x + z + x \times z$(1) set. In this set, there can only be one covariate present at the time; there are therefore three combinations, which can be written as $\binom{3}{1}$. In this set, since there is only one covariate, there is only one interaction effect containing that same covariate. As no other combinations can be made, this corresponds to $\binom{1}{1}$. Using the same shorthand notation with the binomial coefficient as above, we can write the number of models in the $x + z + x \times z$(1) set as
\[\#\ of\ models\ in\ x + z + x \times z\textit{(1)}\ =\left( \begin{array}{c}
1 \\ 
1 \end{array}
\right)\left( \begin{array}{c}
3 \\ 
1 \end{array}
\right)=\sum^1_{i=1}{\left( \begin{array}{c}
1 \\ 
i \end{array}
\right)}\left( \begin{array}{c}
3 \\ 
1 \end{array}
\right)=\left(2^1-1\right)\left( \begin{array}{c}
3 \\ 
1 \end{array}
\right)=3\] 
Using this notation is useful when we later have to combine the three sets of $x + z + x \times z$ into one to calculate the total number of models in $x + z + x \times z$.
\raggedbottom
For the $x + z + x \times z$(2) set, we first need to calculate the number of combinations for the main effect that can be made when there are two covariates present at all times and that is $\binom{3}{2}$. Next we need to calculate the number of combinations of interactions that can be present. As we have two covariates present all the time, these are the ones that can be used in the interaction effects. Since the covariates interact with $x_1$, there can either be one or two interaction effects present. Hence, we have $\binom{2}{1}$ and $\binom{2}{2}$ combinations of the interaction effects possible. Combining the number of covariates with the number of possible interactions we get
\raggedbottom
\begin{equation*}
\begin{aligned}
\#\ of\ models\ in\ x + z +x \times z\textit{(2)}=\\
& \left( \begin{array}{c}
2 \\ 
1 \end{array}
\right)\left( \begin{array}{c}
3 \\ 
2 \end{array}
\right)+\left( \begin{array}{c}
2 \\ 
2 \end{array}
\right)\left( \begin{array}{c}
3 \\ 
2 \end{array}
\right)=\\
&\sum^2_{i=1}{\left( \begin{array}{c}
2 \\ 
i \end{array}
\right)}\left( \begin{array}{c}
3 \\ 
2 \end{array}
\right)=\left(2^2-1\right)\left( \begin{array}{c}
3 \\ 
2 \end{array}
\right)=9

\end{aligned}
\end{equation*}
\raggedbottom
The same reasoning goes for the $x + z + x \times z$(3) set; the only difference here is that we have one more covariate. So if we write the combinations that are possible with three covariates in the same way as before, we get
\begin{equation*}
\begin{aligned}
\#\ of\ models\ in\ x + z + x \times z\textit{(3)}=\\
&\left( \begin{array}{c}
3 \\ 
1 \end{array}
\right)\left( \begin{array}{c}
3 \\ 
3 \end{array}
\right)+\left( \begin{array}{c}
3 \\ 
2 \end{array}
\right)\left( \begin{array}{c}
3 \\ 
3 \end{array}
\right)+\left( \begin{array}{c}
3 \\ 
3 \end{array}
\right)\left( \begin{array}{c}
3 \\ 
3 \end{array}
\right)= \\
&\sum^3_{i=1}{\left( \begin{array}{c}
2 \\ 
i \end{array}
\right)}\left( \begin{array}{c}
3 \\ 
2 \end{array}
\right)= \\
&\left(2^3-1\right)\left( \begin{array}{c}
3 \\ 
3 \end{array}
\right)=7
\end{aligned}
\end{equation*}
We can then sum all three sets that is, $x + z + x \times z$(1), $x + z + x \times z$(2), and $x + z + x \times z$(3), to get the size of the $x + z + x \times z$ set
\[\#\ of\ models\ in\ x + z + x \times z=\sum^3_{k=1}{(2^k-1)\left( \begin{array}{c}
3 \\ 
k \end{array}
\right)}\]where $k$ is the number of covariates that enter as main effects, meaning it is an indicator for $x + z + x \times z$(1), $x + z + x \times z$(2) and $x + z + x \times z$(3). For the case with $n$ covariates, this becomes
\[\#\ of\ models\ in\ x + z + x \times z=\sum^n_{k=1}{(2^k-1)\left( \begin{array}{c}
n \\ 
k \end{array}
\right)}\] 

\subsubsection{Formally written} Again, we can calculate the size of this set in a more formal way
As before, let $Z$ be a set of covariates 
\[Z=\{\left.z_1,z_2,\dots ,z_n\right.\}\] \[|Z|=n\] 

Let $J$ be a set of indicators of how many covariates enter the model set as main effects
\[J=\{\left.1,2,3,4,\dots ,n\right.\}\] 
Let $L$ be a subset of $Z$ of size $k$
\[L=\{\left.S:S\subset Z,\left|S\right|=k,k\in J\right.\}\] 
The number of interactions that can be made between a covariate and the variable of interest for a given $k$ is then
\[I_h\left(Z\right)=\{\{\left.\left.z_i,x_1\right.\}:z_i\in L\right.\}\] 
\[\left|I_h\left(Z\right)\right|=k\] 

For each $k$, the number of interactions is just the power set of $Z$ excluding the empty set
\[\left|\mathcal{P}\left(I_h\left(Z\right):I_h\left(Z\right)\neq \textrm{\O}\right)\right|=2^k-1\] 

For each $k$, there is $\left( \begin{array}{c}
n \\ 
k \end{array}
\right)$ combinations of interactions that can be present with the number of covariates $k$. To get the full set, we need to sum over these combinations from $k=1$ to $n$ (with $k=0$ it would not be possible to make any interactions so this one is excluded).
\[\left|x + z + x \times z\right|=\sum^n_{k=1}{\left( \begin{array}{c}
n \\ 
k \end{array}
\right)\left(2^k-1\right)}\] 

% section
\subsubsection{Set $x + z + z \times z$}
Here we can use the same line of reasoning as for the $x + z + x \times z$ set. Let us begin by examining the case where $n=3$ and then develop a more general notion. To make it easier to understand how to calculate the set size, we split this set into subsets as we did above for the $x + z + x \times z$ set. This example can be seen in Table \ref{tab:model4}. 

\begin{table}
\centering
\caption{Division of the $x + z + z \times z$ set into two subsets when there are three covariates and one dependent variable. $x + z + z \times z$(2) contains all the models with two covariates as main effects, and $x + z + z \times z$(3) all the models with three covariates. $x + z + z \times z$(1) is an empty set since at least two covariates are necessary to make an interaction.}
\label{tab:model4}
\begin{tabular}{lc} 
\toprule
Set & Models \\ 
\midrule
\multirow{3}{*}{$x + z + z \times z$(2)} & $y=x_1+z_1+z_2+z_1z_2$\\ & $y=x_1+z_1+z_3+z_1z_3$\\ & $y=x_1+z_2+z_3+z_2z_3$\\ &  \\  
\multirow{7}{*}{$x + z + z \times z$(3)} & $y=x_1+z_1+z_2+z_3+z_1z_2$\\ & $y=x_1+z_1+z_3+z_2+z_1z_3$\\ & $y=x_1+z_2+z_3+z_1+z_2z_3$\\ & $y=x_1+z_1+z_2+z_3+z_1z_2+z_1z_3$\\ & $y=x_1+z_1+z_3+z_2+z_1z_3+z_2z_3$\\ & $y=x_1+z_2+z_3+z_1+z_1z_2+z_2z_3$\\ & $y=x_1+z_1+z_2+z_3+z_1z_2+z_2z_3+z_1z_3$\\ 
\bottomrule
\end{tabular}
\end{table}

A model must always include at least two covariates to fulfil the restriction that main effects follow interactions effects. Therefore, we can divide the $x + z + z \times z$ set into two subsets; the first subset has two covariates present as main effects and the second subset has three covariates present as main effects. The first set, $x + z + z \times z$(2), is the number of combinations that can be made from two out of three covariates, meaning $\binom{3}{2}$, multiplied with the combinations of the two covariates that are included, meaning $\binom{2}{2}$.
\[\#\ of\ models\ in\ x + z + z \times z\textit{(2)} =\left( \begin{array}{c}
3 \\ 
2 \end{array}
\right)\left( \begin{array}{c}
2 \\ 
2 \end{array}
\right)=3\] 
For the second subset, $x + z + z \times z$(3), the first term is the number of combinations that can be made with three covariates so this is $\binom{3}{3}$. We then have to calculate the number of interactions that can be made with the three included covariates. To do that, first we need to calculate the number of two-way interactions that can be made with three covariates and this is $\binom{3}{2}=3$. The total number of models in the set is then the sum of the number of combinations
\[\#\ of\ models\ in\ x + z + z \times z\textit{(3)}=\left( \begin{array}{c}
3 \\ 
3 \end{array}
\right)\left(\left( \begin{array}{c}
3 \\ 
1 \end{array}
\right)+\left( \begin{array}{c}
3 \\ 
2 \end{array}
\right)+\left( \begin{array}{c}
3 \\ 
3 \end{array}
\right)\right)=\left( \begin{array}{c}
3 \\ 
3 \end{array}
\right)\sum^3_{i=1}{\left( \begin{array}{c}
3 \\ 
i \end{array}
\right)}\] 
However, it is only in this example that we sum to $n=3$. In general, this will be the number of pairs that can be made in this subset. The number of pairs can be written as $\left( \begin{array}{c}
n \\ 
2 \end{array}
\right)=\frac{n\left(n-1\right)}{2}$. This then becomes
\[\#\ of\ models\ in\ x + z + z \times z\textit{(3)}=\left( \begin{array}{c}
3 \\ 
3 \end{array}
\right)\sum^3_{i=1}{\left( \begin{array}{c}
3 \\ 
i \end{array}
\right)}=\left( \begin{array}{c}
3 \\ 
3 \end{array}
\right)\sum^{\frac{3\left(3-1\right)}{2}}_{i=1}{\left( \begin{array}{c}
\frac{3\left(3-1\right)}{2} \\ 
i \end{array}
\right)}=\left( \begin{array}{c}
3 \\ 
3 \end{array}
\right)\left(2^{\frac{3\left(3-1\right)}{2}}-1\right)\] 
We can write the first subset in the same way
\[\#\ of\ models\ in\ x + z + z \times z(2)=\left( \begin{array}{c}
3 \\ 
2 \end{array}
\right)=\left( \begin{array}{c}
3 \\ 
2 \end{array}
\right)\left(2^{\frac{2\left(2-1\right)}{2}}-1\right)\] 
Now we just add these two subsets up
\[\#\ of\ models\ in\ x + z + z \times z=\left( \begin{array}{c}
3 \\ 
2 \end{array}
\right)\left(2^{\frac{2\left(2-1\right)}{2}}-1\right)+\left( \begin{array}{c}
3 \\ 
3 \end{array}
\right)\left(2^{\frac{3\left(3-1\right)}{2}}-1\right)=\sum^3_{k=2}{\left( \begin{array}{c}
3 \\ 
k \end{array}
\right)\left(2^{\frac{k\left(k-1\right)}{2}}-1\right)}\] 
We can then generalize this case for $n$ covariates
\[\#\ of\ models\ in\ x + z+z \times z=\sum^n_{k=2}{\left( \begin{array}{c}
n \\ 
k \end{array}
\right)\left(2^{\frac{k\left(k-1\right)}{2}}-1\right)}\] 

\subsubsection{Formally written} As done in the other two cases we let $Z$ be a set of covariates 
\[Z=\{\left.z_1,z_2,\dots ,z_n\right.\}\] 
\[|Z|=n\] 
Let $k$ be an indicator of how many covariates are included. Since we are using interactions, $k$ must begin at two. Let $T$ be a set of indicators of the number of covariates beginning at two
\[T=\{\left.2,3,4,\dots ,n\right.\}\] 
Let $L$ be a subset of $Z$ of size $k$
\[L=\{\left.S:S\subset Z,\left|S\right|=k,k\in T\right.\}\] 

The number of interactions that can be made for a given $k$ is then
\[I_k\left(Z\right)=\{\left\{\left.z_i,z_j\right\}:z_i\in L,z_j\in L,z_i\neq z_j\right.\}\] 
For a given $k$, the number of two-way interactions that can be made is then
\[\left|I_k\left(Z\right)\right|=\left( \begin{array}{c}
k \\ 
2 \end{array}
\right)=\frac{k!}{2!\left(k-2\right)!}=\frac{k\left(k-1\right)\left(k-2\right)!}{\left(k-2\right)!}\frac{1}{2}=\frac{k\left(k-1\right)}{2}\] 
We then need the combinations of all the two-way interactions. For all $k$, we can take the power set of $I_k\left(H\right)$ excluding the empty set
\[P\left(I_k\left(Z\right)\right)=\{\left.T:T\subset I_k\left(Z\right),T\neq \textrm{\O}\right.\}\] 
\[\left|P\left(I_k\left(Z\right)\right)\right|=2^{\left|I_k\left(Z\right)\right|}-1=2^{\frac{k\left(k-1\right)}{2}}-1\] 
With $n$ covariates, there exist $\left( \begin{array}{c}
n \\ 
k \end{array}
\right)$ subsets of size $k$. To get the $x + z + z \times z$ set, we need to sum over the number of covariates $k$ starting from $k=2$
\[\left|x + z+z \times z\right|=\sum^n_{k=2}{\left( \begin{array}{c}
n \\ 
k \end{array}
\right)}\left(2^{\frac{k\left(k-1\right)}{2}}-1\right)\ \] 

% section
\subsubsection{Set $x + z + x \times z$ + $z \times z$}
This set is the product of the $x + z + x \times z$ and $x + z + z \times z$ sets, but without including the main effect twice. Hence, the set consists of the combinations that can be made between the variable of interest and the covariates and between the covariates alone. We can therefore write 
\[\#\ of\ models\ in\ x + z + x \times z+\ z \times z=\sum^n_{k=2}{\left( \begin{array}{c}
n \\ 
k \end{array}
\right)\left(2^k-1\right)\left(2^{\frac{k\left(k-1\right)}{2}}-1\right)}\] 

\subsubsection{Formally written}
The design of the subsets is the same as earlier and we just need to take the product of
$P\left(I_k\left(Z\right)\right)$ and $P\left(I_h\left(Z\right)\right)$, as defined above, and sum over $k$ beginning at two.
For $n$ covariates, there exists $\left( \begin{array}{c}
n \\ 
k \end{array}
\right)$ subsets of size $k$. We then need to sum the subsets as follows
\[|x + z + x \times z+z \times z|=\sum^n_{k=2}{\left( \begin{array}{c}
n \\ 
k \end{array}
\right)\left(2^k-1\right)\left(2^{\frac{k\left(k-1\right)}{2}}-1\right)}\] 


% section
\subsection{Full model set with restrictions}
Equation (1) shows how to calculate the total number of models given one key variable of interest and a given number of covariates with the main effects restriction that models with interaction effects must include the corresponding main effects. Each underlined section in Equation (1) can also be used to calculate the size of the specified model set.
 restriction. \\
\begin{equation} 
\begin{aligned}
\#\ models={} & \underbrace{\left(2^n\right)}_{x + z}+\underbrace{\sum^n_{k=1}{\left(2^k-1\right)\binom{n}{k}}}_{x + z + x \times z} + \\ 
& \underbrace{\sum^n_{k=2}{\binom{n}{k}\left(2^{\frac{k\left(k-1\right)}{2}}-1\right)}}_{x + z + z \times z} + \\
& \underbrace{\sum^n_{k=2}{\binom{n}{k}\left(2^k-1\right)\left(2^{\frac{k\left(k-1\right)}{2}}-1\right)}}_{x + z + x \times z + z \times z}\ \  
\end{aligned}
\end{equation}  
where $n$ denotes the number of covariates (all the $z$ variables) collected and $\binom{n}{k}$ is the binomial coefficient, the calculation of which is explained in the \textit{Appendix} in the section \nameref{binomial}.
Equation (1) does not include the ($x \times z$), ($z \times z$), and ($x \times z + z \times z$) sets as these include interaction effects without the corresponding main effects, and these are therefore empty sets. For the case with two covariates, the total number of models is calculated as follows:

\begin{equation*}
\centering
\#\ of\ models=\left(2^2\right)+
\sum^2_{k=1}{\left(2^k-1\right) \binom{2}{k}}+
\sum^2_{k=2}{\binom{2}{k}\left(2^{\frac{k\left(k-1\right)}{2}}-1\right)}+  
\sum^2_{k=2}{\binom{2}{k}\left(2^k-1\right)\left(2^{\frac{k\left(k-1\right)}{2}}-1\right)}
\end{equation*}

which is equal to $4+5+1+3*1=13$. The number of models in each model set for different numbers of covariates depending on main effect restrictions can be seen in Table \ref{FullModel}. In the simulation, we focus on two to three covariates since this is enough to provide an impression of how the increase in the size of the model set affects the FPP and FPR.  \\
% latex table generated in R 4.0.0 by xtable 1.8-4 package
% Tue Jan 19 16:32:31 2021
\begin{table}[!h]
\centering
\caption{The total number of models for any given set considering the different number of covariates, with and without restriction that main effects should be present when having interaction effects.} 
\begin{tabular}{lccccc}
  \hline
  & \multicolumn{5}{c}{Number of covariates} \\\cmidrule{2-6}
With restrictions & 2 & 3 & 4 & 5 & 6 \\ 
  \hline
$x + z$ & 4 & 8 & 16 & 32 & 64 \\ 
  $x + z + x \times z$ & 5 & 19 & 65 & 211 & 665 \\ 
  $x + z + z \times z$ & 1 & 10 & 97 & 1418 & 40005 \\ 
  $x + z+ x \times z + z \times z$ & 3 & 58 & 1159 & 36958 & 2269799 \\
  \hline 
  Number of total models & 13 & 95 & 1337 & 38619 & 2310533 \\ 
  \hline \\
  Without restrictions \\ 
  \hline
  $x + z$ & 4 & 8 & 16 & 32 & 64 \\ 
  $x \times z$ & 3 & 7 & 15 & 31 & 63 \\ 
  $z \times z$ & 1 & 4 & 32 & 512 & 16384 \\ 
  $x + z + x \times z$ & 9 & 49 & 225 & 961 & 3969 \\ 
  $x + z + z \times z$ & 3 & 49 & 945 & 31713 & 2064321 \\ 
  $x \times z + z \times z$ & 3 & 49 & 945 & 31713 & 2064321 \\ 
  $x + z + x \times z + z$ \times z & 9 & 343 & 14175 & 983103 & 130052223 \\ 
  \hline
  Number of total models & 32 & 509 & 16353 & 1048065 & 134201345 \\ 
   \hline 
\multicolumn{6}{p{13cm}}{\footnotesize{Note: $x + z$ = models with main effects only; $x \times z$ = models with interactions between the variable of interest and covariates; $z \times z$ = models with interactions between covariates;  $x + z + x \times z$ = models with main effects and interactions between the variable of interest and covariates; $x + z + z \times z$ = models with main effects and interactions between covariates; $x \times z + z \times z$ = models with interactions between covariates and variable of interest and interactions between covariates; $x + z + x \times z + z \times z$ = models with main effects and interactions between the variable of interest and covariates and the interactions between covariates.}} 

\end{tabular}
\end{table}


In Table \ref{modelsets2.1}, can be found an examples of all model sets with one variable of interest, $x_{1}$, and two covariates, $z_{1}$ and $z_{2}$. The calculations for the model sets with no restrictions can be found in Appendix section \nameref{witout_res}.
\begin{table}[!htbp]
\centering
\caption{}
\caption*{\footnotesize An overview of all of the model sets when there are two covariates and one dependent variable with the restriction that main effects should be present when having interactions between variables (column two) and with no restrictions that main effects must follow interactions (column three). The ME set contains all of the main effects; the $ME + X \times Cov$  set includes the main effects and interactions between the variable of interest and the covariates; the $ME + Cov \times Cov$ set contains the main effects and  interactions between the covariates; the $ME + X \times Cov + Cov \times Cov$ set contains the main effects, the interactions between the variable of interest and covariates, and the interactions between the covariates; the $X \times Cov$, $Cov \times Cov$ and $X \times Cov + Cov \times Cov$ sets are empty sets when there is restrictions on our interactions since they only contain the interactions but not the required main effects.}
\label{modelsets2}
\begin{threeparttable}
\scalebox{0.9}{%
\begin{tabular}{ccc}
\hline
Model set & Models with restrictions & Models without restrictions \\ \hline
ME & \begin{tabular}[c]{@{}c@{}}$y=x_1$\\  $y=x_1+z_1$\\  $y=x_1+z_2$ \\  $y=x_1+z_1+z_2$\end{tabular} & \begin{tabular}[c]{@{}c@{}}$y=x_1$\\  $y=x_1+z_1$\\  $y=x_1+z_2$ \\  $y=x_1+z_1+z_2$\end{tabular} \\
$X \times Cov$ & Empty set & \begin{tabular}[c]{@{}c@{}}\rule{0pt}{5ex}$y=x_1+x_1z_1$ \\ $y=x_1+x_1z_2$\\ $y=x_1+x_1z_1+x_1z_2$\end{tabular} \\
$Cov \times Cov$ & Empty set & \rule{0pt}{5ex}$y=x_1+z_1z_2$ \\
$ME + X \times Cov$ & \begin{tabular}[c]{@{}c@{}}$y=x_1+z_1+x_1z_1$  \\ $y=x_1+z_2+x_1z_2$ \\ $y=x_1+z_1+z_2+x_1z_1$  \\ $y=x_1+z_1+z_2+x_1z_2$  \\ $y=x_1+z_1+z_2+x_1z_1+x_1z_2$\end{tabular} & \begin{tabular}[c]{@{}c@{}}\rule{0pt}{5ex}$y=x_1+z_1+x_1z_1$\\ $y=x_1+z_1+x_1z_2$\\ $y=x_1+z_1+x_1z_1+x_1z_2$\\ $y=x_1+z_2+x_1z_1$\\ $y=x_1+z_2+x_1z_2$\\ $y=x_1+z_2+x_1z_1+x_1z_2$\\ $y=x_1+z_1+z_2+x_1z_1$\\ $y=x_1+z_1+z_2+x_1z_2$\\ $y=x_1+z_1+z_2+x_1z_1+x_1z_2$\end{tabular} \\
$ME + Cov \times Cov$ & $y=x_1+z_1+z_2+z_1z_2$ & \begin{tabular}[c]{@{}c@{}}\rule{0pt}{5ex}$y=x_1+z_1+z_1z_2$\\ $y=x_1+z_2+z_1z_2$\\ $y=x_1+z_1+z_2+z_1z_2$\end{tabular} \\
$X \times Cov + Cov \times Cov$ & Empty set & \begin{tabular}[c]{@{}c@{}}\rule{0pt}{5ex}$y=x_1+x_1z_1+z_1z_2$\\ $y=x_1+x_1z_2+z_1z_2$\\ $y=x_1+x_1z_1+x_1z_2+z_1z_2$ \end{tabular} \\  
$ME + X \times Cov + Cov \times Cov$ & \begin{tabular}[c]{@{}c@{}}$y=x_1+z_1+z_2+x_1z_1+z_1z_2$ \\ $y=x_1+z_1+z_2+x_1z_2+z_1z_2$ \\ $y=x_1+z_1+z_2+x_1z_1+x_1z_2+z_1z_2$\end{tabular} & \begin{tabular}[c]{@{}c@{}}\rule{0pt}{5ex}$y=x_1+z_1+x_1z_1+z_1z_2$\\ $y=x_1+z_1+x_1z_2+z_1z_2$\\ $y=x_1+z_1+x_1z_1+x_1z_2+z_1z_2$\\ $y=x_1+z_2+x_1z_1+z_1z_2$\\ $y=x_1+z_2+x_1z_2+z_1z_2$\\ $y=x_1+z_2+x_1z_1+x_1z_2+z_1z_2$\\ $y=x_1+z_1+z_2+x_1z_1+z_1z_2$\\  $y=x_1+z_1+z_2+x_1z_2+z_1z_2$\\ $y=x_1+z_1+z_2+x_1z_1+x_1z_2+z_1z_2$\end{tabular} \\ \hline
\end{tabular}
}%
\end{threeparttable}

\end{table}

 \\
 
\section{Simulation analysis}
Focusing on model selection in data analysis, we investigated the false-positive probability (FPP) and false-positive ratio (FPR) for a key independent variable that was uncorrelated with any other variables using a simulation. We used different number of covariates and interaction effects as well as a varying sample size. The effect of the different model sets were tested using different correlations between the dependent variable and covariates, and different data structures. Even though in some simple cases it would be possible to calculate a closed-form solution for the FPP and FPR e.g., when looking at only one main effect, we used a simulation since it was not clear how different flexibilities would impact the FPP and FPR. In this section it will be described how to split different models into different sets and how to calculate how many models there are within each set\\


\subsection{Data Generating Process}
As the goal of the simulation was to understand the different model sets and what can lead to false-positive results, the variable of interest was simulated without any correlation with either the dependent variable or any of the covariates. The variable of interest could either have a normal distribution (continuous variable) or a binomial distribution (binary variable). The covariates were simulated using the same distribution so that all covariate distributions in a dataset were either normal or binomially distributed. As our target analysis was linear regression, the dependent variable was always simulated as a normal distribution. There were four different combinations of the variable of interest $x$ and covariate $z$ distributions: $x$ and $z$ are continuous, $x$ and $z$ are binary, $x$ is continuous and $z$ is binary, $x$ is binary and $z$ is continuous. The correlation between the dependent variable and covariates was simulated with three different levels (\textit{r} = 0.2, \textit{r} = 0.3, and \textit{r} = 0.4) corresponding to medium-strength correlations \citep{Cohen1989}. The correlation between the variable of interest and covariates as well as the correlation between the covariates themselves was always set to zero. The correlation matrix for the data structure when there is only one dependent variable is presented in Table \ref{tab:correlation} in the \textit{Appendix}. If a variable was generated as a normally distributed variable, it was generated with a mean of 0 and a standard deviation of 1, while binomial variables were generated with a 50\% chance of success in each trial. To test how the sample size affected the FPP and FPR, the simulation included samples with sizes ranging from 50 to 400 with increasing steps of 50. 

\subsection{Simulation}
In the simulation results presented in the main results section, we only included cases in which the variable of interest and covariates had the same distribution. The two other cases i.e., when the variable of interest was binary and the covariates were continuous and the other way around can be found in the online \textit{Supplementary Information}. For analyses not addressing the effect of sample size, the default sample size was kept constant at 200. When the effect of different correlation levels between covariates was not analyzed, the correlation between the dependent variable and covariates was set to a default of $r=0.2$. We did not apply any outlier deletion.\\
We used R \citep{Team2018} to perform our simulation with 10,000 simulation iterations. When the variables had a binomial distribution, the variables were simulated using the package BinNor \citep{Demirtas2014}. The data were analyzed with linear regressions. We defined a false-positive result as any model where the variable of interest was significant ($p < .05$), either in a main effect or in an interaction with a covariate variable. The FPP was defined as \\

\[FPP_i=\left. \left\{\begin{array}{c}
1\ if\ any\ model\ in\ iteration\ i\ produces\ a\ false-positive\ result \\ 
0\ otherwise\  \end{array}
\right.\] 
\[FPP=\frac{\sum_{i}^{N}{FPP_i}}{N}\] 

Where  $N$ denotes the number of iterations in the simulation. The FPR is the ratio of the models with a false-positive result and was defined as \\

\[FPR_i=\frac{\#\ false-positive\ models\ in\ iteration\ i}{\#\ all\ models\ in\ the\ set\ in\ iteration\ i}\] 
\[FPR=\frac{\sum_{i}^{N}{FPR_i}}{N}\] 


% method section
\section{Method}

To investigate our main research question on how often a hypothesis variable with no effect on the dependent variable can become significant, we performed a simulation. We tested this using the discussed flexibilities, i.e. under different model specifications, with and without outlier deletion, with several dependent variables, and with different coding of the binary variable. These were tested using different sample sizes, different correlations between the dependent variable and the covariates, and different types of data structure. In a very simple case in which there would only be one test to perform for each model, the FPP and FPR would be 5\%. In this case, however, we are expanding the model set with several tests and controlling for interactions that are not present. With several tests within each regression and interactions without main effects, it would be expected that the FPP and FPR would no longer be at 5\%. However, how the different tests interact with each other and how the increase in sample size and different correlation structures affect is not obvious. Therefore, we chose to test the FPP and FPR using simulations as calculating the FPP and FPR was not straightforward.

\subsection{Model Set}
To build the model set, we first needed to define the variables being used. We denote the hypothesis variable as h1, all the covariates as X=(x1,x2,..,xn), and the dependent variable as y. To better understand the building of the model sets, the different interactions were split into three sets. The first set only had the main effects including only the hypothesis variable without any covariates (Ma), the second set included models that had interactions between the hypothesis variable and the covariates (HCI), and the last set contained the models that had interactions between the covariates (CCI). An example with two covariates can be seen in Table 1 (excluding the constant and the coefficient in front of each variable for the sake of simplicity). The hypothesis variable was present in all three sets.
To obtain the full set of models, we also needed the combinations of the three sets. There were two possible choices: restricting our sets so the main effect was always present when there was an interaction or allowing for interactions without any main effects. In Table 1, we show the combination of all sets where the main effect is always included provided there is an interaction term. The other case can be seen in Supplementary Information. \\


\begin{table}[]
\caption{}
\caption*{\footnotesize Set of models when there are two covariates and only one dependent variable}
\centering
\begin{tabular}{cc}
\toprule
Model set & Models \\ 
\midrule
\multirow{4}{*}{Ma} & $y=h_1$ \\ & $y=h_1+x_1$ \\ & $y=h_1+x_2$ \\ & $y=h_1+x_1+x_2$  \\ & \\
\multirow{3}{*}{HCI} & $y=h_1+h_1x_1$ \\ & $y=h_1+h_1x_2$ \\ & $y=h_1+h_1x_1+h_1x_2$  \\& \\
CCI & $y=h_1+x_1x_2$ \\ & \\
\multirow{5}{*}{Ma+HCI} & $y=h_1+x_1+h_1x_1$  \\ & $y=h_1+x_2+h_1x_2$  \\& $y=h_1+x_1+x_2+h_1x_1$  \\& $y=h_1+x_1+x_2+h_1x_2$  \\& $y=h_1+x_1+x_2+h_1x_1+h_1x_2$ \\ & \\
Ma+CCI & $y=h_1+x_1+x_2+x_1x_2$ \\ & \\
HCI+CCI & Empty set \\ & \\
\multirow{3}{*}{Ma+HCI+CCI} & $y=h_1+x_1+x_2+h_1x_2+x_1x_2$ \\ & $y=h_1+x_1+x_2+h_1x_2+x_1x_2$ \\ & $y=h_1+x_1+x_2+h_1x_1+h_1x_2+x_1x_2$ \\
\bottomrule
\end{tabular}
\end{table}


If we require that the main effect to follow all interactions, the union of HCI and CCI will always be empty. This would not be the case if we did not have this constraint (see Supplementary Information). After dividing our models into the different sets, it was possible to calculate how many models were in each set and how many possible models existed. These calculations are shown in Equation (1). \\

\begin{equation} 
\begin{aligned}
\#models={} & \underbrace{\left(2^n\right)}_{Ma}+\underbrace{\sum^n_{j=1}{\left(2^j-1\right)\left( \begin{array}{c}
n \\ 
j \end{array}
\right)}}_{Ma+HCI} + \\ 
& \underbrace{\sum^n_{j=2}{\left( \begin{array}{c}
n \\ 
j \end{array}
\right)\left(2^{\frac{j\left(j-1\right)}{2}}-1\right)}}_{Ma+CCI} + \\
& \underbrace{\sum^n_{j=2}{\left( \begin{array}{c}
n \\ 
j \end{array}
\right)\left(2^j-1\right)\left(2^{\frac{j\left(j-1\right)}{2}}-1\right)}}_{Ma+HCI+CCI}\ \  
\end{aligned}
\end{equation} 

n denotes the number of the covariates collected. A more thorough work through of this equation can be found in the supplementary material.
We did not include the sets HCI, CCI, and HCI+CCI as these were interaction terms with no corresponding main effect. For the case, where we had two covariates, the number of models became as follows: \\

$
\begin{aligned}
\centering
\left(2^2-1\right)+
\sum^2_{j=1}{\left(2^j-1\right)\left( \begin{array}{c}
2 \\ 
j \end{array}
\right)}+
\sum^2_{j=2}{\left( \begin{array}{c}
2 \\ 
j \end{array}
\right)\left(2^{\frac{j\left(j-1\right)}{2}}-1\right)} +  
 \sum^2_{j=2}{\left( \begin{array}{c}
2 \\ 
j \end{array}
\right)\left(2^j-1\right)\left(2^{\frac{j\left(j-1\right)}{2}}-1\right)}= \\
 3+5+1+3*1=12 
\end{aligned}
$

The number of models in each set for the different numbers of covariates under the restriction that the main effect must always follow the interactions can be seen in Table 2. The other case can be found in the Supplementary Information. \\

% latex table generated in R 4.0.2 by xtable 1.8-4 package
% Tue Oct 20 10:04:22 2020
\begin{table}[!h]
\centering
\caption{The total number of models for any given set considering the different number of covariates with the restriction that the main effects should always be present when there are the interaction effects.} 
\begin{tabular}{lccccc}
  \hline
Number of covariates & ME & ME+HCI & ME+CCI & ME+HCI+CCI & Number of models \\ 
  \hline
2 & 4 & 5 & 1 & 3 & 13 \\ 
  3 & 8 & 19 & 10 & 58 & 95 \\ 
  4 & 16 & 65 & 97 & 1159 & 1337 \\ 
  5 & 32 & 211 & 1418 & 36958 & 38619 \\ 
  6 & 64 & 665 & 40005 & 2269799 & 2310533 \\ 
   \hline 
{\footnotesize{Note: ME = models with the main effects only; \\ ME + HCI = models with the main effects and interactions \\ between the variable of interest and covariates; \\ ME + CCI = models with the main effects \\ and interactions between covariates; \\ ME + HCI + CCI = models with the main effects and the interactions \\ between the variable of interest and covariates and the interactions between covariates.} 
 \hline
\end{tabular}
\end{table}


In the simulation, the focus was on two to four covariates as this was enough to give us an idea of how the increase in the model set affected the FPP. 

\subsection{Outlier Criteria}
Following the findings of \cite{Leyes2013}, four different outlier criteria were included. Three of the outlier criteria were based on the standard deviation (2, 2.5, and 3), while one used the interquartile range \citep{Rousseeuw2011}. Each outlier criterion was used on the entire dataset and not only on individual variables. If any observation fulfilled the outlier criteria, the observation was omitted from the analysis. As the outlier criteria chosen here only worked for continuous variables, there were cases where only the dependent variable was tested for outliers.

\subsection{Collecting Multiple Dependent Variables}
To test the consequences of collecting several dependent variables, three variables were collected simultaneously. These were correlated with r=0.5. Furthermore, the average of the three dependent variables was computed to yield a fourth dependent variable. With these four dependent variables, the model set would expand four times compared to when there is only one dependent variable. This implies that every time three dependent variables were collected, there were four different regressions to run.

\subsection{Coding of Variables}
For the cases in which either the covariates or the hypothesis variable were binary, these were coded in two different ways. They were either dummy coded with 1/0 or effects coded with 1/-1. These were tested separately to determine if any of the coding would result in a higher FPP and FPR with the other flexibilities.

\subsection{Data Generating Process}
As the goal of the simulation was to determine how often our completely random hypothesis variable could be made significant, this was generated with zero correlation with the dependent variable. The hypothesis variable could have two different distributions: a normal distribution (continuous variable) or a binomial distribution (binary variable). The covariates were simulated using the same distribution, but all the covariates presented in one simulation had the same distribution. As linear regressions were used, the dependent variable could only be normally distributed. Four different cases could be built using these kinds of distributions for the hypothesis variable and the covariates: a continuous hypothesis variable and continuous covariates; a binary hypothesis variable and continuous covariates; a continuous hypothesis variable and binary covariates; and a binary hypothesis variable and binary covariates. If a variable was generated as a binary variable, it was both dummy coded and effects coded.
The correlation between y and X was set to three different levels: r = 0.2; r = 0.3; and r = 0.4, which denote medium-strength correlations \citep{Cohen1989}. The correlation between the hypothesis variable and all other variables was set to zero. The correlations between the covariates were also set to zero to ensure that the results are not driven by omitted variable bias. If any variable was not included, it would only be reflected in the error term, and this would still be uncorrelated with the rest of the variables. The correlation matrix for when there was only one dependent variable is presented in Table 3. \\

\begin{table}
\caption{}
\centering
\caption*{\footnotesize Correlation of simulated data with only one dependent variable, r = $\{$.2,.3,.4$\}$}
\begin{tabular}{c|ccccccc} 
\toprule
 & y & h${}_{1}$ & x${}_{1}$ & x${}_{2}$ & x${}_{3}$ & ... & x${}_{n}$ \\ 
 \midrule
y & 1 &  &  &  &  &  &  \\ 
h${}_{1}$ & 0 & 1 &  &  &  &  &  \\ 
x${}_{1}$ & r & 0 & 1 &  &  &  &  \\  
x${}_{2}$ & r & 0 & 0 & 1 &  &  &  \\  
x${}_{3}$ & r & 0 & 0 & 0 & 1 &  &  \\  
... & r & 0 & 0 & 0 & 0 & 1 &  \\ 
x${}_{n}$ & r & 0 & 0 & 0 & 0 & 0 & 1 \\ 
\bottomrule
\end{tabular}
\end{table}


If a variable was generated as a normally distributed variable, it was generated with a mean of 0 and a standard deviation of 1, while binomial variables were generated with a 50\% chance of success in each trial. To test how the sample size affected the FPP and FPR, the simulation included samples with sizes ranging from 100 to 300 with increasing steps of 50. 

\subsection{Simulation}
We investigated the different flexibilities one at the time. This implies that we started with the different model sets to determine the FPP and FPR in these and thereafter looked what happend when more flexibilities were added. With the post-hoc inclusion of hypothesis and interactions used by researchers, we focused on how the different flexibilities affected each of the model sets. For the data structure, we only included cases in which the hypothesis variable and covariates were continuous and where these were binary. The results for the two other cases looked identical to the cases when using three or more covariates as the results were mainly driven by the distribution of the covariates. The results for the other two cases can be found in the supplementary material for the paper. When the effect of increasing samples was not investigated, the sample was set to 200 to ensure a reasanable sample compared to the amount of variables. When the effect of different correlation levels was not tested, the correlation between the dependent variable and covariates was set to \textit{r}=0.2.\\

A code for the different kinds of flexibilities along with the simulation was programmed in R \citep{Team2018}. When the covariates had a binomial distribution, the variables were simulated using the package BinNor \citep{Demirtas2014}. This was done to ensure that the correlation matrix would also hold under this data type. The data were analyzed with a simple t-test using linear regressions. Every time h1 became significant, either by itself or in an interaction with another variable, it was considered a “success". The iteration number was 10,000. The FPP was defined as \\

\[FPP_i=\left. \left\{\begin{array}{c}
1\ if\ any\ h_1\ is\ significant \\ 
0\ if\ no\ h_1\ is\ significant\  \end{array}
\right.\] 
\[FPP=\frac{\sum^{10000}_i{FPP_i}}{10000}\] 


Here FPP${}_{i}$ indicates if any of the models in the model set had a significant result. If that is the case, it takes the value 1, otherwise 0. FPR is the ratio of the models with a significant result and can be written as \\

\[FPR_i=\frac{\#models\ with\ significant\ result}{\#models\ in\ the\ model\ set}\] 
\[FPR=\frac{\sum^{10000}_i{FPR_i}}{10000}\] 


The full code for the simulation and figures can be found in the supplementary materials. 



% results section
\section{Results} 
Under the right conditions, the false-positive probability was 100\%. There were several ways to obtain a result like this. In general the covariates needed to be binary and dummy coded and allowing for interactions without main effects. In this case, this can be obtained just with two covariates and a big enough sample (see Figure 1B) or a smaller sample and then using extra covariates (see supplementary information). In the same sets the FPR was as high as 35\%. Meaning in the sets where there was a 100\% chance to find a significant result around 35\% of the models had a significant main effect or interaction affect with the hypothesis variable. 

\subsection{Model set}
When investigating the model set we kept everything else constant. This means that we did not include different outlier criteria, only used one dependent variable, the binary variables was dummy coded, and used a sample of 200 observations. The FPP and FPR for all sets can be seen in Figure 1A.
By splitting the FPP and FPR between the different sets, it was clear that the highest FPP and FPR could be found in the sets HCI and Ma + HCI. This was the case both when the constitutive terms were present and when they were not. When they were not present and the covariates were binary with dummy coding, the high FPP and FPR occurred because the interaction was only splitting the sample, and the main effect from the covariate was captured in the interaction. The FPP for a sample of 200 was 83.4 \% for the set HCI and 86.9\% for Ma + HCI. When the variables were continuous and still no main effects following the interaction, it was 18.9\% and 24.3\%, respectively. When we have the restriction that main effects following the interaction, then the set HCI does no longer exist, but we can compare with the set Ma + HCI. The FPP was 20.6\% for binary variables and 18.6\% for continuous variables. Across all sets the FPP for having two binary variables was 87.2\%, and for continuous, it was 24.7\% when we have no restrictions on the main effect and 22\% and 18.5\% respectively when we need the main effect with all interactions. 
It was not only the FPP that increased under the different conditions as the number of models in the model set with a significant hypothesis variable also increased (see red bars in Figure 1A). In general, the number of these models was around the expected 5\%, but when interactions between the hypothesis variable and the covariates were allowed, the FPR increased. The percent of models with a significant hypothesis variable or interaction with the hypothesis variable was 31.9\% for the set Ma + HCI when constitutive terms are not needed and using binary data. One way to remedy this was by coding the binary variables using effects coding. Here, the FPP falls to 20.6\%, and FPR in the set Ma + HCI was 9.7\%, the same level as if the covariates were continuous (see Figure 1ASI in Supplementary information). In general, the sets where HCI was included had a higher number of models with a significant effect. 
Adding one more covariate (such that the analysis included three) to the analysis just increased the FPP across all model sets where it is still possible to get a higher FPP (see Figure 1D). This increase was highest for the binary data and where there was no requirement that the main effect should follow the interaction. Several of the sets was here increased to a rate of just below a 100\% FPP. This increase was also seen when we restrict the sets such that the main effects should always be present. Here the growth of the FPP was 14.3\% for the set Ma + HCI + CCI with binary data and 13.1\% for continuous data in the same set. Also the FPR increased. This increase however was only for sets where there was interactions between the hypothesis variable and the covariates. Even though there was a higher increase when there was no restrictions that the main effect should be present, there was still a significant increase when the models were correctly specified. Just adding one more covariate increased the FPR for Ma + HCI + CCI with around 3\%. 

% plot of main analysis
\begin{figure}[t]
\includegraphics[width=0.6\textwidth]{R/Analysis/Result/Figures/Figure1A.jpeg}
\centering
\caption{Black is the FPP and red is FPR.  Splitting the false-positive rate by model set and whether the main effects should follow interactions or not. If Main=False we had no restrictions. Sample size set at 200, a correlation between the dependent variable and covariates set at r=0.2 and using two covariates.}
\label{fig:mainfigure}
\end{figure}

\subsection{Outlier criteria}
For the outlier deletion, we were interested in how this affected across the different model sets as well. Therefore, the added effect of using the four different outlier criteria compared to not using them was plotted. Again, the sample was set to 200, and all binary variables where dummy coded. In Figure 1C, the added effect of using the outlier criteria can be seen. Outlier deletion had a different added effect on the different model sets and data structures. The main contribution was within the sets where there were interactions between the covariates. Overall, the added effect to the FPP was between 4\% and 19.4\% for model sets where there was still room for an added effect. The lower added effects seen in some cases came from the fact that the FPP were already close to 100\%. This was the case for the sets where there was an interaction between the hypothesis variable and the covariates and where we allowed for models that did not have the main effect present. Using outlier criteria did not affect the FPR. 

\subsection{Multiple dependent variables}
When looking at multiple dependent variables, we again looked at the different sets of the models. We did not include outlier deletion and all binary variables were dummy coded. Collecting multiple dependent variables and using their averages increased the FPP. This was a general result overall data structures, model sets, and if main effects were included or not. The effect of using three dependent variables, and the average can be seen in Figure 3SI in supplementary information. This increase was highest for sets where there were interactions between the hypothesis variable and the covariates, no matter the data structure and other requirements. For these sets the increase in the FPP was around 15\%. The increase in the false-positive rate seems to be mainly be driven by the increase in the number of models, as the FPR did not increase for any sets. 

\subsection{Sample size and correlation}
Increasing the sample size and thereby the precision of the estimates did not seem to lower the false-positive probability. And even worse, when the constitutive terms were not included, the sample size increased the FPP (See Figure 1B). In this case, the FPP went just under 100\% when the sample increased. The larger sample did not only increase the FPP, but the FPR also increased. The FPR got as high as 41\% for the set HCI with binary data and no restrictions. This means that in these sets where the FPR was 41\% just running one model from these sets would give a FPP at 41\%. 
A higher correlation between the dependent variable and the covariates also increased the false-positive rate for some sets. In general, when the correlation was higher, the FPP was higher when we did not require that the main effect followed the interactions. The sets with the highest increase were then the sets that contained HCI in some way. The effect was greatest when the data was binary and dummy coded (see Supplementary information). The increase in the FPP for these sets was between 15\% and 23\% when the correlation increased from r=0.2 to r=0.3. When the correlation increased from r=0.3 to r=0.4 it was only the sets where there was still room for a higher FPP that increased as a high number of the sets had a FPP of just below 100\% (see Figure 1SI). With this increase in the correlation also increased the FPR for the same sets. This increase was as high as 10\%. When it was required that the main effect should always be present, the FPP and FPR seemed to be stable as the correlation increased (See Figure 1SI).

% plot of main analysis
\begin{figure}[t]
\includegraphics[width=0.6\textwidth]{R/Analysis/Result/Figures/Figure1.jpeg}
\centering
\caption{Black is the FPP and red is FPR. A: Splitting the false-positive rate by model set and whether the main effects should follow interactions or not. If Main=False we had no restrictions. Sample size set at 200, a correlation between the dependent variable and covariates set at r=0.2 and using two covariates. B: The effect of increasing sample size. Correlation between the dependent variable and covariates set at r=0.2 and using two covariates. C: Added effect of using the four outlier criteria compared to what is seen in Figure 1A. Otherwise the same restrictions. D: Added effect of using one more covariate compared to Figure 1A. Otherwise the same restrictions.}
\label{fig:mainfigure}
\end{figure}

% discussion section
\section{Discussion}
There is an increasing awareness among scientists that we need to  lower the number of false positive findings to increase the replicability of research. One way to reduce false positive findings is by preregistering a single data plan for pre-processing and analysis since this removes researcher flexibilities that can otherwise inflate the risk of false positives \citep{Simmons2018}. However, even with a plan for pre-processing and analysis there is still a risk that the FPP may be higher than 5\%. Simulation research shows that type I error rates can increase for several reasons such as model misspecification, noise or ceiling effects in independent variables, or dichotomization of continuous covariates \citep{Dennis2019, Litiere2007, Brunner2009, Austin2003, Austin2004}. To better understand the consequences of exploratory analysis where multiple testing can take place, and how it would look if a study was preregistered such that only one test can be made, we used a simulation to examine both how the FPP and FPR were affected by different flexibilities. Specifically, we looked into the following flexibilities: constructing a model set with several covariates, allowing for interactions with and without main effects, using multiple outlier criteria, and collecting multiple dependent variables. We also examined how the precision of the estimates (a larger sample) and the correlations between the dependent variable and the covariates affected the ability to find a significant effect. \\

When we talk about analysis where we use FPP and FPR as measurement for the consequences of flexibilities, we can interpret these in different ways. The FPP tells us what is the probability that a researcher will get a false-positive result if he was to test all models and flexibilities. This kind of measurement is important when we talk about exploratory analysis where there is no clear plan for what should be analysed. Whereas the FPR tells us the chance of a false-positive result if the researcher is not sure about the model or what outlier criteria to use and then by random pick one of them in a given set and only test that one. This means that the FPR result can be seen as risk of a  false-positive result for a researcher that preregister his analysis plan where it in many cases are not clear beforehand hat outlier criteria to use or how the different variables should enter the model (i.e., only as main effects or as interactions).
It should be mentioned that when we talk about preregistration here, we talk about a preregistration in a perfect form (if one like this exists), meaning that it is clear what is being analysed, what the variables of interest are and what variables one needs to control for. However, this is not always clear and these decisions can have an influence on the outcome of the analysis \citep{Bryan25535,gilbert2016comment}.\\ 

In many cases the FPR lies in the expected neighborhood of around 5\% which is an indicator that preregistration is very helpful lowering the risk of obtaining a false-positive result. However, not in all cases is the preregistration itself sufficient to lower the FPR to the desired 5\%. In particular, the preregistration does not help when the main effects are not included in the model and the covariates are dummy coded. In this case, the FPR can reach 42\%, meaning that on average 42\% of the models in a given set contain a significant effect (either the interaction or the main effect). This happens due to the fact that the interaction effect captures the true effect that would have been in the main effect of the covariate. So even though some literature still suggests that one can exclude the main effect when having interactions as long as it is theory based, there seems to be no good reason for this when the covariates are dummy coded. When the variables are effects coded, the FPR decreases, but it is still above 5\%. The only way to overcome this issue is to always include the main effects when the interaction effects are of interest. However, even then, the FPR will still be slightly higher than 5\%. This is due to the fact that multiple tests are performed at the same time. A simple solution to overcome this issue is to correct the p-values by  using a correction such as Bonferroni correction \citep{dunn1961multiple} (see Figure S6).
 \\

Given that it is possible to obtain an FPP of 100\% even when preregistering one's research plan, labeling research that goes beyond the main analysis as exploratory can have little to no meaning if using p-values. Even when performing correct analyses such as including all the main effects when doing interactions and correcting p-values to follow the number of tests in a given model, the FPP is still higher than 5\%. P-values are therefore not a good indicator for determining the presence of an effect if the analysis is exploratory. This is not to say that exploratory research does not have its value, but using p-values might not be a good indicator for if an exploratory analysis is of interest. \\
    
It has been suggested that one way to lower the FPP would be to ensure having a well-powered study, that is enough observations per cell to detect a true effect \citep{Simmons2011, simmons2018}. This is, however, not an effective way to lower the FPP or FPR since, in combination with bad practices (e.g., leaving out main effects when using interactions), larger samples can produce higher rates of the FPP. This may seem counter-intuitive; however, the reader should bear in mind that in this article we investigated how often a true null effect can become significant. In a situation where there is a true effect, a bigger sample is indeed needed as small samples can produce exaggerated effects or effects in the opposite direction to the true direction \citep{gelman2014beyond}. We do not recommend to run studies with small sample sizes, but simply remind that just having a big enough sample does not mean that the FPP and FPR will be bound to 5\%.\\

There are, of course, factors that are not under the control of the researcher, such as the correlation between the covariates and the dependent variable. But it is still worth a mention, that the higher the correlation between the covariates and the dependent variable, the easier it is to find a significant model. This is the case for both exploratory analyses and for a preregistered analysis. However, it is less clear how to solve this problem. Study designs with common method bias (e.g., measuring all covariates in one survey) could be particularly problematic due to  the increased correlation between the covariates \citep{podsakoff2003}. A potential remedy could be to report correlation matrices which would allow readers to assess the inflated risk of false positive findings.  \\ 

To sum up, using several researcher flexibilities increases the risk of finding a model with a significant effect even though no true effect is present. However, preregistration does not seem to be enough to bring the FPP and FPR down to the desired level of 5\%. We therefore made the following recommendations to reduce the probability of getting a false positive result. 

\subsection{Guidelines for researchers}

\subsubsection{Follow the principles of open science and use preregistration}
This reduces the number of researcher flexibilities and is therefore the most important step towards reducing the FPP. Researchers should therefore follow the already developed guidelines \citep{Nosek2015}. This means that research should be preregistered and all raw data and analyses code should be made available for other researchers to make it easier to see if any mistakes or mentioned flexibilities occurred during the analysis. 
\subsubsection{Use Bonferroni corrections when looking at interactions.}
A Bonferroni correction should be used when running regressions with interaction effects. If this is not done, the FPR will be higher than 5\%, and therefore the FPP would also be higher than 5\%. In some cases the correction can lower the FPR to less than 5\%, but we believe that this is a better outcome than having one that is higher than 5\%. 
\subsubsection{Large samples do not protect against FPP and FPR}
Having a large enough sample does not legitimize exploratory analyses. In general, the sample size only affects the power of the study and the precision of the estimates, it does not remedy the FPR and FPP. 

\subsection{Guidelines for reviewers and editors}

\subsubsection{Look for correction of p-values}
If multiple comparisons were made or interaction effects were used, there should be a correction of the p-values.
\subsubsection{Require that main effects always follow interaction effects}
A reviewer should never accept an interaction effect to be true if the main effects are not present and the variables are binary and dummy coded. 
\subsection{Exploratory analysis}
If exploratory analysis have been made, there should never be made any conclusions on these if the measurement of effect is p-values. Exploratory analysis still have a place within research, but concluding on these is a bad idea.  

\subsection{Additional contributions}
The size of a given model set and its permutations may not only be of interest to the open science community. Our description of model sets could also be of interest to researchers that use machine learning as a model selection tool. Since researchers who work with machine learning rarely care about one specific variable but rather about the predictability of a general model, using the sets where HCI is included is not always meaningful. Instead, what matters is whether the researcher will allow for interactions between the independent variables. The calculations of the model sets are therefore the same as if there was a variable of interest, but excluding the set that has HCI in it. This suggests that considering the requirement that the main effect should follow along with the interactions, the sets of interest are ME and ME + CCI. With no such requirements, the sets of interest are Ma, CCI, and ME + CCI. The number of models with and without restrictions, different number of variables, and how to calculate them can be found in Table SI9 and Table SI10. 


 


% author contributions
%\section{Author contributions}
%JDC and JLO developed the study concept. JDC and SP performed the analyses. JDC, SP, CJL, and JLO wrote the manuscript. 

% data and code
\section{Data availability}
All code for simulation and figures in the paper can be found at \url{https://osf.io/tcpx4/}.

% reference list
\bibliographystyle{apalike}
\bibliography{references.bib}
\clearpage

% appendix
 \part*{Supplementary Material} 

\setcounter{table}{0}
\setcounter{figure}{0}
\renewcommand{\thetable}{SI\arabic{table}}
\renewcommand{\thefigure}{SI\arabic{figure}}

\subsection{Formula for Set Size}
To calculate the size of the different sets, we will use the binomial coefficients to calculate the possible combinations. The general form of the binominal coefficient is written as
\[\left( \begin{array}{c}
n \\ 
k \end{array}
\right)=\frac{n!}{k!\left(n-k\right)!}\] 

Where ``!'' is the factorial operator. So $n!=n\times \left(n-1\right)\times \left(n-2\right)\times \left(n-3\right)\times \dots \times 3\times 2\times 1$. One rule that is being used over and over again is the fact that we can rewrite the sum of binomial coefficients excluding the empty set (i.e. i=0) as 
\[\sum^n_{i=1}{\left( \begin{array}{c}
n \\ 
i \end{array}
\right)}=2^n-1\] 

When we need to calculate the size of each set, we need to split them into two different cases; one where we allow interactions without having main terms present, and the other where main terms should always follow the interaction. \\

\section{Case 1 – No Restriction}
\subsection{Ma}

To calculate the size of the sets, we can use the binomial coefficient. Since the size of the set Ma will be all the combinations of the covariates. Let us denote the number of covariates with n. If we have an example with three covariates meaning n=3, then the combination of these where we have all three present can be written with the binomial coefficient as
\[\left( \begin{array}{c}
3 \\ 
3 \end{array}
\right)=\frac{3!}{3!\left(3-3\right)!}=1\] 
If only two covariates are to be present at the same time then the possible combinations become 
\[\left( \begin{array}{c}
3 \\ 
2 \end{array}
\right)=\frac{3!}{2!\left(3-2\right)!}=3\] 
And the same if only one should be present
\[\left( \begin{array}{c}
3 \\ 
1 \end{array}
\right)=\frac{3!}{1!\left(3-1\right)!}=3\] 
The final one is the one where no covariates are present
\[\left( \begin{array}{c}
3 \\ 
0 \end{array}
\right)=\frac{3!}{0!\left(3-0\right)!}=1\] 
Then the number of models in this set will be the sum of all these


\[\#models\ in\ Ma=\left( \begin{array}{c}
3 \\ 
3 \end{array}
\right)+\left( \begin{array}{c}
3 \\ 
2 \end{array}
\right)+\left( \begin{array}{c}
3 \\ 
1 \end{array}
\right)+\left( \begin{array}{c}
3 \\ 
0 \end{array}
\right)=\sum^3_{i=0}{\left( \begin{array}{c}
3 \\ 
i \end{array}
\right)}\] 
Using the rules from binomial coefficients, we can rewrite this into
\[\#models\ in\ Ma=\sum^3_{i=0}{\left( \begin{array}{c}
3 \\ 
i \end{array}
\right)}=\left(2^3\right)=8\] 
We can generalize this to the n covariates case with 
\[\#models\ in\ Ma=\sum^n_{i=1}{\left( \begin{array}{c}
n \\ 
i \end{array}
\right)}=\left(2^n\right)\] 

\subsubsection{Formally Written} \break 
\noindent Let X be the set of covariates 
\[X=\left.x_1,x_2,..,x_n\right.\] 
\[\left|X\right|=n\] 
\[\left|\mathcal{P}\left(X\right)\right|=2^n\] 

\noindent The set Ma is then the power set of H excluding the empty set
\[Ma=\left.S:S\subseteq X\right.\] 

Each element $S\in Ma$ is a set of the covariates e.g. $S=\left.x_3,x_7\right.$ or any combination of the covariates, including the empty set
\[\left|Ma\right|=2^n\] 
\subsection{HCI}

The same line of reasoning goes for the set HCI. Here, it is all the interactions between the covariates and the hypothesis variable 

\[I_h(X)=\left.\left.x_i,h\right.:x_i\in X\right.\] 
\[HCI=\left.T:T\subseteq I_h\left(X\right),T\neq \textrm{\O}\right.\] 
The number of models in this set is only the power set excluding the empty set
\[\left|HCI\right|\boldsymbol{=}2^n-1\] 

\subsection{CCI}

For the set CCI, we need the combinations of the combinations. As this is the combination of the two-way interaction of the covariates. The first thing we need is the number of combinations there is of the covariates. If we stick to the case with n=3, then the amount of combinations is 
\[\#\ of\ two-way\ interactions=\left( \begin{array}{c}
3 \\ 
2 \end{array}
\right)=\frac{3!}{2!\left(3-2\right)!}=3\] 
Then we can do the same as for the Ma and HCI set, but in this, our starting point becomes the number of two-way interactions. 

\noindent 
\[\#models\ in\ CCI=\left( \begin{array}{c}
3 \\ 
3 \end{array}
\right)+\left( \begin{array}{c}
3 \\ 
2 \end{array}
\right)+\left( \begin{array}{c}
3 \\ 
1 \end{array}
\right)=\sum^3_{i=1}{\left( \begin{array}{c}
3 \\ 
i \end{array}
\right)}=7\] 
In the general case, let us define the number of two-way interactions with k; this gives us
\[\left( \begin{array}{c}
n \\ 
2 \end{array}
\right)=n(n-1)/2\] 
So, the general term for the number of models in CCI is then
\[\#models\ in\ CCI=\sum^{n(n-1)/2}_{i=1}{\left( \begin{array}{c}
n(n-1)/2 \\ 
i \end{array}
\right)}=(2^{n(n-1)/2}-1)\] 
Since we have no restrictions on the combinations of these sets, the combination of these are only the sets multiplied with each other. Collecting all this, the number in the different sets for the general case becomes

\subsubsection{Formally written}
First step is to make the set of all the combinations of the covariates
\[I_k\left(X\right)=\left.\left.x_i,x_j\right.:x_i\in X,x_j\in X,x_i\neq x_j\right.\] 
\[\left|I_k\left(X\right)\right|=\left( \begin{array}{c}
n \\ 
2 \end{array}
\right)=\frac{n\left(n-1\right)\left(n-2\right)!}{\left(n-2\right)!}\frac{1}{2}=n(n-1)/2\] 
The set of all combinations is then all the subsets of $I_k\left(X\right)$ excluding the empty set
\[P\left(I_k\left(X\right)\right)=\left.J:J\subseteq I_k\left(X\right),J\neq \textrm{\O}\right.\] 
\[\left|CCI\right|=\left|P\left(I_k\left(X\right)\right)\right|=2^{\left|I_k\left(X\right)\right|}-1=2^{n(n-1)/2}-1\] 
In the previous part, $\left( \begin{array}{c}
n \\ 
2 \end{array}
\right)$ was just defined as k, so we can write this as
\[\left|CCI\right|=2^k-1\] \\

\subsection{Full model set}
For the final parts, the math follows directly as this is just the product of the sets calculated here. 


\[\#models\ in\ Ma=\left(2^n\right)\] 
\[\#models\ in\ HCI=\left(2^n-1\right)\] 
\[\#models\ in\ CCI=\left(2^k-1\right)\] 
\[\#models\ in\ Ma+HCI={\left(2^n-1\right)}^2\] 
\[\#models\ in\ Ma+CCI=\left(2^n-1\right)\left(2^k-1\right)\] 
\[\#models\ in\ Ma+HCI+CCI={\left(2^n-1\right)}^2\left(2^k-1\right)\] 
So, the number of models then becomes
\begin{equation*}
\begin{aligned}
\#model=\\
& \underbrace{\left(2^n\right)}_{Ma}+\underbrace{\left(2^n-1\right)}_{HCI}+\underbrace{\left(2^k-1\right)}_{CCI}+\\
&\underbrace{{\left(2^n-1\right)}^2}_{Ma+HCI}+\underbrace{\left(2^n-1\right)\left(2^k-1\right)}_{HCI+CCI}+\underbrace{\left(2^n-1\right)\left(2^k-1\right)}_{Ma+CCI}+\\
&\underbrace{{\left(2^n-1\right)}^2\left(2^k-1\right)}_{Ma+HCI+CCI} 
\end{aligned}
\end{equation*} \\

For the case where n=2, (as in the Table SI1) this gives us
\[k=\left( \begin{array}{c}
2 \\ 
2 \end{array}
\right)=1\] 
\[\#model=\left(2^2\right)+\left(2^2-1\right)+\left(2^1-1\right)+{\left(2^2-1\right)}^2+2\left(2^2-1\right)\left(2^1-1\right)+{\left(2^2-1\right)}^2\left(2^1-1\right)=32\] 
\eject 



\begin{table}[]
\caption{}
\caption*{\footnotesize Set of models when there are two covariates and only one dependent variable}
\centering
\begin{tabular}{cc}
\toprule
Model set & Models \\ 
\midrule
\multirow{4}{*}{Ma} & $y=h_1$ \\ & $y=h_1+x_1$ \\ & $y=h_1+x_2$ \\ & $y=h_1+x_1+x_2$ & \\ 
\multirow{3}{*}{HCI} & $y=h_1+h_1x_1$ \\ & $y=h_1+h_1x_2$ \\ & $y=h_1+h_1x_1+h_1x_2$ & \\
CCI & $y=h_1+x_1x_2$ & \\ 
\multirow{9}{*}{Ma+HCI} & $y=h_1+x_1+h_1x_1$\\ & $y=h_1+x_1+h_1x_2$\\ & $y=h_1+x_1+h_1x_1+h_1x_2$\\ & $y=h_1+x_2+h_1x_1$\\ & $y=h_1+x_2+h_1x_2$\\ & $y=h_1+x_2+h_1x_1+h_1x_2$\\ & $y=h_1+x_1+x_2+h_1x_1$\\ & $y=h_1+x_1+x_2+h_1x_2$\\ & $y=h_1+x_1+x_2+h_1x_1+h_1x_2$ & \\ 
\multirow{3}{*}{Ma+CCI} & $y=h_1+x_1+x_1x_2$\\ & $y=h_1+x_2+x_1x_2$\\ & $y=h_1+x_1+x_2+x_1x_2$ & \\
\multirow{3}{*}{HCI+CCI} & $y=h_1+h_1x_1+x_1x_2$\\ & $y=h_1+h_1x_2+x_1x_2$\\ & $y=h_1+h_1x_1+h_1x_2+x_1x_2$ & \\
\multirow{9}{*}{Ma+HCI+CCI} & $y=h_1+x_1+h_1x_1+x_1x_2$\\ & $y=h_1+x_1+h_1x_2+x_1x_2$\\ & $y=h_1+x_1+h_1x_1+h_1x_2+x_1x_2$\\ & $y=h_1+x_2+h_1x_1+x_1x_2$\\ & $y=h_1+x_2+h_1x_2+x_1x_2$\\ & $y=h_1+x_2+h_1x_1+h_1x_2+x_1x_2$\\ & $y=h_1+x_1+x_2+h_1x_1+x_1x_2$\\ & $y=h_1+x_1+x_2+h_1x_2+x_1x_2$\\ & $y=h_1+x_1+x_2+h_1x_1+h_1x_2+x_1x_2$ \\ 
\bottomrule
\end{tabular}
\end{table}



\section{Case 2 – With Restriction}

In this case, we need to ensure that any time there is an interaction, the main terms follow. This means that if our hypothesis variable interacts with a covariate, both the hypothesis variable and the covariate need to enter the model linearly as well, e.g.,
\[y=h_1+x_1+h_1x_1\] 

Here, we have an interaction between our hypothesis variable h${}_{1\ }$and a covariate x${}_{1}$${}_{.}$ Therefore, the terms also enter the model by themselves. This restriction does not change how we calculate the set size of Ma as there are no interactions here. The way to calculate the size of this is therefore the same in the two cases, we will therefore focus on the other relevant sets. However, it means that we cannot count the models that lie in the sets HCI and CCI as these do not have the main effect following the interactions. The first set we then need to calculate is the set Ma + HCI.

\subsection{Ma + HCI}
For the sake of simplicity, we start with a case where n=3. The models that are within this set can be seen in Table SI2 in the case where n=3 \\

\begin{table}[]
\caption{}
\caption*{\footnotesize Set of models when there are two covariates and only one dependent variable}
\centering
\begin{tabular}{cc}
\toprule
Model set & Models \\ 
\midrule
\multirow{19}{*}{Ma+HCI} & $y=h_1+x_1+h_1x_1$\\ &  $y=h_1+x_2+h_1x_2$\\ &  $y=h_1+x_3+h_1x_3$\\ & $y=h_1+x_1+x_2+h_1x_1$\\ & $y=h_1+x_1+x_3+h_1x_1$\\ & $y=h_1+x_3+x_1+h_1x_3$\\ & $y=h_1+x_3+x_2+h_1x_3$\\ & $y=h_1+x_2+x_1+h_1x_2$\\ & $y=h_1+x_2+x_3+h_1x_2$\\ & $y=h_1+x_1+x_2+x_3+h_1x_1$\\ & $y=h_1+x_1+x_2+x_3+h_1x_2$\\ & $y=h_1+x_1+x_2+x_3+h_1x_3$\\ & $y=h_1+x_1+x_2+h_1x_1+h_1x_2$\\ & $y=h_1+x_1+x_3+h_1x_1+h_1x_3$\\ & $y=h_1+x_1+x_2+h_1x_3+h_1x_2$\\ & $y=h_1+x_1+x_2+x_3+h_1x_1+h_1x_2$\\ & $y=h_1+x_1+x_2+x_3+h_1x_1+h_1x_3$\\ & $y=h_1+x_1+x_2+x_3+h_1x_3+h_1x_2$\\ & $y=h_1+x_1+x_2+x_3+h_1x_1+h_1x_2+h_1x_3$\\  
\bottomrule
\end{tabular}
\end{table}


We can split this set into three parts. One where there is only one main effect and one where there is two main effects and the final one with three main effects as shown in Table SI3.\\


\begin{table}
\caption{}
\centering
\begin{tabular}{lc}  
\toprule
Set & Models \\
\midrule
\multirow{3}{*}{Ma+HCI(1)} & $y=h_1+x_1+h_1x_1$\\ & $y=h_1+x_2+h_1x_2$\\ & $y=h_1+x_3+h_1x_3$\\ & \\ 
\multirow{9}{*}{Ma+HCI(2)} & $y=h_1+x_1$$+x_2+h_1x_1$\\ & $y=h_1+x_1+x_3+h_1x_1$\\ & $y=h_1+x_3+x_1+h_1x_3$\\ & $y=h_1+x_3+x_2+h_1x_3$\\ & $y=h_1+x_2+x_1+h_1x_2$\\ & $y=h_1+x_2+x_3+h_1x_2$\\ & $y=h_1+x_1+x_2+h_1x_1+h_1x_2$\\ & $y=h_1+x_1+x_3+h_1x_1+h_1x_3$\\ & $y=h_1+x_1+x_2+h_1x_3+h_1x_2$\\  & \\  
\multirow{7}{*}{Ma+HCI(3)} & $y=h_1+x_1+x_2+x_3+h_1x_1$\\ & $y=h_1+x_1+x_2+x_3+h_1x_2$\\ & $y=h_1+x_1+x_2+x_3+h_1x_3$\\ & $y=h_1+x_1+x_2+x_3+h_1x_1+h_1x_2$\\ & $y=h_1+x_1+x_2+x_3+h_1x_1+h_1x_3$\\ & $y=h_1+x_1+x_2+x_3+h_1x_3+h_1x_2$\\ & $y=h_1+x_1+x_2+x_3+h_1x_1+h_1x_2+h_1x_3$\\  
\bottomrule
\end{tabular}
\end{table}
\\

We can then calculate the size of each of these sets separately. Let us start with Ma+HCI (1). Here the number of models is the product of how many combinations that can be made with one variable (so this is 1) and how many interactions that can be made. Using the same shorthand notation with the binomial coefficient, we can write this one as
\[\#\ of\ models\ in\ Ma+HCI\ \left(1\right)=\left( \begin{array}{c}
1 \\ 
1 \end{array}
\right)\left( \begin{array}{c}
3 \\ 
1 \end{array}
\right)=\sum^1_{i=1}{\left( \begin{array}{c}
1 \\ 
i \end{array}
\right)}\left( \begin{array}{c}
3 \\ 
1 \end{array}
\right)=\left(2^1-1\right)\left( \begin{array}{c}
3 \\ 
1 \end{array}
\right)=3\] 

This way of writing up the formula makes sense when we have to put them all together.

For the set Ma+HCI(2), the first part we need is the number of combinations of the main effect, and we can make when we need to have two at all time. This is just $\left( \begin{array}{c}
3 \\ 2 \end{array}\right)$. The next thing is to figure out is the combinations of the interaction. As we have two covariates present all the time, these are the ones that can be used in the interaction term. There can either be one or two of these interactions present then as only one is used in the interaction at the time. So, we have $\left( \begin{array}{c}
2 \\ 
1 \end{array}
\right)\ $and $\left( \begin{array}{c}
2 \\ 
2 \end{array}
\right)$ combinations of the interactions possible. Combining this give us
\begin{equation*}
\begin{aligned}
\#\ of\ models\ in\ Ma+HCI\ \left(2\right)=\\
& \left( \begin{array}{c}
2 \\ 
1 \end{array}
\right)\left( \begin{array}{c}
3 \\ 
2 \end{array}
\right)+\left( \begin{array}{c}
2 \\ 
2 \end{array}
\right)\left( \begin{array}{c}
3 \\ 
2 \end{array}
\right)=\\
&\sum^2_{i=1}{\left( \begin{array}{c}
2 \\ 
i \end{array}
\right)}\left( \begin{array}{c}
3 \\ 
2 \end{array}
\right)=\left(2^2-1\right)\left( \begin{array}{c}
3 \\ 
2 \end{array}
\right)=9

\end{aligned}
\end{equation*}
The same reasoning goes for Ma + HCI(3); here we just have one more as we can also have all three interaction terms in the model at the same time. So, if we write this is in same way as before we get
\begin{equation*}
\begin{aligned}
\#\ of\ models\ in\ Ma+HCI\ \left(3\right)=\\
&\left( \begin{array}{c}
3 \\ 
1 \end{array}
\right)\left( \begin{array}{c}
3 \\ 
3 \end{array}
\right)+\left( \begin{array}{c}
3 \\ 
2 \end{array}
\right)\left( \begin{array}{c}
3 \\ 
3 \end{array}
\right)+\left( \begin{array}{c}
3 \\ 
3 \end{array}
\right)\left( \begin{array}{c}
3 \\ 
3 \end{array}
\right)= \\
&\sum^3_{i=1}{\left( \begin{array}{c}
2 \\ 
i \end{array}
\right)}\left( \begin{array}{c}
3 \\ 
2 \end{array}
\right)= \\
&\left(2^3-1\right)\left( \begin{array}{c}
3 \\ 
3 \end{array}
\right)=7
\end{aligned}
\end{equation*}


We can then sum all these together to get the size of the set
\[\#\ of\ models\ in\ Ma+HCI=\sum^3_{k=1}{(2^k-1)\left( \begin{array}{c}
3 \\ 
k \end{array}
\right)}\] 
For the case with n covariates, this becomes
\[\#\ of\ models\ in\ Ma+HCI=\sum^n_{k=1}{(2^k-1)\left( \begin{array}{c}
n \\ 
k \end{array}
\right)}\] 

\subsection{Ma + CCI}

Let J be the set of indicators of how many interactions are between the covariates and the hypothesis variable 
\[J=\left.1,2,3,4,\dots ,n\right.\] 
Let L be a subset of X of size k.
\[L=\left.S:S\subset X,\left|S\right|=k,k\in J\right.\] 

The number of interactions that can be made between a covariate and the hypothesis variable for a given k is then
\[I_h\left(X\right)=\left.\left.x_i,h\right.:x_i\in L\right.\] 
\[\left|I_h\left(X\right)\right|=k\] 

For each k, the number of interactions is just the power set of X excluding the empty set
\[\left|\mathcal{P}\left(I_h\left(X\right):I_h\left(X\right)\neq \textrm{\O}\right)\right|=2^k-1\] 

For each k, there is $\left( \begin{array}{c}
n \\ 
k \end{array}
\right)$ combinations of the covariates following the interaction. So, to get the full set, we just need to sum over these from k=1 to n.
\[\left|Ma+HCI\right|=\sum^n_{k=1}{\left( \begin{array}{c}
n \\ 
k \end{array}
\right)\left(2^k-1\right)}\] 
\textbf{}

\subsection{Ma + CCI}

We can use the same line of thought as we did in the Ma + HCI set. Let us begin by looking at the n=3 case and then develop a more general notion from there. This example can be seen in Table SI4 
\begin{table}
\centering
\caption{}
\begin{tabular}{lc} \hline 
\toprule
Set & Models \\
\midrule
\multirow{9}{*}{Ma+CCI} & $y=h_1+x_1+x_2+x_1x_2$\\ & $y=h_1+x_1+x_3+x_1x_3$\\ & $y=h_1+x_2+x_3+x_2x_3$\\ & $y=h_1+x_1+x_2+x_3+x_1x_2$\\ & $y=h_1+x_1+x_3+x_2+x_1x_3$\\ & $y=h_1+x_2+x_3+x_1+x_2x_3$\\ & $y=h_1+x_1+x_2+x_3+x_1x_2+x_1x_3$\\ & $y=h_1+x_1+x_3+x_2+x_1x_3+x_2x_3$\\ & $y=h_1+x_2+x_3+x_1+x_1x_2+x_2x_3$ \\
\bottomrule
\end{tabular}
\end{table}
We can split this set as before. This can be seen in Table SI5

\begin{table}
\centering
\caption{}
\begin{tabular}{lc} 
\toprule
Set & Models \\ 
\midrule
\multirow{3}{*}{Ma+CCI(2)} & $y=h_1+x_1+x_2+x_1x_2$\\ & $y=h_1+x_1+x_3+x_1x_3$\\ & $y=h_1+x_2+x_3+x_2x_3$\\ &  \\  
\multirow{7}{*}{Ma+CCI(3)} & $y=h_1+x_1+x_2+x_3+x_1x_2$\\ & $y=h_1+x_1+x_3+x_2+x_1x_3$\\ & $y=h_1+x_2+x_3+x_1+x_2x_3$\\ & $y=h_1+x_1+x_2+x_3+x_1x_2+x_1x_3$\\ & $y=h_1+x_1+x_3+x_2+x_1x_3+x_2x_3$\\ & $y=h_1+x_2+x_3+x_1+x_1x_2+x_2x_3$\\ & $y=h_1+x_1+x_2+x_3+x_1x_2+x_2x_3+x_1x_3$\\ & \\ 
\bottomrule
\end{tabular}
\end{table}

As in this case, we always need to have at least two covariates for this set to fulfil the requirement that we have set; we can only split out Ma+CCI into two parts; the first part has two covariates and the second part has three. The first set Ma+CCI(2) is straightforward as this set is only the number of combinations of the three covariates that can be made multiplied with the combinations of two of these covariates that are included

\noindent 
\[\#\ of\ models\ in\ Ma+CCI\ \left(2\right)=\left( \begin{array}{c}
3 \\ 
2 \end{array}
\right)\left( \begin{array}{c}
2 \\ 
2 \end{array}
\right)=3\] 

For the second part, the first term will be the number of combinations of three that can be made with three covariates; so this is just $\left( \begin{array}{c}
3 \\ 
3 \end{array}
\right)$. We then have to figure out the number of combinations that can be made with the number of covariate interaction terms. The first thing is to figure out the number of two-way interactions that can be made with three covariates. This is just $\left( \begin{array}{c}
3 \\ 
2 \end{array}
\right)=3$. The number of models in the set is then the sum over the amount of combinations we include
\[\#\ of\ models\ in\ Ma+CCI\ \left(3\right)=\left( \begin{array}{c}
3 \\ 
3 \end{array}
\right)\left(\left( \begin{array}{c}
3 \\ 
1 \end{array}
\right)+\left( \begin{array}{c}
3 \\ 
2 \end{array}
\right)+\left( \begin{array}{c}
3 \\ 
3 \end{array}
\right)\right)=\left( \begin{array}{c}
3 \\ 
3 \end{array}
\right)\sum^3_{i=1}{\left( \begin{array}{c}
3 \\ 
i \end{array}
\right)}\] 

However, it is only in this special case that we can just sum to our n=3. In general, this will be the number of pairs that can be made in this subset. So, here the number of pairs will be $\left( \begin{array}{c}
n \\ 
2 \end{array}
\right)=\frac{n\left(n-1\right)}{2}$. This then becomes
\[\#of models in Ma+CCI\ \left(3\right)=\left( \begin{array}{c}
3 \\ 
3 \end{array}
\right)\sum^3_{i=1}{\left( \begin{array}{c}
3 \\ 
i \end{array}
\right)}=\left( \begin{array}{c}
3 \\ 
3 \end{array}
\right)\sum^{\frac{3\left(3-1\right)}{2}}_{i=1}{\left( \begin{array}{c}
\frac{3\left(3-1\right)}{2} \\ 
i \end{array}
\right)}=\left( \begin{array}{c}
3 \\ 
3 \end{array}
\right)\left(2^{\frac{3\left(3-1\right)}{2}}-1\right)\] 
We can write the first subset in the same way
\[\#\ of\ models\ in\ Ma+CCI\ \left(2\right)=\left( \begin{array}{c}
3 \\ 
2 \end{array}
\right)=\left( \begin{array}{c}
3 \\ 
2 \end{array}
\right)\left(2^{\frac{2\left(2-1\right)}{2}}-1\right)\] 
Now we just put all this together
\[\#\ of\ models\ in\ Ma+CCI=\left( \begin{array}{c}
3 \\ 
2 \end{array}
\right)\left(2^{\frac{2\left(2-1\right)}{2}}-1\right)+\left( \begin{array}{c}
3 \\ 
3 \end{array}
\right)\left(2^{\frac{3\left(3-1\right)}{2}}-1\right)=\sum^3_{j=2}{\left( \begin{array}{c}
3 \\ 
j \end{array}
\right)\left(2^{\frac{j\left(j-1\right)}{2}}-1\right)}\] 
We can then generalize this case into n covariates
\[\#\ of\ models\ in\ Ma+CCI=\sum^n_{j=2}{\left( \begin{array}{c}
n \\ 
j \end{array}
\right)\left(2^{\frac{j\left(j-1\right)}{2}}-1\right)}\] 
The last part that we need to calculate the complete set is Ma + HCI + CCI

\subsubsection{Formally written}

Here, we are here using the same notation and definitions for the sets and used in the case without restrictions. 

Let k be an indicator of how many covariates that are included. Since we are using interactions, k must start at two 
\[T=\left.2,3,4,\dots ,n\right.\] 
Let L be a subset of X of size k.
\[L=\left.S:S\subset X,\left|S\right|=k,k\in T\right.\] 

\noindent The number of interactions that can be made for a given k is then
\[I_k\left(X\right)=\left.\left.x_i,x_j\right.:x_i\in L,x_j\in L,x_i\neq x_j\right.\] 
For a given k, the number of two-way interactions that can be made is then
\[\left|I_k\left(X\right)\right|=\left( \begin{array}{c}
k \\ 
2 \end{array}
\right)=\frac{k!}{2!\left(k-2\right)!}=\frac{k\left(k-1\right)\left(k-2\right)!}{\left(k-2\right)!}\frac{1}{2}=\frac{k\left(k-1\right)}{2}\] 

We then need the combination of all the two-way interactions. For all k, we can take the power set of $I_k\left(H\right)$ excluding the empty set
\[P\left(I_k\left(X\right)\right)=\left.T:T\subset I_k\left(X\right),T\neq \textrm{\O}\right.\] 
\[\left|P\left(I_k\left(X\right)\right)\right|=2^{\left|I_k\left(X\right)\right|}-1=2^{\frac{k\left(k-1\right)}{2}}-1\] 
With n covariates, there exist $\left( \begin{array}{c}
n \\ 
k \end{array}
\right)$ subsets of H of size k. To get the set Ma + CCI, we just need to sum over these combinations
\[\left|Ma+CCI\right|=\sum^n_{k=2}{\left( \begin{array}{c}
n \\ 
k \end{array}
\right)}\left(2^{\frac{k\left(k-1\right)}{2}}-1\right)\ \] 

\subsection{Ma + HCI + CCI}
This one is a bit easier to work with. This is the product of Ma + HCI and Ma + CCI, but without taking the combinations with the main effect twice. So, it is the combinations that can be made between the covariates and between the covariates and the hypothesis variable. We can therefore just write this one as 
\[\#\ of\ models\ in\ Ma+HCI+\ CCI=\sum^n_{k=2}{\left( \begin{array}{c}
n \\ 
k \end{array}
\right)\left(2^k-1\right)\left(2^{\frac{k\left(k-1\right)}{2}}-1\right)}\] 
\subsubsection{Formally written}

Given the sets $P\left(I_k\left(X\right)\right)$ and $P\left(I_h\left(X\right)\right)$. To get the combinations of all these given a specific k we just take the product. For n covariates there exist $\left( \begin{array}{c}
n \\ 
k \end{array}
\right)$ subsets of H of size k. We only need to sum over the subsets
\[Ma+HCI+\ CCI=\sum^n_{k=2}{\left( \begin{array}{c}
n \\ 
k \end{array}
\right)\left(2^k-1\right)\left(2^{\frac{k\left(k-1\right)}{2}}-1\right)}\] 
We now have the full set of models in the n covariates case
\[\#\ of\ models=\left(2^n\right)+\sum^n_{k=1}{\left(2^k-1\right)\left( \begin{array}{c}
n \\ 
k \end{array}
\right)}+\sum^n_{k=2}{\left( \begin{array}{c}
n \\ 
k \end{array}
\right)\left(2^{\frac{k\left(k-1\right)}{2}}-1\right)}+\sum^n_{k=2}{\left( \begin{array}{c}
n \\ 
k \end{array}
\right)\left(2^k-1\right)\left(2^{\frac{k\left(k-1\right)}{2}}-1\right)}\] 
\subsection{Number of Models}

\noindent In the where we allow interaction with no main effect (case 1 here), the number of models can be seen in Table SI6. For case 2, the number of models can be found in either Table 2 in the paper or Table SI7. Table SI 6

% latex table generated in R 4.0.0 by xtable 1.8-4 package
% Tue Jan 19 16:10:56 2021
\begin{table}[!h]
\centering
\caption{The total number of models for any given set considering the different number of covariates and with no restriction that main effects should be present when having interaction effects.} 
\scalebox{0.8}{
\begin{tabular}{lccccc}
  \hline
 & 2 & 3 & 4 & 5 & 6 \\ 
  \hline
ME & 4 & 8 & 16 & 32 & 64 \\ 
  X * Cov & 3 & 7 & 15 & 31 & 63 \\ 
  Cov * Cov & 1 & 4 & 32 & 512 & 16384 \\ 
  ME + X * Cov & 9 & 49 & 225 & 961 & 3969 \\ 
  ME + Cov * Cov & 3 & 49 & 945 & 31713 & 2064321 \\ 
  X * Cov + Cov * Cov & 3 & 49 & 945 & 31713 & 2064321 \\ 
  ME + X * Cov + Cov * Cov & 9 & 343 & 14175 & 983103 & 130052223 \\ 
  Number of total models & 32 & 509 & 16353 & 1048065 & 134201345 \\ 
   \hline 
\multicolumn{6}{p{12cm}}{\footnotesize{Note: ME = models with main effects only; X * Cov = models with interactions between the variable of interest and covariates; Cov * Cov = models with interactions between covariates;  ME + X * Cov = models with main effects and interactions between the variable of interest and covariates; ME + Cov * Cov = models with main effects and interactions between covariates; X * Cov + Cov * Cov = models with interactions between covariates and variable of interest and interactions between covariates; ME + X * Cov + Cov * Cov = models with main effects and interactions between the variable of interest and covariates and the interactions between covariates.}} 
 \hline
\end{tabular}
}
\end{table}


\noindent The number of models split into the different sets and number of covariates when there is no restriction on the main effect.

% latex table generated in R 4.0.2 by xtable 1.8-4 package
% Tue Oct 20 10:04:22 2020
\begin{table}[!h]
\centering
\caption{The total number of models for any given set considering the different number of covariates with the restriction that the main effects should always be present when there are the interaction effects.} 
\begin{tabular}{lccccc}
  \hline
Number of covariates & ME & ME+HCI & ME+CCI & ME+HCI+CCI & Number of models \\ 
  \hline
2 & 4 & 5 & 1 & 3 & 13 \\ 
  3 & 8 & 19 & 10 & 58 & 95 \\ 
  4 & 16 & 65 & 97 & 1159 & 1337 \\ 
  5 & 32 & 211 & 1418 & 36958 & 38619 \\ 
  6 & 64 & 665 & 40005 & 2269799 & 2310533 \\ 
   \hline 
{\footnotesize{Note: ME = models with the main effects only; \\ ME + HCI = models with the main effects and interactions \\ between the variable of interest and covariates; \\ ME + CCI = models with the main effects \\ and interactions between covariates; \\ ME + HCI + CCI = models with the main effects and the interactions \\ between the variable of interest and covariates and the interactions between covariates.} 
 \hline
\end{tabular}
\end{table}


\noindent Number of models split into the different sets and number of covariates when we need constitutive terms.


\section{Model set for machine learning}
When a researcher is not interested in one specific variable but rather the overall prediction of a given model, such as in machine learning, the same type of sets can be used to calculate the size of the model sets. 
As is with the case of hypothesis testing the researcher can either demand that the main effect are always present when having interaction or it can loosen on this assumption and not require this. \\

\subsection{When no restrictions are in place.}
In this case there are three different sets; Ma, CCI and Ma + CCI. In this case however there is no variable that needs to be present all the time as earlier ($h_1$). The calculations of the sets are however simelar as an example with two covarietes can be seen in Table SI8.  

\begin{table}
\caption{}
\centering
\begin{tabular}{lc} 
\toprule
Set & Models \\ 
\midrule
\multirow{4}{*}{Ma} & $y=c$\\ & $y=x_1$\\ & $y=x_2$\\ & $y=x_1+x_2$\\ &  \\  
\multirow{1}{*}{CCI} & $y=x_1x_2$\\  & \\ 
\multirow{3}{*}{Ma + CCI}  & $y=x_1 + x_1x_2$\\ & $y=x_2 + x_1x_2$\\ & $y=x_1+x_2 + x_1x_2$\\ &  \\  
\bottomrule
\end{tabular}
\end{table}

The calcultation for the set size follows along as it did earlier and is therefore
\\
\[\#models\ in\ Ma=\left(2^n\right)\] 
\[\#models\ in\ CCI=\left(2^k-1\right)\] 
\[\#models\ in\ Ma+CCI=\left(2^n-1\right)\left(2^k-1\right)\] 
where $k=\left( \begin{array}{c}
n \\ 
2 \end{array}
\right)$. 

Given different number of variables the number of possible models can be seen in Table SI9. These are when only allowing for two-way interactions, but this framework can easliy be expanded to include higher order interactions. \\

% latex table generated in R 4.0.0 by xtable 1.8-4 package
% Wed Dec 30 11:38:53 2020
\begin{table}[!h]
\centering
\caption{} 
\begin{tabular}{rrrrr}
  \hline
Number of covariates & ME & CCI & ME+CCI & Number of models \\ 
  \hline
2 & 4 & 1 & 3 & 8 \\ 
  3 & 8 & 4 & 49 & 61 \\ 
  4 & 16 & 32 & 945 & 993 \\ 
  5 & 32 & 512 & 31713 & 32257 \\ 
  6 & 64 & 16384 & 2064321 & 2080769 \\ 
   \hline
\end{tabular}
\end{table}



\subsection{When restrictions are in place.}
This follows along the calculations of earlier. But here the only sets that are of interest is Ma and Ma + CCI.\\
The number of models in this case is then \\

\[\#models\ in\ Ma=\left(2^n\right)\] 
\[\#models\ in\ Ma+CCI=\sum^n_{k=2}{\left( \begin{array}{c}
n \\ 
k \end{array}
\right)}\left(2^{\frac{k\left(k-1\right)}{2}}-1\right)\ \]  

The number of models with different numbers of variables can be seen in Table SI10

% latex table generated in R 4.0.2 by xtable 1.8-4 package
% Tue Oct 20 10:24:03 2020
\begin{table}[!h]
\centering
\caption{} 
\begin{tabular}{rrrr}
  \hline
Number of covariates & ME & ME+CCI & Number of models \\ 
  \hline
2 & 4 & 1 & 5 \\ 
  3 & 8 & 10 & 18 \\ 
  4 & 16 & 97 & 113 \\ 
  5 & 32 & 1418 & 1450 \\ 
  6 & 64 & 40005 & 40069 \\ 
   \hline
\end{tabular}
\end{table}


\section{Further results from the simulation}
FPP and FPR for full model sets under different condetions. Here there is no split between each model set


% latex table generated in R 4.0.0 by xtable 1.8-4 package
% Wed Feb 03 14:17:11 2021
\begin{longtable}{llrr}
\caption{False positive probability (FPP) and false positive ratio (FPR) when looking at all the models possible when the sample size is 200, no outlier criteria is being used and having two covariates. When restrictions on interactions are on main effects should always be present when there is interactions, this is not the case when restrictions on interactions is off.} \\ 
  \hline
Restrictions on interactions & Type & FPP & FPR \\ 
  \hline
Without & Normal & 0.25 & 0.10 \\ 
  Without & Binomial & 0.87 & 0.23 \\ 
  With & Normal & 0.19 & 0.08 \\ 
  With & Binomial & 0.21 & 0.08 \\ 
   \hline
\hline
\end{longtable}



\begin{figure}[t]
\includegraphics[width=0.8\textwidth]{R/Analysis/Result/Figures/Figure1BF.jpeg}
\centering
\caption{Black is the FPP and red is FPR.  Splitting the false-positive rate by model set and whether the main effects should follow interactions or not. Same figures as Figure 1-4 but with Bonferroni correction}
\label{fig:mainfigure}
\end{figure}

\begin{figure}[t]
\includegraphics[width=0.8\textwidth]{R/Analysis/Result/Figures/Figure1BF.jpeg}
\centering
\caption{Black is the FPP and red is FPR.  Splitting the false-positive rate by model set and whether the main effects should follow interactions or not. Same figures as Figure 1-4 but with Bonferroni correction}
\label{fig:mainfigure}
\end{figure}
 
% supplementary information
 \section{Supplementary Information} 
\setcounter{table}{0}
\setcounter{figure}{0}
\pagenumbering{arabic}
\renewcommand*{\thepage}{\arabic{page}}
\renewcommand{\thetable}{S\arabic{table}}
\renewcommand{\thefigure}{S\arabic{figure}}

In the paper "Preregistration is important, but not enough" we examine the consequences of two different cases of the data structure; where both the variable of interest $x$ and the covariates $z$ have the same distribution i.e. both binary or both normally distributed. However, there are also the cases where one is binary and the other is normally distributed. The result for these cases can be found in this \textit{Supplementary Information} in the section "\nameref{result}". The first figure, \ref{fig:appfigure1}, contains the baseline model. This is the figure that all the other figures are compared to to see the increase of the false-positive probability (FPP) or false-positive ratio (FPR) for any of the flexibilities investigated. For this figure the sample size is set to 200, the correlation between the dependent variable and the covariates is $\textit{r}=0.2$, two covariates are used and no outlier criteria are used. For each of the figures in this section one of these flexibilities are changed. Figure \ref{fig:appfigure2} illustrates the consequences of a increase in the correlation between the dependent variable and the covariates, Figure \ref{fig:appfigure3} illustrates what happens when using one more dependent variable and the average between these two, Figure \ref{fig:appfigure4} illustrates the consequences of using several outlier criteria, Figure \ref{fig:appfigure5} illustrates the increase in FPP and FPR when using three instead of two covariates and finally in Figure \ref{fig:appfigure6} shows what happens to the FPP and FPR when increasing the sample size.  \\

In the second part of this \textit{Supplementary Information}, in the section "\nameref{resultBC}", is the same set of figures, but in this case the critical value has been corrected using Bonferroni correction. The figures follow the same order as the first section, first a baseline model with Bonferroni correction in Figure \ref{fig:appfigure7} and afterwards the effect of changing one of the flexibilities in Figure \ref{fig:appfigure8} to Figure \ref{fig:appfigure12}. \\
In the final part of the \textit{Supplementary Information}, in the section "\nameref{overviewtable}", full tables of all the results from the simulation can be found. Here are  results both with and without Bonferroni correction.\\
\newpage

\subsection{Results without correcting p-values}
The following supporting figures ($S1 - S6$) show additional conditions that influence the FPP and FPR when p-values are not corrected.

\label{result}
\begin{figure}[ht!]
%\figuretitle{Baseline model with all the different data structures}
\includegraphics[width=1\textwidth]{R/Analysis/Result/Figures/Figure1ASI.jpeg}
\centering
\caption{The false-positive probability and false-positive ratio given different model sets, the presence of main effects when having interactions (i.e., Without restrictions or With restrictions) and different distributions of the variable of interest and covariates. This includes the two other cases compared to the main paper i.e., when the variable of interest is binary and the covariates are continuous and the other way around. Sample size is set to 200, a correlation between the dependent variable and covariates is $\textit{r}=0.2$ and there are two covariates. The false-positive probability is shown in black and the false-positive ratio in red. Dashed blacked line indicates 0.05. This figure adds the two other cases not shown in Figure \ref{fig:mainfigure1}.}
\label{fig:appfigure1}
\end{figure}

\begin{landscape}
\begin{figure}[ht!]
%\figuretitle{Effect of increasing the correlation between the dependent variable and the covariates}
\includegraphics[scale=0.65]{R/Analysis/Result/Figures/Figure2SI.jpeg}
\centering
\caption{False-positive probability and false-positive for different levels of correlation between the dependent variable and the covariates ranging from  $\textit{r}=0.2$ to  $\textit{r}=0.4$. Black denotes the false-positive probability and red denotes the false-positive ratio. Dashed blacked line indicates 0.05. The description of the figure is otherwise the same as for Figure \ref{fig:appfigure1}.}
\label{fig:appfigure2}
\end{figure}
\end{landscape}


\begin{figure}[ht!]
%\figuretitle{Effect of using several dependent variables}
\includegraphics[width=0.8\textwidth]{R/Analysis/Result/Figures/Figure3SI.jpeg}
\centering
\caption{False-positive probability and false-positive ratio when using two dependent variables and the average of the two (three dependent variables in total). The correlation between the dependent variables is set to  $\textit{r}=0.5$ with the correlation between the dependent variables and covariates still at  $\textit{r}=0.2$. Black denotes the the false-positive probability and red denotes the false-positive ratio. Dashed blacked line indicates 0.05. The description of the figure is otherwise the same as for Figure \ref{fig:appfigure1}.}
\label{fig:appfigure3}
\end{figure}


\begin{figure}[ht!]
%\figuretitle{Effect of using multiple outlier criteria}
\includegraphics[width=1\textwidth]{R/Analysis/Result/Figures/Figure1BSI.jpeg}
\centering
\caption{False-positive probability and false-positive ratio when using multiple outlier criteria. Black denotes the the false-positive probability and red denotes the false-positive ratio. Dashed blacked line indicates 0.05. The description of the figure is otherwise the same as for Figure \ref{fig:appfigure1}. This figure adds the two other cases not shown in Figure \ref{fig:mainfigure3}.
}
\label{fig:appfigure4}
\end{figure}

\begin{figure}[ht!]
%\figuretitle{Effect of using three covariates}
\includegraphics[width=1\textwidth]{R/Analysis/Result/Figures/Figure1CSI.jpeg}
\centering
\caption{False-positive probability and false-positive ratio when using three covariates instead of two as in the base line model. Black denotes the the false-positive probability and red denotes the false-positive ratio. Dashed blacked line indicates 0.05. The description of the figure is otherwise the same as for Figure \ref{fig:appfigure1}. This figure adds the two other cases not shown in Figure \ref{fig:mainfigure2}.
}
\label{fig:appfigure5}
\end{figure}

\begin{landscape}
\begin{figure}[ht!]
%\figuretitle{Effect of increasing the sample size}
\includegraphics[scale=0.75]{R/Analysis/Result/Figures/Figure1DSI.jpeg}
\centering
\caption{Effect of increasing sample size for each model set and all combinations of data distributions (i.e., binary and normal). Black denotes the false-positive probability and red denotes the false-positive ratio. Dashed blacked line indicates 0.05. The description of the figure is otherwise the same as for Figure \ref{fig:appfigure1}. This figure adds the two other cases not shown in Figure \ref{fig:mainfigure4}}
\label{fig:appfigure6}
\end{figure}
\end{landscape}

\clearpage
\subsection{Results when using Bonferroni correction of p-values}
\label{resultBC}

In this section, we present the same analyses, but with Bonferroni correction of p-values. This means that the critical value ($\alpha$) now has a correction such that $\alpha = 0.05 / #Number~of~test~in~each~model$. So the critical value will only be 0.05 in the case where there are no interactions between the variable of interest and the covariates. The critical value is corrected for each test such that it can differ within each model set depending on how many interactions there are per model. 

Table \ref{tab:resultFullBC} shows the FPP and FPR for the full model sets under different conditions with Bonferroni correction. Here there is no split between each model set. The correction lowers the FPR to the expected 5\%, but there is still an issue when not including main effects and having binomial variables. 

% latex table generated in R 4.0.0 by xtable 1.8-4 package
% Wed Feb 03 16:21:31 2021
\begin{longtable}{llrr}
\caption{False positive probability (FPP) and false positive ratio (FPR) with Bonferroni correction when looking at all the models possible when the sample size is 200, no outlier criteria is being used and having two covariates. When restrictions on interactions are on main effects should always be present when there is interactions, this is not the case when restrictions on interactions is off.} \\ 
  \hline
Restrictions on interactions & Type & FPP & FPR \\ 
  \hline
Without & Normal & 0.17 & 0.05 \\ 
  Without & Binomial & 0.76 & 0.16 \\ 
  With & Normal & 0.13 & 0.05 \\ 
  With & Binomial & 0.14 & 0.05 \\ 
   \hline
\hline
\label{tab:resultFullBC}
\end{longtable}


\begin{figure}[ht!]
%\figuretitle{Baseline model with all the different data structures when using Bonferroni correction}
\includegraphics[width=1\textwidth]{R/Analysis/Result/Figures/Figure1ASIBon.jpeg}
\centering
\caption{The description of this figure is the same as for Figure \ref{fig:appfigure1} with the only difference being that we applied Bonferroni correction. This figure therefore shows the false-positive probability and false-positive ratio given different model sets, the presence of main effects when having interactions (i.e., Without restrictions or With restrictions) and different distributions of the variable of interest and covariates. Sample size was set to 200, a correlation between the dependent variable and covariates was $\textit{r}=0.2$ and there are two covariates. The false-positive probability is shown in black and the false-positive ratio in red. Dashed blacked line indicates 0.05. }
\label{fig:appfigure7}
\end{figure}

\begin{landscape}
\begin{figure}[ht!]
%\figuretitle{Effect of increasing the correlation between the dependent variable and the covariates when using Bonferroni correction}
\includegraphics[scale=0.65]{R/Analysis/Result/Figures/Figure2SIBon.jpeg}
\centering
\caption{False-positive probability and false-positive ratio for different levels of correlation between the dependent variable and the covariates ranging from  $\textit{r}=0.2$ to  $\textit{r}=0.4$ when using Bonferroni correction. Black denotes the false-positive probability and red denotes the false-positive ratio. Dashed blacked line indicates 0.05. The description of the figure is otherwise the same as for Figure \ref{fig:appfigure7}.}
\label{fig:appfigure8}
\end{figure}
\end{landscape}



\begin{figure}[ht!]
%\figuretitle{Effect of using several dependent variables when using Bonferroni correction}
\includegraphics[width=0.8\textwidth]{R/Analysis/Result/Figures/Figure3SIBon.jpeg}
\centering
\caption{False-positive probability and false-positive ratio when using two dependent variables and the average of the two (three dependent variables in total) when using Bonferroni correction. The correlation between the dependent variables is set to  $\textit{r}=0.5$ with the correlation between the dependent variables and covariates still at  $\textit{r}=0.2$. Black denotes the false-positive probability and red denotes the false-positive ratio. Dashed blacked line indicates 0.05. The description of the figure is otherwise the same as for Figure \ref{fig:appfigure7}.}
\label{fig:appfigure9}
\end{figure}


\begin{figure}[ht!]
%\figuretitle{Effect of using multiple outlier criteria when using Bonferroni correction}
\includegraphics[width=1\textwidth]{R/Analysis/Result/Figures/Figure1BSIBon.jpeg}
\centering
\caption{False-positive probability and false-positive ratio when using multiple outlier criteria when using Bonferroni correction. Black denotes the the false-positive probability and red denotes the false-positive ratio. Dashed blacked line indicates 0.05. The description of the figure is otherwise the same as for Figure \ref{fig:appfigure7}.
}
\label{fig:appfigure10}
\end{figure}


\begin{figure}[ht!]
%\figuretitle{Effect of using three covariates when using Bonferroni correction}
\includegraphics[width=1\textwidth]{R/Analysis/Result/Figures/Figure1CSIBon.jpeg}
\centering
\caption{False-positive probability and false-positive ratio when using three covariates instead of two as in the base line model and using Bonferroni correction. Black denotes the the false-positive probability and red denotes the false-positive ratio. Dashed blacked line indicates 0.05. The description of the figure is otherwise the same as for Figure \ref{fig:appfigure7}.
}
\label{fig:appfigure11}
\end{figure}


\begin{landscape}
\begin{figure}[ht!]
%\figuretitle{Effect of increasing the sample size when using Bonferroni correction}
\includegraphics[scale=0.75]{R/Analysis/Result/Figures/Figure1DSIBon.jpeg}
\centering
\caption{Effect of increasing sample size for each model set and all combinations of data distributions (i.e., binary and normal) when using Bonferroni correction. Black denotes the false-positive probability and red denotes the false-positive ratio. Dashed blacked line indicates 0.05. The description of the figure is otherwise the same as for Figure \ref{fig:appfigure1}.}
\label{fig:appfigure12}
\end{figure}
\end{landscape}

\subsection{Overview tables}
\label{overviewtable}
Readers interested in examining specific results from the simulation can consult the following overview tables. Table \ref{tab:apptabFull} and Table \ref{tab:apptabFullBC} contain the results for the full simulation of the different sets. Table \ref{tab:apptabFull} contains all the results when there is no correction of the p-value and \ref{tab:apptabFullBC} contains the results when Bonferroni correction is used. For each set the result of the false-positive probability (FPP) and false-positive ratio (FPR) is shown with and without the restriction that main effects should always follow an interaction and for the different other flexibilities used in the paper i.e. increasing the number of covariates, increasing the number of dependent variables, using several outlier criteria. This is shown with different data structures for the variable of interest and covariates. 
\begin{landscape}
\scriptsize
% latex table generated in R 4.0.0 by xtable 1.8-4 package
% Thu Feb 04 12:26:31 2021
\begin{longtable}{lccccccccc}
\caption{False positive probability (FPP) and false positive ratio (FPR) for the different model sets} \\ 
  \hline
Restrictions & Set & Type & Sample Size & Outlier exclusion & Correlation & Covariates & Dependent variables & FPP & FPR \\ 
  \hline
 & $\textit{x} + \textit{z}$ & $\textit{x} \sim Normal , \textit{z} \sim Normal$ & 200 & FALSE & 0.20 & 2.00 & 1.00 & 0.07 & 0.05 \\ 
   & $\textit{x} + \textit{z}$ & $\textit{x} \sim Normal , \textit{z} \sim Normal$ & 200 & FALSE & 0.20 & 3.00 & 1.00 & 0.08 & 0.05 \\ 
   & $\textit{x} + \textit{z}$ & $\textit{x} \sim Binary, \textit{z} \sim Binary$ & 200 & FALSE & 0.20 & 2.00 & 1.00 & 0.07 & 0.05 \\ 
   & $\textit{x} + \textit{z}$ & $\textit{x} \sim Binary, \textit{z} \sim Binary$ & 200 & FALSE & 0.20 & 3.00 & 1.00 & 0.08 & 0.05 \\ 
   & $\textit{x} + \textit{z}$ & $\textit{x} \sim Normal, \textit{z} \sim Binary$ & 200 & FALSE & 0.20 & 2.00 & 1.00 & 0.07 & 0.05 \\ 
   & $\textit{x} + \textit{z}$ & $\textit{x} \sim Normal, \textit{z} \sim Binary$ & 200 & FALSE & 0.20 & 3.00 & 1.00 & 0.08 & 0.05 \\ 
   & $\textit{x} + \textit{z}$ & $\textit{x} \sim Binary, \textit{z} \sim Normal$ & 200 & FALSE & 0.20 & 2.00 & 1.00 & 0.07 & 0.05 \\ 
   & $\textit{x} + \textit{z}$ & $\textit{x} \sim Binary, \textit{z} \sim Normal$ & 200 & FALSE & 0.20 & 3.00 & 1.00 & 0.08 & 0.05 \\ 
   & $\textit{x} + \textit{z}$ & $\textit{x} \sim Normal , \textit{z} \sim Normal$ & 200 & FALSE & 0.20 & 2.00 & 1.00 & 0.07 & 0.05 \\ 
   & $\textit{x} + \textit{z}$ & $\textit{x} \sim Normal , \textit{z} \sim Normal$ & 200 & FALSE & 0.20 & 3.00 & 1.00 & 0.08 & 0.05 \\ 
   & $\textit{x} + \textit{z}$ & $\textit{x} \sim Binary, \textit{z} \sim Binary$ & 200 & FALSE & 0.20 & 2.00 & 1.00 & 0.07 & 0.05 \\ 
   & $\textit{x} + \textit{z}$ & $\textit{x} \sim Binary, \textit{z} \sim Binary$ & 200 & FALSE & 0.20 & 3.00 & 1.00 & 0.08 & 0.05 \\ 
   & $\textit{x} + \textit{z}$ & $\textit{x} \sim Normal, \textit{z} \sim Binary$ & 200 & FALSE & 0.20 & 2.00 & 1.00 & 0.07 & 0.05 \\ 
   & $\textit{x} + \textit{z}$ & $\textit{x} \sim Normal, \textit{z} \sim Binary$ & 200 & FALSE & 0.20 & 3.00 & 1.00 & 0.08 & 0.05 \\ 
   & $\textit{x} + \textit{z}$ & $\textit{x} \sim Binary, \textit{z} \sim Normal$ & 200 & FALSE & 0.20 & 2.00 & 1.00 & 0.07 & 0.05 \\ 
   & $\textit{x} + \textit{z}$ & $\textit{x} \sim Binary, \textit{z} \sim Normal$ & 200 & FALSE & 0.20 & 3.00 & 1.00 & 0.08 & 0.05 \\ 
   & $\textit{x} + \textit{z}$ & $\textit{x} \sim Normal , \textit{z} \sim Normal$ & 100 & FALSE & 0.20 & 2.00 & 1.00 & 0.07 & 0.05 \\ 
   & $\textit{x} + \textit{z}$ & $\textit{x} \sim Binary, \textit{z} \sim Binary$ & 100 & FALSE & 0.20 & 2.00 & 1.00 & 0.07 & 0.05 \\ 
   & $\textit{x} + \textit{z}$ & $\textit{x} \sim Normal, \textit{z} \sim Binary$ & 100 & FALSE & 0.20 & 2.00 & 1.00 & 0.07 & 0.05 \\ 
   & $\textit{x} + \textit{z}$ & $\textit{x} \sim Binary, \textit{z} \sim Normal$ & 100 & FALSE & 0.20 & 2.00 & 1.00 & 0.06 & 0.05 \\ 
   & $\textit{x} + \textit{z}$ & $\textit{x} \sim Normal , \textit{z} \sim Normal$ & 150 & FALSE & 0.20 & 2.00 & 1.00 & 0.07 & 0.05 \\ 
   & $\textit{x} + \textit{z}$ & $\textit{x} \sim Binary, \textit{z} \sim Binary$ & 150 & FALSE & 0.20 & 2.00 & 1.00 & 0.07 & 0.05 \\ 
   & $\textit{x} + \textit{z}$ & $\textit{x} \sim Normal, \textit{z} \sim Binary$ & 150 & FALSE & 0.20 & 2.00 & 1.00 & 0.07 & 0.05 \\ 
   & $\textit{x} + \textit{z}$ & $\textit{x} \sim Binary, \textit{z} \sim Normal$ & 150 & FALSE & 0.20 & 2.00 & 1.00 & 0.07 & 0.05 \\ 
   & $\textit{x} + \textit{z}$ & $\textit{x} \sim Normal , \textit{z} \sim Normal$ & 250 & FALSE & 0.20 & 2.00 & 1.00 & 0.07 & 0.05 \\ 
   & $\textit{x} + \textit{z}$ & $\textit{x} \sim Binary, \textit{z} \sim Binary$ & 250 & FALSE & 0.20 & 2.00 & 1.00 & 0.07 & 0.05 \\ 
   & $\textit{x} + \textit{z}$ & $\textit{x} \sim Normal, \textit{z} \sim Binary$ & 250 & FALSE & 0.20 & 2.00 & 1.00 & 0.07 & 0.05 \\ 
   & $\textit{x} + \textit{z}$ & $\textit{x} \sim Binary, \textit{z} \sim Normal$ & 250 & FALSE & 0.20 & 2.00 & 1.00 & 0.07 & 0.05 \\ 
   & $\textit{x} + \textit{z}$ & $\textit{x} \sim Normal , \textit{z} \sim Normal$ & 300 & FALSE & 0.20 & 2.00 & 1.00 & 0.07 & 0.05 \\ 
   & $\textit{x} + \textit{z}$ & $\textit{x} \sim Binary, \textit{z} \sim Binary$ & 300 & FALSE & 0.20 & 2.00 & 1.00 & 0.06 & 0.05 \\ 
   & $\textit{x} + \textit{z}$ & $\textit{x} \sim Normal, \textit{z} \sim Binary$ & 300 & FALSE & 0.20 & 2.00 & 1.00 & 0.07 & 0.05 \\ 
   & $\textit{x} + \textit{z}$ & $\textit{x} \sim Binary, \textit{z} \sim Normal$ & 300 & FALSE & 0.20 & 2.00 & 1.00 & 0.07 & 0.05 \\ 
   & $\textit{x} + \textit{z}$ & $\textit{x} \sim Normal , \textit{z} \sim Normal$ & 100 & FALSE & 0.20 & 2.00 & 1.00 & 0.07 & 0.05 \\ 
   & $\textit{x} + \textit{z}$ & $\textit{x} \sim Binary, \textit{z} \sim Binary$ & 100 & FALSE & 0.20 & 2.00 & 1.00 & 0.06 & 0.05 \\ 
   & $\textit{x} + \textit{z}$ & $\textit{x} \sim Normal, \textit{z} \sim Binary$ & 100 & FALSE & 0.20 & 2.00 & 1.00 & 0.06 & 0.05 \\ 
   & $\textit{x} + \textit{z}$ & $\textit{x} \sim Binary, \textit{z} \sim Normal$ & 100 & FALSE & 0.20 & 2.00 & 1.00 & 0.07 & 0.05 \\ 
   & $\textit{x} + \textit{z}$ & $\textit{x} \sim Normal , \textit{z} \sim Normal$ & 150 & FALSE & 0.20 & 2.00 & 1.00 & 0.07 & 0.05 \\ 
   & $\textit{x} + \textit{z}$ & $\textit{x} \sim Binary, \textit{z} \sim Binary$ & 150 & FALSE & 0.20 & 2.00 & 1.00 & 0.07 & 0.05 \\ 
   & $\textit{x} + \textit{z}$ & $\textit{x} \sim Normal, \textit{z} \sim Binary$ & 150 & FALSE & 0.20 & 2.00 & 1.00 & 0.06 & 0.05 \\ 
   & $\textit{x} + \textit{z}$ & $\textit{x} \sim Binary, \textit{z} \sim Normal$ & 150 & FALSE & 0.20 & 2.00 & 1.00 & 0.07 & 0.05 \\ 
   & $\textit{x} + \textit{z}$ & $\textit{x} \sim Normal , \textit{z} \sim Normal$ & 250 & FALSE & 0.20 & 2.00 & 1.00 & 0.07 & 0.05 \\ 
   & $\textit{x} + \textit{z}$ & $\textit{x} \sim Binary, \textit{z} \sim Binary$ & 250 & FALSE & 0.20 & 2.00 & 1.00 & 0.06 & 0.05 \\ 
   & $\textit{x} + \textit{z}$ & $\textit{x} \sim Normal, \textit{z} \sim Binary$ & 250 & FALSE & 0.20 & 2.00 & 1.00 & 0.07 & 0.05 \\ 
   & $\textit{x} + \textit{z}$ & $\textit{x} \sim Binary, \textit{z} \sim Normal$ & 250 & FALSE & 0.20 & 2.00 & 1.00 & 0.07 & 0.05 \\ 
   & $\textit{x} + \textit{z}$ & $\textit{x} \sim Normal , \textit{z} \sim Normal$ & 300 & FALSE & 0.20 & 2.00 & 1.00 & 0.07 & 0.05 \\ 
   & $\textit{x} + \textit{z}$ & $\textit{x} \sim Binary, \textit{z} \sim Binary$ & 300 & FALSE & 0.20 & 2.00 & 1.00 & 0.07 & 0.05 \\ 
   & $\textit{x} + \textit{z}$ & $\textit{x} \sim Normal, \textit{z} \sim Binary$ & 300 & FALSE & 0.20 & 2.00 & 1.00 & 0.06 & 0.05 \\ 
   & $\textit{x} + \textit{z}$ & $\textit{x} \sim Binary, \textit{z} \sim Normal$ & 300 & FALSE & 0.20 & 2.00 & 1.00 & 0.07 & 0.05 \\ 
   & $\textit{x} + \textit{z}$ & $\textit{x} \sim Normal , \textit{z} \sim Normal$ & 200 & FALSE & 0.30 & 2.00 & 1.00 & 0.07 & 0.05 \\ 
   & $\textit{x} + \textit{z}$ & $\textit{x} \sim Binary, \textit{z} \sim Binary$ & 200 & FALSE & 0.30 & 2.00 & 1.00 & 0.08 & 0.05 \\ 
   & $\textit{x} + \textit{z}$ & $\textit{x} \sim Normal, \textit{z} \sim Binary$ & 200 & FALSE & 0.30 & 2.00 & 1.00 & 0.07 & 0.05 \\ 
   & $\textit{x} + \textit{z}$ & $\textit{x} \sim Binary, \textit{z} \sim Normal$ & 200 & FALSE & 0.30 & 2.00 & 1.00 & 0.08 & 0.05 \\ 
   & $\textit{x} + \textit{z}$ & $\textit{x} \sim Normal , \textit{z} \sim Normal$ & 200 & FALSE & 0.40 & 2.00 & 1.00 & 0.09 & 0.05 \\ 
   & $\textit{x} + \textit{z}$ & $\textit{x} \sim Binary, \textit{z} \sim Binary$ & 200 & FALSE & 0.40 & 2.00 & 1.00 & 0.09 & 0.05 \\ 
   & $\textit{x} + \textit{z}$ & $\textit{x} \sim Normal, \textit{z} \sim Binary$ & 200 & FALSE & 0.40 & 2.00 & 1.00 & 0.09 & 0.05 \\ 
   & $\textit{x} + \textit{z}$ & $\textit{x} \sim Binary, \textit{z} \sim Normal$ & 200 & FALSE & 0.40 & 2.00 & 1.00 & 0.09 & 0.05 \\ 
   & $\textit{x} + \textit{z}$ & $\textit{x} \sim Normal , \textit{z} \sim Normal$ & 200 & FALSE & 0.30 & 2.00 & 1.00 & 0.08 & 0.05 \\ 
   & $\textit{x} + \textit{z}$ & $\textit{x} \sim Binary, \textit{z} \sim Binary$ & 200 & FALSE & 0.30 & 2.00 & 1.00 & 0.08 & 0.05 \\ 
   & $\textit{x} + \textit{z}$ & $\textit{x} \sim Normal, \textit{z} \sim Binary$ & 200 & FALSE & 0.30 & 2.00 & 1.00 & 0.08 & 0.05 \\ 
   & $\textit{x} + \textit{z}$ & $\textit{x} \sim Binary, \textit{z} \sim Normal$ & 200 & FALSE & 0.30 & 2.00 & 1.00 & 0.08 & 0.05 \\ 
   & $\textit{x} + \textit{z}$ & $\textit{x} \sim Normal , \textit{z} \sim Normal$ & 200 & FALSE & 0.40 & 2.00 & 1.00 & 0.09 & 0.05 \\ 
   & $\textit{x} + \textit{z}$ & $\textit{x} \sim Binary, \textit{z} \sim Binary$ & 200 & FALSE & 0.40 & 2.00 & 1.00 & 0.09 & 0.05 \\ 
   & $\textit{x} + \textit{z}$ & $\textit{x} \sim Normal, \textit{z} \sim Binary$ & 200 & FALSE & 0.40 & 2.00 & 1.00 & 0.09 & 0.05 \\ 
   & $\textit{x} + \textit{z}$ & $\textit{x} \sim Binary, \textit{z} \sim Normal$ & 200 & FALSE & 0.40 & 2.00 & 1.00 & 0.09 & 0.05 \\ 
   & $\textit{x} + \textit{z}$ & $\textit{x} \sim Normal , \textit{z} \sim Normal$ & 200 & TRUE & 0.20 & 2.00 & 1.00 & 0.13 & 0.05 \\ 
   & $\textit{x} + \textit{z}$ & $\textit{x} \sim Binary, \textit{z} \sim Binary$ & 200 & TRUE & 0.20 & 2.00 & 1.00 & 0.11 & 0.05 \\ 
   & $\textit{x} + \textit{z}$ & $\textit{x} \sim Normal, \textit{z} \sim Binary$ & 200 & TRUE & 0.20 & 2.00 & 1.00 & 0.11 & 0.05 \\ 
   & $\textit{x} + \textit{z}$ & $\textit{x} \sim Binary, \textit{z} \sim Normal$ & 200 & TRUE & 0.20 & 2.00 & 1.00 & 0.13 & 0.05 \\ 
   & $\textit{x} + \textit{z}$ & $\textit{x} \sim Normal , \textit{z} \sim Normal$ & 200 & TRUE & 0.20 & 2.00 & 1.00 & 0.13 & 0.05 \\ 
   & $\textit{x} + \textit{z}$ & $\textit{x} \sim Binary, \textit{z} \sim Binary$ & 200 & TRUE & 0.20 & 2.00 & 1.00 & 0.11 & 0.05 \\ 
   & $\textit{x} + \textit{z}$ & $\textit{x} \sim Normal, \textit{z} \sim Binary$ & 200 & TRUE & 0.20 & 2.00 & 1.00 & 0.12 & 0.05 \\ 
   & $\textit{x} + \textit{z}$ & $\textit{x} \sim Binary, \textit{z} \sim Normal$ & 200 & TRUE & 0.20 & 2.00 & 1.00 & 0.13 & 0.05 \\ 
   & $\textit{x} + \textit{z}$ & $\textit{x} \sim Normal , \textit{z} \sim Normal$ & 200 & FALSE & 0.20 & 2.00 & 3.00 & 0.13 & 0.05 \\ 
   & $\textit{x} + \textit{z}$ & $\textit{x} \sim Binary, \textit{z} \sim Binary$ & 200 & FALSE & 0.20 & 2.00 & 3.00 & 0.13 & 0.05 \\ 
   & $\textit{x} + \textit{z}$ & $\textit{x} \sim Normal, \textit{z} \sim Binary$ & 200 & FALSE & 0.20 & 2.00 & 3.00 & 0.14 & 0.05 \\ 
   & $\textit{x} + \textit{z}$ & $\textit{x} \sim Binary, \textit{z} \sim Normal$ & 200 & FALSE & 0.20 & 2.00 & 3.00 & 0.13 & 0.05 \\ 
   & $\textit{x} + \textit{z}$ & $\textit{x} \sim Normal , \textit{z} \sim Normal$ & 200 & FALSE & 0.20 & 2.00 & 3.00 & 0.13 & 0.05 \\ 
   & $\textit{x} + \textit{z}$ & $\textit{x} \sim Binary, \textit{z} \sim Binary$ & 200 & FALSE & 0.20 & 2.00 & 3.00 & 0.13 & 0.05 \\ 
   & $\textit{x} + \textit{z}$ & $\textit{x} \sim Normal, \textit{z} \sim Binary$ & 200 & FALSE & 0.20 & 2.00 & 3.00 & 0.13 & 0.05 \\ 
   & $\textit{x} + \textit{z}$ & $\textit{x} \sim Binary, \textit{z} \sim Normal$ & 200 & FALSE & 0.20 & 2.00 & 3.00 & 0.13 & 0.05 \\ 
   & $\textit{x} \times \textit{z}$ & $\textit{x} \sim Normal , \textit{z} \sim Normal$ & 200 & FALSE & 0.20 & 2.00 & 1.00 & 0.19 & 0.09 \\ 
   & $\textit{x} \times \textit{z}$ & $\textit{x} \sim Normal , \textit{z} \sim Normal$ & 200 & FALSE & 0.20 & 3.00 & 1.00 & 0.26 & 0.10 \\ 
   & $\textit{x} \times \textit{z}$ & $\textit{x} \sim Binary, \textit{z} \sim Binary$ & 200 & FALSE & 0.20 & 2.00 & 1.00 & 0.83 & 0.34 \\ 
   & $\textit{x} \times \textit{z}$ & $\textit{x} \sim Binary, \textit{z} \sim Binary$ & 200 & FALSE & 0.20 & 3.00 & 1.00 & 0.94 & 0.38 \\ 
   & $\textit{x} \times \textit{z}$ & $\textit{x} \sim Normal, \textit{z} \sim Binary$ & 200 & FALSE & 0.20 & 2.00 & 1.00 & 0.80 & 0.34 \\ 
   & $\textit{x} \times \textit{z}$ & $\textit{x} \sim Normal, \textit{z} \sim Binary$ & 200 & FALSE & 0.20 & 3.00 & 1.00 & 0.92 & 0.37 \\ 
   & $\textit{x} \times \textit{z}$ & $\textit{x} \sim Binary, \textit{z} \sim Normal$ & 200 & FALSE & 0.20 & 2.00 & 1.00 & 0.20 & 0.07 \\ 
   & $\textit{x} \times \textit{z}$ & $\textit{x} \sim Binary, \textit{z} \sim Normal$ & 200 & FALSE & 0.20 & 3.00 & 1.00 & 0.29 & 0.08 \\ 
   & $\textit{x} \times \textit{z}$ & $\textit{x} \sim Normal , \textit{z} \sim Normal$ & 100 & FALSE & 0.20 & 2.00 & 1.00 & 0.19 & 0.09 \\ 
   & $\textit{x} \times \textit{z}$ & $\textit{x} \sim Binary, \textit{z} \sim Binary$ & 100 & FALSE & 0.20 & 2.00 & 1.00 & 0.59 & 0.23 \\ 
   & $\textit{x} \times \textit{z}$ & $\textit{x} \sim Normal, \textit{z} \sim Binary$ & 100 & FALSE & 0.20 & 2.00 & 1.00 & 0.55 & 0.22 \\ 
   & $\textit{x} \times \textit{z}$ & $\textit{x} \sim Binary, \textit{z} \sim Normal$ & 100 & FALSE & 0.20 & 2.00 & 1.00 & 0.20 & 0.07 \\ 
   & $\textit{x} \times \textit{z}$ & $\textit{x} \sim Normal , \textit{z} \sim Normal$ & 150 & FALSE & 0.20 & 2.00 & 1.00 & 0.19 & 0.09 \\ 
   & $\textit{x} \times \textit{z}$ & $\textit{x} \sim Binary, \textit{z} \sim Binary$ & 150 & FALSE & 0.20 & 2.00 & 1.00 & 0.73 & 0.29 \\ 
   & $\textit{x} \times \textit{z}$ & $\textit{x} \sim Normal, \textit{z} \sim Binary$ & 150 & FALSE & 0.20 & 2.00 & 1.00 & 0.69 & 0.28 \\ 
   & $\textit{x} \times \textit{z}$ & $\textit{x} \sim Binary, \textit{z} \sim Normal$ & 150 & FALSE & 0.20 & 2.00 & 1.00 & 0.20 & 0.08 \\ 
   & $\textit{x} \times \textit{z}$ & $\textit{x} \sim Normal , \textit{z} \sim Normal$ & 250 & FALSE & 0.20 & 2.00 & 1.00 & 0.18 & 0.09 \\ 
   & $\textit{x} \times \textit{z}$ & $\textit{x} \sim Binary, \textit{z} \sim Binary$ & 250 & FALSE & 0.20 & 2.00 & 1.00 & 0.89 & 0.38 \\ 
   & $\textit{x} \times \textit{z}$ & $\textit{x} \sim Normal, \textit{z} \sim Binary$ & 250 & FALSE & 0.20 & 2.00 & 1.00 & 0.87 & 0.38 \\ 
   & $\textit{x} \times \textit{z}$ & $\textit{x} \sim Binary, \textit{z} \sim Normal$ & 250 & FALSE & 0.20 & 2.00 & 1.00 & 0.20 & 0.07 \\ 
   & $\textit{x} \times \textit{z}$ & $\textit{x} \sim Normal , \textit{z} \sim Normal$ & 300 & FALSE & 0.20 & 2.00 & 1.00 & 0.18 & 0.08 \\ 
   & $\textit{x} \times \textit{z}$ & $\textit{x} \sim Binary, \textit{z} \sim Binary$ & 300 & FALSE & 0.20 & 2.00 & 1.00 & 0.94 & 0.42 \\ 
   & $\textit{x} \times \textit{z}$ & $\textit{x} \sim Normal, \textit{z} \sim Binary$ & 300 & FALSE & 0.20 & 2.00 & 1.00 & 0.92 & 0.41 \\ 
   & $\textit{x} \times \textit{z}$ & $\textit{x} \sim Binary, \textit{z} \sim Normal$ & 300 & FALSE & 0.20 & 2.00 & 1.00 & 0.20 & 0.07 \\ 
   & $\textit{x} \times \textit{z}$ & $\textit{x} \sim Normal , \textit{z} \sim Normal$ & 200 & FALSE & 0.30 & 2.00 & 1.00 & 0.22 & 0.09 \\ 
   & $\textit{x} \times \textit{z}$ & $\textit{x} \sim Binary, \textit{z} \sim Binary$ & 200 & FALSE & 0.30 & 2.00 & 1.00 & 0.99 & 0.48 \\ 
   & $\textit{x} \times \textit{z}$ & $\textit{x} \sim Normal, \textit{z} \sim Binary$ & 200 & FALSE & 0.30 & 2.00 & 1.00 & 0.99 & 0.48 \\ 
   & $\textit{x} \times \textit{z}$ & $\textit{x} \sim Binary, \textit{z} \sim Normal$ & 200 & FALSE & 0.30 & 2.00 & 1.00 & 0.21 & 0.07 \\ 
   & $\textit{x} \times \textit{z}$ & $\textit{x} \sim Normal , \textit{z} \sim Normal$ & 200 & FALSE & 0.40 & 2.00 & 1.00 & 0.26 & 0.10 \\ 
   & $\textit{x} \times \textit{z}$ & $\textit{x} \sim Binary, \textit{z} \sim Binary$ & 200 & FALSE & 0.40 & 2.00 & 1.00 & 1.00 & 0.52 \\ 
   & $\textit{x} \times \textit{z}$ & $\textit{x} \sim Normal, \textit{z} \sim Binary$ & 200 & FALSE & 0.40 & 2.00 & 1.00 & 1.00 & 0.52 \\ 
   & $\textit{x} \times \textit{z}$ & $\textit{x} \sim Binary, \textit{z} \sim Normal$ & 200 & FALSE & 0.40 & 2.00 & 1.00 & 0.22 & 0.07 \\ 
   & $\textit{x} \times \textit{z}$ & $\textit{x} \sim Normal , \textit{z} \sim Normal$ & 200 & TRUE & 0.20 & 2.00 & 1.00 & 0.36 & 0.08 \\ 
   & $\textit{x} \times \textit{z}$ & $\textit{x} \sim Binary, \textit{z} \sim Binary$ & 200 & TRUE & 0.20 & 2.00 & 1.00 & 0.88 & 0.33 \\ 
   & $\textit{x} \times \textit{z}$ & $\textit{x} \sim Normal, \textit{z} \sim Binary$ & 200 & TRUE & 0.20 & 2.00 & 1.00 & 0.86 & 0.29 \\ 
   & $\textit{x} \times \textit{z}$ & $\textit{x} \sim Binary, \textit{z} \sim Normal$ & 200 & TRUE & 0.20 & 2.00 & 1.00 & 0.35 & 0.07 \\ 
   & $\textit{x} \times \textit{z}$ & $\textit{x} \sim Normal , \textit{z} \sim Normal$ & 200 & FALSE & 0.20 & 2.00 & 3.00 & 0.33 & 0.09 \\ 
   & $\textit{x} \times \textit{z}$ & $\textit{x} \sim Binary, \textit{z} \sim Binary$ & 200 & FALSE & 0.20 & 2.00 & 3.00 & 0.95 & 0.36 \\ 
   & $\textit{x} \times \textit{z}$ & $\textit{x} \sim Normal, \textit{z} \sim Binary$ & 200 & FALSE & 0.20 & 2.00 & 3.00 & 0.93 & 0.35 \\ 
   & $\textit{x} \times \textit{z}$ & $\textit{x} \sim Binary, \textit{z} \sim Normal$ & 200 & FALSE & 0.20 & 2.00 & 3.00 & 0.35 & 0.07 \\ 
   & $\textit{z} \times \textit{z}$ & $\textit{x} \sim Normal , \textit{z} \sim Normal$ & 200 & FALSE & 0.20 & 2.00 & 1.00 & 0.07 & 0.05 \\ 
   & $\textit{z} \times \textit{z}$ & $\textit{x} \sim Normal , \textit{z} \sim Normal$ & 200 & FALSE & 0.20 & 3.00 & 1.00 & 0.09 & 0.05 \\ 
   & $\textit{z} \times \textit{z}$ & $\textit{x} \sim Binary, \textit{z} \sim Binary$ & 200 & FALSE & 0.20 & 2.00 & 1.00 & 0.07 & 0.05 \\ 
   & $\textit{z} \times \textit{z}$ & $\textit{x} \sim Binary, \textit{z} \sim Binary$ & 200 & FALSE & 0.20 & 3.00 & 1.00 & 0.09 & 0.05 \\ 
   & $\textit{z} \times \textit{z}$ & $\textit{x} \sim Normal, \textit{z} \sim Binary$ & 200 & FALSE & 0.20 & 2.00 & 1.00 & 0.07 & 0.05 \\ 
   & $\textit{z} \times \textit{z}$ & $\textit{x} \sim Normal, \textit{z} \sim Binary$ & 200 & FALSE & 0.20 & 3.00 & 1.00 & 0.09 & 0.05 \\ 
   & $\textit{z} \times \textit{z}$ & $\textit{x} \sim Binary, \textit{z} \sim Normal$ & 200 & FALSE & 0.20 & 2.00 & 1.00 & 0.07 & 0.05 \\ 
   & $\textit{z} \times \textit{z}$ & $\textit{x} \sim Binary, \textit{z} \sim Normal$ & 200 & FALSE & 0.20 & 3.00 & 1.00 & 0.09 & 0.05 \\ 
   & $\textit{z} \times \textit{z}$ & $\textit{x} \sim Normal , \textit{z} \sim Normal$ & 100 & FALSE & 0.20 & 2.00 & 1.00 & 0.07 & 0.05 \\ 
   & $\textit{z} \times \textit{z}$ & $\textit{x} \sim Binary, \textit{z} \sim Binary$ & 100 & FALSE & 0.20 & 2.00 & 1.00 & 0.07 & 0.05 \\ 
   & $\textit{z} \times \textit{z}$ & $\textit{x} \sim Normal, \textit{z} \sim Binary$ & 100 & FALSE & 0.20 & 2.00 & 1.00 & 0.07 & 0.05 \\ 
   & $\textit{z} \times \textit{z}$ & $\textit{x} \sim Binary, \textit{z} \sim Normal$ & 100 & FALSE & 0.20 & 2.00 & 1.00 & 0.07 & 0.05 \\ 
   & $\textit{z} \times \textit{z}$ & $\textit{x} \sim Normal , \textit{z} \sim Normal$ & 150 & FALSE & 0.20 & 2.00 & 1.00 & 0.07 & 0.05 \\ 
   & $\textit{z} \times \textit{z}$ & $\textit{x} \sim Binary, \textit{z} \sim Binary$ & 150 & FALSE & 0.20 & 2.00 & 1.00 & 0.07 & 0.05 \\ 
   & $\textit{z} \times \textit{z}$ & $\textit{x} \sim Normal, \textit{z} \sim Binary$ & 150 & FALSE & 0.20 & 2.00 & 1.00 & 0.07 & 0.05 \\ 
   & $\textit{z} \times \textit{z}$ & $\textit{x} \sim Binary, \textit{z} \sim Normal$ & 150 & FALSE & 0.20 & 2.00 & 1.00 & 0.07 & 0.05 \\ 
   & $\textit{z} \times \textit{z}$ & $\textit{x} \sim Normal , \textit{z} \sim Normal$ & 250 & FALSE & 0.20 & 2.00 & 1.00 & 0.07 & 0.05 \\ 
   & $\textit{z} \times \textit{z}$ & $\textit{x} \sim Binary, \textit{z} \sim Binary$ & 250 & FALSE & 0.20 & 2.00 & 1.00 & 0.07 & 0.05 \\ 
   & $\textit{z} \times \textit{z}$ & $\textit{x} \sim Normal, \textit{z} \sim Binary$ & 250 & FALSE & 0.20 & 2.00 & 1.00 & 0.07 & 0.05 \\ 
   & $\textit{z} \times \textit{z}$ & $\textit{x} \sim Binary, \textit{z} \sim Normal$ & 250 & FALSE & 0.20 & 2.00 & 1.00 & 0.07 & 0.05 \\ 
   & $\textit{z} \times \textit{z}$ & $\textit{x} \sim Normal , \textit{z} \sim Normal$ & 300 & FALSE & 0.20 & 2.00 & 1.00 & 0.07 & 0.05 \\ 
   & $\textit{z} \times \textit{z}$ & $\textit{x} \sim Binary, \textit{z} \sim Binary$ & 300 & FALSE & 0.20 & 2.00 & 1.00 & 0.07 & 0.05 \\ 
   & $\textit{z} \times \textit{z}$ & $\textit{x} \sim Normal, \textit{z} \sim Binary$ & 300 & FALSE & 0.20 & 2.00 & 1.00 & 0.07 & 0.05 \\ 
   & $\textit{z} \times \textit{z}$ & $\textit{x} \sim Binary, \textit{z} \sim Normal$ & 300 & FALSE & 0.20 & 2.00 & 1.00 & 0.07 & 0.05 \\ 
   & $\textit{z} \times \textit{z}$ & $\textit{x} \sim Normal , \textit{z} \sim Normal$ & 200 & FALSE & 0.30 & 2.00 & 1.00 & 0.08 & 0.05 \\ 
   & $\textit{z} \times \textit{z}$ & $\textit{x} \sim Binary, \textit{z} \sim Binary$ & 200 & FALSE & 0.30 & 2.00 & 1.00 & 0.08 & 0.05 \\ 
   & $\textit{z} \times \textit{z}$ & $\textit{x} \sim Normal, \textit{z} \sim Binary$ & 200 & FALSE & 0.30 & 2.00 & 1.00 & 0.08 & 0.05 \\ 
   & $\textit{z} \times \textit{z}$ & $\textit{x} \sim Binary, \textit{z} \sim Normal$ & 200 & FALSE & 0.30 & 2.00 & 1.00 & 0.08 & 0.05 \\ 
   & $\textit{z} \times \textit{z}$ & $\textit{x} \sim Normal , \textit{z} \sim Normal$ & 200 & FALSE & 0.40 & 2.00 & 1.00 & 0.09 & 0.05 \\ 
   & $\textit{z} \times \textit{z}$ & $\textit{x} \sim Binary, \textit{z} \sim Binary$ & 200 & FALSE & 0.40 & 2.00 & 1.00 & 0.10 & 0.05 \\ 
   & $\textit{z} \times \textit{z}$ & $\textit{x} \sim Normal, \textit{z} \sim Binary$ & 200 & FALSE & 0.40 & 2.00 & 1.00 & 0.10 & 0.05 \\ 
   & $\textit{z} \times \textit{z}$ & $\textit{x} \sim Binary, \textit{z} \sim Normal$ & 200 & FALSE & 0.40 & 2.00 & 1.00 & 0.09 & 0.05 \\ 
   & $\textit{z} \times \textit{z}$ & $\textit{x} \sim Normal , \textit{z} \sim Normal$ & 200 & TRUE & 0.20 & 2.00 & 1.00 & 0.14 & 0.05 \\ 
   & $\textit{z} \times \textit{z}$ & $\textit{x} \sim Binary, \textit{z} \sim Binary$ & 200 & TRUE & 0.20 & 2.00 & 1.00 & 0.11 & 0.05 \\ 
   & $\textit{z} \times \textit{z}$ & $\textit{x} \sim Normal, \textit{z} \sim Binary$ & 200 & TRUE & 0.20 & 2.00 & 1.00 & 0.12 & 0.05 \\ 
   & $\textit{z} \times \textit{z}$ & $\textit{x} \sim Binary, \textit{z} \sim Normal$ & 200 & TRUE & 0.20 & 2.00 & 1.00 & 0.13 & 0.05 \\ 
   & $\textit{z} \times \textit{z}$ & $\textit{x} \sim Normal , \textit{z} \sim Normal$ & 200 & FALSE & 0.20 & 2.00 & 3.00 & 0.14 & 0.05 \\ 
   & $\textit{z} \times \textit{z}$ & $\textit{x} \sim Binary, \textit{z} \sim Binary$ & 200 & FALSE & 0.20 & 2.00 & 3.00 & 0.14 & 0.05 \\ 
   & $\textit{z} \times \textit{z}$ & $\textit{x} \sim Normal, \textit{z} \sim Binary$ & 200 & FALSE & 0.20 & 2.00 & 3.00 & 0.14 & 0.05 \\ 
   & $\textit{z} \times \textit{z}$ & $\textit{x} \sim Binary, \textit{z} \sim Normal$ & 200 & FALSE & 0.20 & 2.00 & 3.00 & 0.13 & 0.05 \\ 
   & $\textit{x} + \textit{z} + \textit{x} \times \textit{z}$ & $\textit{x} \sim Normal , \textit{z} \sim Normal$ & 200 & FALSE & 0.20 & 2.00 & 1.00 & 0.18 & 0.08 \\ 
   & $\textit{x} + \textit{z} + \textit{x} \times \textit{z}$ & $\textit{x} \sim Normal , \textit{z} \sim Normal$ & 200 & FALSE & 0.20 & 3.00 & 1.00 & 0.27 & 0.09 \\ 
   & $\textit{x} + \textit{z} + \textit{x} \times \textit{z}$ & $\textit{x} \sim Binary, \textit{z} \sim Binary$ & 200 & FALSE & 0.20 & 2.00 & 1.00 & 0.21 & 0.08 \\ 
   & $\textit{x} + \textit{z} + \textit{x} \times \textit{z}$ & $\textit{x} \sim Binary, \textit{z} \sim Binary$ & 200 & FALSE & 0.20 & 3.00 & 1.00 & 0.31 & 0.08 \\ 
   & $\textit{x} + \textit{z} + \textit{x} \times \textit{z}$ & $\textit{x} \sim Normal, \textit{z} \sim Binary$ & 200 & FALSE & 0.20 & 2.00 & 1.00 & 0.18 & 0.08 \\ 
   & $\textit{x} + \textit{z} + \textit{x} \times \textit{z}$ & $\textit{x} \sim Normal, \textit{z} \sim Binary$ & 200 & FALSE & 0.20 & 3.00 & 1.00 & 0.27 & 0.09 \\ 
   & $\textit{x} + \textit{z} + \textit{x} \times \textit{z}$ & $\textit{x} \sim Binary, \textit{z} \sim Normal$ & 200 & FALSE & 0.20 & 2.00 & 1.00 & 0.21 & 0.07 \\ 
   & $\textit{x} + \textit{z} + \textit{x} \times \textit{z}$ & $\textit{x} \sim Binary, \textit{z} \sim Normal$ & 200 & FALSE & 0.20 & 3.00 & 1.00 & 0.32 & 0.08 \\ 
   & $\textit{x} + \textit{z} + \textit{x} \times \textit{z}$ & $\textit{x} \sim Normal , \textit{z} \sim Normal$ & 200 & FALSE & 0.20 & 2.00 & 1.00 & 0.24 & 0.10 \\ 
   & $\textit{x} + \textit{z} + \textit{x} \times \textit{z}$ & $\textit{x} \sim Normal , \textit{z} \sim Normal$ & 200 & FALSE & 0.20 & 3.00 & 1.00 & 0.37 & 0.13 \\ 
   & $\textit{x} + \textit{z} + \textit{x} \times \textit{z}$ & $\textit{x} \sim Binary, \textit{z} \sim Binary$ & 200 & FALSE & 0.20 & 2.00 & 1.00 & 0.87 & 0.32 \\ 
   & $\textit{x} + \textit{z} + \textit{x} \times \textit{z}$ & $\textit{x} \sim Binary, \textit{z} \sim Binary$ & 200 & FALSE & 0.20 & 3.00 & 1.00 & 0.97 & 0.41 \\ 
   & $\textit{x} + \textit{z} + \textit{x} \times \textit{z}$ & $\textit{x} \sim Normal, \textit{z} \sim Binary$ & 200 & FALSE & 0.20 & 2.00 & 1.00 & 0.83 & 0.32 \\ 
   & $\textit{x} + \textit{z} + \textit{x} \times \textit{z}$ & $\textit{x} \sim Normal, \textit{z} \sim Binary$ & 200 & FALSE & 0.20 & 3.00 & 1.00 & 0.95 & 0.41 \\ 
   & $\textit{x} + \textit{z} + \textit{x} \times \textit{z}$ & $\textit{x} \sim Binary, \textit{z} \sim Normal$ & 200 & FALSE & 0.20 & 2.00 & 1.00 & 0.25 & 0.09 \\ 
   & $\textit{x} + \textit{z} + \textit{x} \times \textit{z}$ & $\textit{x} \sim Binary, \textit{z} \sim Normal$ & 200 & FALSE & 0.20 & 3.00 & 1.00 & 0.39 & 0.11 \\ 
   & $\textit{x} + \textit{z} + \textit{x} \times \textit{z}$ & $\textit{x} \sim Normal , \textit{z} \sim Normal$ & 100 & FALSE & 0.20 & 2.00 & 1.00 & 0.18 & 0.08 \\ 
   & $\textit{x} + \textit{z} + \textit{x} \times \textit{z}$ & $\textit{x} \sim Binary, \textit{z} \sim Binary$ & 100 & FALSE & 0.20 & 2.00 & 1.00 & 0.21 & 0.07 \\ 
   & $\textit{x} + \textit{z} + \textit{x} \times \textit{z}$ & $\textit{x} \sim Normal, \textit{z} \sim Binary$ & 100 & FALSE & 0.20 & 2.00 & 1.00 & 0.18 & 0.08 \\ 
   & $\textit{x} + \textit{z} + \textit{x} \times \textit{z}$ & $\textit{x} \sim Binary, \textit{z} \sim Normal$ & 100 & FALSE & 0.20 & 2.00 & 1.00 & 0.21 & 0.07 \\ 
   & $\textit{x} + \textit{z} + \textit{x} \times \textit{z}$ & $\textit{x} \sim Normal , \textit{z} \sim Normal$ & 150 & FALSE & 0.20 & 2.00 & 1.00 & 0.18 & 0.08 \\ 
   & $\textit{x} + \textit{z} + \textit{x} \times \textit{z}$ & $\textit{x} \sim Binary, \textit{z} \sim Binary$ & 150 & FALSE & 0.20 & 2.00 & 1.00 & 0.21 & 0.07 \\ 
   & $\textit{x} + \textit{z} + \textit{x} \times \textit{z}$ & $\textit{x} \sim Normal, \textit{z} \sim Binary$ & 150 & FALSE & 0.20 & 2.00 & 1.00 & 0.18 & 0.08 \\ 
   & $\textit{x} + \textit{z} + \textit{x} \times \textit{z}$ & $\textit{x} \sim Binary, \textit{z} \sim Normal$ & 150 & FALSE & 0.20 & 2.00 & 1.00 & 0.21 & 0.07 \\ 
   & $\textit{x} + \textit{z} + \textit{x} \times \textit{z}$ & $\textit{x} \sim Normal , \textit{z} \sim Normal$ & 250 & FALSE & 0.20 & 2.00 & 1.00 & 0.18 & 0.08 \\ 
   & $\textit{x} + \textit{z} + \textit{x} \times \textit{z}$ & $\textit{x} \sim Binary, \textit{z} \sim Binary$ & 250 & FALSE & 0.20 & 2.00 & 1.00 & 0.21 & 0.07 \\ 
   & $\textit{x} + \textit{z} + \textit{x} \times \textit{z}$ & $\textit{x} \sim Normal, \textit{z} \sim Binary$ & 250 & FALSE & 0.20 & 2.00 & 1.00 & 0.18 & 0.08 \\ 
   & $\textit{x} + \textit{z} + \textit{x} \times \textit{z}$ & $\textit{x} \sim Binary, \textit{z} \sim Normal$ & 250 & FALSE & 0.20 & 2.00 & 1.00 & 0.21 & 0.07 \\ 
   & $\textit{x} + \textit{z} + \textit{x} \times \textit{z}$ & $\textit{x} \sim Normal , \textit{z} \sim Normal$ & 300 & FALSE & 0.20 & 2.00 & 1.00 & 0.19 & 0.08 \\ 
   & $\textit{x} + \textit{z} + \textit{x} \times \textit{z}$ & $\textit{x} \sim Binary, \textit{z} \sim Binary$ & 300 & FALSE & 0.20 & 2.00 & 1.00 & 0.21 & 0.07 \\ 
   & $\textit{x} + \textit{z} + \textit{x} \times \textit{z}$ & $\textit{x} \sim Normal, \textit{z} \sim Binary$ & 300 & FALSE & 0.20 & 2.00 & 1.00 & 0.18 & 0.08 \\ 
   & $\textit{x} + \textit{z} + \textit{x} \times \textit{z}$ & $\textit{x} \sim Binary, \textit{z} \sim Normal$ & 300 & FALSE & 0.20 & 2.00 & 1.00 & 0.21 & 0.07 \\ 
   & $\textit{x} + \textit{z} + \textit{x} \times \textit{z}$ & $\textit{x} \sim Normal , \textit{z} \sim Normal$ & 100 & FALSE & 0.20 & 2.00 & 1.00 & 0.24 & 0.10 \\ 
   & $\textit{x} + \textit{z} + \textit{x} \times \textit{z}$ & $\textit{x} \sim Binary, \textit{z} \sim Binary$ & 100 & FALSE & 0.20 & 2.00 & 1.00 & 0.66 & 0.21 \\ 
   & $\textit{x} + \textit{z} + \textit{x} \times \textit{z}$ & $\textit{x} \sim Normal, \textit{z} \sim Binary$ & 100 & FALSE & 0.20 & 2.00 & 1.00 & 0.60 & 0.21 \\ 
   & $\textit{x} + \textit{z} + \textit{x} \times \textit{z}$ & $\textit{x} \sim Binary, \textit{z} \sim Normal$ & 100 & FALSE & 0.20 & 2.00 & 1.00 & 0.25 & 0.09 \\ 
   & $\textit{x} + \textit{z} + \textit{x} \times \textit{z}$ & $\textit{x} \sim Normal , \textit{z} \sim Normal$ & 150 & FALSE & 0.20 & 2.00 & 1.00 & 0.23 & 0.10 \\ 
   & $\textit{x} + \textit{z} + \textit{x} \times \textit{z}$ & $\textit{x} \sim Binary, \textit{z} \sim Binary$ & 150 & FALSE & 0.20 & 2.00 & 1.00 & 0.78 & 0.27 \\ 
   & $\textit{x} + \textit{z} + \textit{x} \times \textit{z}$ & $\textit{x} \sim Normal, \textit{z} \sim Binary$ & 150 & FALSE & 0.20 & 2.00 & 1.00 & 0.74 & 0.27 \\ 
   & $\textit{x} + \textit{z} + \textit{x} \times \textit{z}$ & $\textit{x} \sim Binary, \textit{z} \sim Normal$ & 150 & FALSE & 0.20 & 2.00 & 1.00 & 0.25 & 0.09 \\ 
   & $\textit{x} + \textit{z} + \textit{x} \times \textit{z}$ & $\textit{x} \sim Normal , \textit{z} \sim Normal$ & 250 & FALSE & 0.20 & 2.00 & 1.00 & 0.23 & 0.10 \\ 
   & $\textit{x} + \textit{z} + \textit{x} \times \textit{z}$ & $\textit{x} \sim Binary, \textit{z} \sim Binary$ & 250 & FALSE & 0.20 & 2.00 & 1.00 & 0.92 & 0.36 \\ 
   & $\textit{x} + \textit{z} + \textit{x} \times \textit{z}$ & $\textit{x} \sim Normal, \textit{z} \sim Binary$ & 250 & FALSE & 0.20 & 2.00 & 1.00 & 0.90 & 0.36 \\ 
   & $\textit{x} + \textit{z} + \textit{x} \times \textit{z}$ & $\textit{x} \sim Binary, \textit{z} \sim Normal$ & 250 & FALSE & 0.20 & 2.00 & 1.00 & 0.26 & 0.09 \\ 
   & $\textit{x} + \textit{z} + \textit{x} \times \textit{z}$ & $\textit{x} \sim Normal , \textit{z} \sim Normal$ & 300 & FALSE & 0.20 & 2.00 & 1.00 & 0.24 & 0.10 \\ 
   & $\textit{x} + \textit{z} + \textit{x} \times \textit{z}$ & $\textit{x} \sim Binary, \textit{z} \sim Binary$ & 300 & FALSE & 0.20 & 2.00 & 1.00 & 0.95 & 0.39 \\ 
   & $\textit{x} + \textit{z} + \textit{x} \times \textit{z}$ & $\textit{x} \sim Normal, \textit{z} \sim Binary$ & 300 & FALSE & 0.20 & 2.00 & 1.00 & 0.94 & 0.39 \\ 
   & $\textit{x} + \textit{z} + \textit{x} \times \textit{z}$ & $\textit{x} \sim Binary, \textit{z} \sim Normal$ & 300 & FALSE & 0.20 & 2.00 & 1.00 & 0.24 & 0.09 \\ 
   & $\textit{x} + \textit{z} + \textit{x} \times \textit{z}$ & $\textit{x} \sim Normal , \textit{z} \sim Normal$ & 200 & FALSE & 0.30 & 2.00 & 1.00 & 0.20 & 0.08 \\ 
   & $\textit{x} + \textit{z} + \textit{x} \times \textit{z}$ & $\textit{x} \sim Binary, \textit{z} \sim Binary$ & 200 & FALSE & 0.30 & 2.00 & 1.00 & 0.23 & 0.07 \\ 
   & $\textit{x} + \textit{z} + \textit{x} \times \textit{z}$ & $\textit{x} \sim Normal, \textit{z} \sim Binary$ & 200 & FALSE & 0.30 & 2.00 & 1.00 & 0.19 & 0.08 \\ 
   & $\textit{x} + \textit{z} + \textit{x} \times \textit{z}$ & $\textit{x} \sim Binary, \textit{z} \sim Normal$ & 200 & FALSE & 0.30 & 2.00 & 1.00 & 0.22 & 0.07 \\ 
   & $\textit{x} + \textit{z} + \textit{x} \times \textit{z}$ & $\textit{x} \sim Normal , \textit{z} \sim Normal$ & 200 & FALSE & 0.40 & 2.00 & 1.00 & 0.22 & 0.08 \\ 
   & $\textit{x} + \textit{z} + \textit{x} \times \textit{z}$ & $\textit{x} \sim Binary, \textit{z} \sim Binary$ & 200 & FALSE & 0.40 & 2.00 & 1.00 & 0.24 & 0.07 \\ 
   & $\textit{x} + \textit{z} + \textit{x} \times \textit{z}$ & $\textit{x} \sim Normal, \textit{z} \sim Binary$ & 200 & FALSE & 0.40 & 2.00 & 1.00 & 0.21 & 0.08 \\ 
   & $\textit{x} + \textit{z} + \textit{x} \times \textit{z}$ & $\textit{x} \sim Binary, \textit{z} \sim Normal$ & 200 & FALSE & 0.40 & 2.00 & 1.00 & 0.25 & 0.07 \\ 
   & $\textit{x} + \textit{z} + \textit{x} \times \textit{z}$ & $\textit{x} \sim Normal , \textit{z} \sim Normal$ & 200 & FALSE & 0.30 & 2.00 & 1.00 & 0.29 & 0.11 \\ 
   & $\textit{x} + \textit{z} + \textit{x} \times \textit{z}$ & $\textit{x} \sim Binary, \textit{z} \sim Binary$ & 200 & FALSE & 0.30 & 2.00 & 1.00 & 1.00 & 0.46 \\ 
   & $\textit{x} + \textit{z} + \textit{x} \times \textit{z}$ & $\textit{x} \sim Normal, \textit{z} \sim Binary$ & 200 & FALSE & 0.30 & 2.00 & 1.00 & 0.99 & 0.47 \\ 
   & $\textit{x} + \textit{z} + \textit{x} \times \textit{z}$ & $\textit{x} \sim Binary, \textit{z} \sim Normal$ & 200 & FALSE & 0.30 & 2.00 & 1.00 & 0.28 & 0.09 \\ 
   & $\textit{x} + \textit{z} + \textit{x} \times \textit{z}$ & $\textit{x} \sim Normal , \textit{z} \sim Normal$ & 200 & FALSE & 0.40 & 2.00 & 1.00 & 0.36 & 0.12 \\ 
   & $\textit{x} + \textit{z} + \textit{x} \times \textit{z}$ & $\textit{x} \sim Binary, \textit{z} \sim Binary$ & 200 & FALSE & 0.40 & 2.00 & 1.00 & 1.00 & 0.51 \\ 
   & $\textit{x} + \textit{z} + \textit{x} \times \textit{z}$ & $\textit{x} \sim Normal, \textit{z} \sim Binary$ & 200 & FALSE & 0.40 & 2.00 & 1.00 & 1.00 & 0.51 \\ 
   & $\textit{x} + \textit{z} + \textit{x} \times \textit{z}$ & $\textit{x} \sim Binary, \textit{z} \sim Normal$ & 200 & FALSE & 0.40 & 2.00 & 1.00 & 0.31 & 0.09 \\ 
   & $\textit{x} + \textit{z} + \textit{x} \times \textit{z}$ & $\textit{x} \sim Normal , \textit{z} \sim Normal$ & 200 & TRUE & 0.20 & 2.00 & 1.00 & 0.42 & 0.10 \\ 
   & $\textit{x} + \textit{z} + \textit{x} \times \textit{z}$ & $\textit{x} \sim Binary, \textit{z} \sim Binary$ & 200 & TRUE & 0.20 & 2.00 & 1.00 & 0.92 & 0.30 \\ 
   & $\textit{x} + \textit{z} + \textit{x} \times \textit{z}$ & $\textit{x} \sim Normal, \textit{z} \sim Binary$ & 200 & TRUE & 0.20 & 2.00 & 1.00 & 0.91 & 0.28 \\ 
   & $\textit{x} + \textit{z} + \textit{x} \times \textit{z}$ & $\textit{x} \sim Binary, \textit{z} \sim Normal$ & 200 & TRUE & 0.20 & 2.00 & 1.00 & 0.42 & 0.09 \\ 
   & $\textit{x} + \textit{z} + \textit{x} \times \textit{z}$ & $\textit{x} \sim Normal , \textit{z} \sim Normal$ & 200 & TRUE & 0.20 & 2.00 & 1.00 & 0.34 & 0.08 \\ 
   & $\textit{x} + \textit{z} + \textit{x} \times \textit{z}$ & $\textit{x} \sim Binary, \textit{z} \sim Binary$ & 200 & TRUE & 0.20 & 2.00 & 1.00 & 0.32 & 0.07 \\ 
   & $\textit{x} + \textit{z} + \textit{x} \times \textit{z}$ & $\textit{x} \sim Normal, \textit{z} \sim Binary$ & 200 & TRUE & 0.20 & 2.00 & 1.00 & 0.32 & 0.08 \\ 
   & $\textit{x} + \textit{z} + \textit{x} \times \textit{z}$ & $\textit{x} \sim Binary, \textit{z} \sim Normal$ & 200 & TRUE & 0.20 & 2.00 & 1.00 & 0.37 & 0.07 \\ 
   & $\textit{x} + \textit{z} + \textit{x} \times \textit{z}$ & $\textit{x} \sim Normal , \textit{z} \sim Normal$ & 200 & FALSE & 0.20 & 2.00 & 3.00 & 0.34 & 0.08 \\ 
   & $\textit{x} + \textit{z} + \textit{x} \times \textit{z}$ & $\textit{x} \sim Binary, \textit{z} \sim Binary$ & 200 & FALSE & 0.20 & 2.00 & 3.00 & 0.37 & 0.07 \\ 
   & $\textit{x} + \textit{z} + \textit{x} \times \textit{z}$ & $\textit{x} \sim Normal, \textit{z} \sim Binary$ & 200 & FALSE & 0.20 & 2.00 & 3.00 & 0.33 & 0.08 \\ 
   & $\textit{x} + \textit{z} + \textit{x} \times \textit{z}$ & $\textit{x} \sim Binary, \textit{z} \sim Normal$ & 200 & FALSE & 0.20 & 2.00 & 3.00 & 0.37 & 0.07 \\ 
   & $\textit{x} + \textit{z} + \textit{x} \times \textit{z}$ & $\textit{x} \sim Normal , \textit{z} \sim Normal$ & 200 & FALSE & 0.20 & 2.00 & 3.00 & 0.42 & 0.11 \\ 
   & $\textit{x} + \textit{z} + \textit{x} \times \textit{z}$ & $\textit{x} \sim Binary, \textit{z} \sim Binary$ & 200 & FALSE & 0.20 & 2.00 & 3.00 & 0.97 & 0.34 \\ 
   & $\textit{x} + \textit{z} + \textit{x} \times \textit{z}$ & $\textit{x} \sim Normal, \textit{z} \sim Binary$ & 200 & FALSE & 0.20 & 2.00 & 3.00 & 0.96 & 0.34 \\ 
   & $\textit{x} + \textit{z} + \textit{x} \times \textit{z}$ & $\textit{x} \sim Binary, \textit{z} \sim Normal$ & 200 & FALSE & 0.20 & 2.00 & 3.00 & 0.43 & 0.09 \\ 
   & $\textit{x} + \textit{z} + \textit{z} \times \textit{z}$ & $\textit{x} \sim Normal , \textit{z} \sim Normal$ & 200 & FALSE & 0.20 & 2.00 & 1.00 & 0.07 & 0.05 \\ 
   & $\textit{x} + \textit{z} + \textit{z} \times \textit{z}$ & $\textit{x} \sim Normal , \textit{z} \sim Normal$ & 200 & FALSE & 0.20 & 3.00 & 1.00 & 0.09 & 0.05 \\ 
   & $\textit{x} + \textit{z} + \textit{z} \times \textit{z}$ & $\textit{x} \sim Binary, \textit{z} \sim Binary$ & 200 & FALSE & 0.20 & 2.00 & 1.00 & 0.07 & 0.05 \\ 
   & $\textit{x} + \textit{z} + \textit{z} \times \textit{z}$ & $\textit{x} \sim Binary, \textit{z} \sim Binary$ & 200 & FALSE & 0.20 & 3.00 & 1.00 & 0.08 & 0.05 \\ 
   & $\textit{x} + \textit{z} + \textit{z} \times \textit{z}$ & $\textit{x} \sim Normal, \textit{z} \sim Binary$ & 200 & FALSE & 0.20 & 2.00 & 1.00 & 0.07 & 0.05 \\ 
   & $\textit{x} + \textit{z} + \textit{z} \times \textit{z}$ & $\textit{x} \sim Normal, \textit{z} \sim Binary$ & 200 & FALSE & 0.20 & 3.00 & 1.00 & 0.09 & 0.05 \\ 
   & $\textit{x} + \textit{z} + \textit{z} \times \textit{z}$ & $\textit{x} \sim Binary, \textit{z} \sim Normal$ & 200 & FALSE & 0.20 & 2.00 & 1.00 & 0.07 & 0.05 \\ 
   & $\textit{x} + \textit{z} + \textit{z} \times \textit{z}$ & $\textit{x} \sim Binary, \textit{z} \sim Normal$ & 200 & FALSE & 0.20 & 3.00 & 1.00 & 0.08 & 0.05 \\ 
   & $\textit{x} + \textit{z} + \textit{z} \times \textit{z}$ & $\textit{x} \sim Normal , \textit{z} \sim Normal$ & 200 & FALSE & 0.20 & 2.00 & 1.00 & 0.07 & 0.05 \\ 
   & $\textit{x} + \textit{z} + \textit{z} \times \textit{z}$ & $\textit{x} \sim Normal , \textit{z} \sim Normal$ & 200 & FALSE & 0.20 & 3.00 & 1.00 & 0.10 & 0.05 \\ 
   & $\textit{x} + \textit{z} + \textit{z} \times \textit{z}$ & $\textit{x} \sim Binary, \textit{z} \sim Binary$ & 200 & FALSE & 0.20 & 2.00 & 1.00 & 0.07 & 0.05 \\ 
   & $\textit{x} + \textit{z} + \textit{z} \times \textit{z}$ & $\textit{x} \sim Binary, \textit{z} \sim Binary$ & 200 & FALSE & 0.20 & 3.00 & 1.00 & 0.10 & 0.05 \\ 
   & $\textit{x} + \textit{z} + \textit{z} \times \textit{z}$ & $\textit{x} \sim Normal, \textit{z} \sim Binary$ & 200 & FALSE & 0.20 & 2.00 & 1.00 & 0.07 & 0.05 \\ 
   & $\textit{x} + \textit{z} + \textit{z} \times \textit{z}$ & $\textit{x} \sim Normal, \textit{z} \sim Binary$ & 200 & FALSE & 0.20 & 3.00 & 1.00 & 0.10 & 0.05 \\ 
   & $\textit{x} + \textit{z} + \textit{z} \times \textit{z}$ & $\textit{x} \sim Binary, \textit{z} \sim Normal$ & 200 & FALSE & 0.20 & 2.00 & 1.00 & 0.07 & 0.05 \\ 
   & $\textit{x} + \textit{z} + \textit{z} \times \textit{z}$ & $\textit{x} \sim Binary, \textit{z} \sim Normal$ & 200 & FALSE & 0.20 & 3.00 & 1.00 & 0.10 & 0.05 \\ 
   & $\textit{x} + \textit{z} + \textit{z} \times \textit{z}$ & $\textit{x} \sim Normal , \textit{z} \sim Normal$ & 100 & FALSE & 0.20 & 2.00 & 1.00 & 0.07 & 0.05 \\ 
   & $\textit{x} + \textit{z} + \textit{z} \times \textit{z}$ & $\textit{x} \sim Binary, \textit{z} \sim Binary$ & 100 & FALSE & 0.20 & 2.00 & 1.00 & 0.07 & 0.05 \\ 
   & $\textit{x} + \textit{z} + \textit{z} \times \textit{z}$ & $\textit{x} \sim Normal, \textit{z} \sim Binary$ & 100 & FALSE & 0.20 & 2.00 & 1.00 & 0.07 & 0.05 \\ 
   & $\textit{x} + \textit{z} + \textit{z} \times \textit{z}$ & $\textit{x} \sim Binary, \textit{z} \sim Normal$ & 100 & FALSE & 0.20 & 2.00 & 1.00 & 0.07 & 0.05 \\ 
   & $\textit{x} + \textit{z} + \textit{z} \times \textit{z}$ & $\textit{x} \sim Normal , \textit{z} \sim Normal$ & 150 & FALSE & 0.20 & 2.00 & 1.00 & 0.07 & 0.05 \\ 
   & $\textit{x} + \textit{z} + \textit{z} \times \textit{z}$ & $\textit{x} \sim Binary, \textit{z} \sim Binary$ & 150 & FALSE & 0.20 & 2.00 & 1.00 & 0.07 & 0.05 \\ 
   & $\textit{x} + \textit{z} + \textit{z} \times \textit{z}$ & $\textit{x} \sim Normal, \textit{z} \sim Binary$ & 150 & FALSE & 0.20 & 2.00 & 1.00 & 0.07 & 0.05 \\ 
   & $\textit{x} + \textit{z} + \textit{z} \times \textit{z}$ & $\textit{x} \sim Binary, \textit{z} \sim Normal$ & 150 & FALSE & 0.20 & 2.00 & 1.00 & 0.07 & 0.05 \\ 
   & $\textit{x} + \textit{z} + \textit{z} \times \textit{z}$ & $\textit{x} \sim Normal , \textit{z} \sim Normal$ & 250 & FALSE & 0.20 & 2.00 & 1.00 & 0.07 & 0.05 \\ 
   & $\textit{x} + \textit{z} + \textit{z} \times \textit{z}$ & $\textit{x} \sim Binary, \textit{z} \sim Binary$ & 250 & FALSE & 0.20 & 2.00 & 1.00 & 0.07 & 0.05 \\ 
   & $\textit{x} + \textit{z} + \textit{z} \times \textit{z}$ & $\textit{x} \sim Normal, \textit{z} \sim Binary$ & 250 & FALSE & 0.20 & 2.00 & 1.00 & 0.07 & 0.05 \\ 
   & $\textit{x} + \textit{z} + \textit{z} \times \textit{z}$ & $\textit{x} \sim Binary, \textit{z} \sim Normal$ & 250 & FALSE & 0.20 & 2.00 & 1.00 & 0.07 & 0.05 \\ 
   & $\textit{x} + \textit{z} + \textit{z} \times \textit{z}$ & $\textit{x} \sim Normal , \textit{z} \sim Normal$ & 300 & FALSE & 0.20 & 2.00 & 1.00 & 0.07 & 0.05 \\ 
   & $\textit{x} + \textit{z} + \textit{z} \times \textit{z}$ & $\textit{x} \sim Binary, \textit{z} \sim Binary$ & 300 & FALSE & 0.20 & 2.00 & 1.00 & 0.07 & 0.05 \\ 
   & $\textit{x} + \textit{z} + \textit{z} \times \textit{z}$ & $\textit{x} \sim Normal, \textit{z} \sim Binary$ & 300 & FALSE & 0.20 & 2.00 & 1.00 & 0.07 & 0.05 \\ 
   & $\textit{x} + \textit{z} + \textit{z} \times \textit{z}$ & $\textit{x} \sim Binary, \textit{z} \sim Normal$ & 300 & FALSE & 0.20 & 2.00 & 1.00 & 0.07 & 0.05 \\ 
   & $\textit{x} + \textit{z} + \textit{z} \times \textit{z}$ & $\textit{x} \sim Normal , \textit{z} \sim Normal$ & 100 & FALSE & 0.20 & 2.00 & 1.00 & 0.07 & 0.05 \\ 
   & $\textit{x} + \textit{z} + \textit{z} \times \textit{z}$ & $\textit{x} \sim Binary, \textit{z} \sim Binary$ & 100 & FALSE & 0.20 & 2.00 & 1.00 & 0.07 & 0.05 \\ 
   & $\textit{x} + \textit{z} + \textit{z} \times \textit{z}$ & $\textit{x} \sim Normal, \textit{z} \sim Binary$ & 100 & FALSE & 0.20 & 2.00 & 1.00 & 0.07 & 0.05 \\ 
   & $\textit{x} + \textit{z} + \textit{z} \times \textit{z}$ & $\textit{x} \sim Binary, \textit{z} \sim Normal$ & 100 & FALSE & 0.20 & 2.00 & 1.00 & 0.07 & 0.05 \\ 
   & $\textit{x} + \textit{z} + \textit{z} \times \textit{z}$ & $\textit{x} \sim Normal , \textit{z} \sim Normal$ & 150 & FALSE & 0.20 & 2.00 & 1.00 & 0.08 & 0.05 \\ 
   & $\textit{x} + \textit{z} + \textit{z} \times \textit{z}$ & $\textit{x} \sim Binary, \textit{z} \sim Binary$ & 150 & FALSE & 0.20 & 2.00 & 1.00 & 0.07 & 0.05 \\ 
   & $\textit{x} + \textit{z} + \textit{z} \times \textit{z}$ & $\textit{x} \sim Normal, \textit{z} \sim Binary$ & 150 & FALSE & 0.20 & 2.00 & 1.00 & 0.07 & 0.05 \\ 
   & $\textit{x} + \textit{z} + \textit{z} \times \textit{z}$ & $\textit{x} \sim Binary, \textit{z} \sim Normal$ & 150 & FALSE & 0.20 & 2.00 & 1.00 & 0.07 & 0.05 \\ 
   & $\textit{x} + \textit{z} + \textit{z} \times \textit{z}$ & $\textit{x} \sim Normal , \textit{z} \sim Normal$ & 250 & FALSE & 0.20 & 2.00 & 1.00 & 0.08 & 0.05 \\ 
   & $\textit{x} + \textit{z} + \textit{z} \times \textit{z}$ & $\textit{x} \sim Binary, \textit{z} \sim Binary$ & 250 & FALSE & 0.20 & 2.00 & 1.00 & 0.07 & 0.05 \\ 
   & $\textit{x} + \textit{z} + \textit{z} \times \textit{z}$ & $\textit{x} \sim Normal, \textit{z} \sim Binary$ & 250 & FALSE & 0.20 & 2.00 & 1.00 & 0.07 & 0.05 \\ 
   & $\textit{x} + \textit{z} + \textit{z} \times \textit{z}$ & $\textit{x} \sim Binary, \textit{z} \sim Normal$ & 250 & FALSE & 0.20 & 2.00 & 1.00 & 0.07 & 0.05 \\ 
   & $\textit{x} + \textit{z} + \textit{z} \times \textit{z}$ & $\textit{x} \sim Normal , \textit{z} \sim Normal$ & 300 & FALSE & 0.20 & 2.00 & 1.00 & 0.07 & 0.05 \\ 
   & $\textit{x} + \textit{z} + \textit{z} \times \textit{z}$ & $\textit{x} \sim Binary, \textit{z} \sim Binary$ & 300 & FALSE & 0.20 & 2.00 & 1.00 & 0.07 & 0.05 \\ 
   & $\textit{x} + \textit{z} + \textit{z} \times \textit{z}$ & $\textit{x} \sim Normal, \textit{z} \sim Binary$ & 300 & FALSE & 0.20 & 2.00 & 1.00 & 0.07 & 0.05 \\ 
   & $\textit{x} + \textit{z} + \textit{z} \times \textit{z}$ & $\textit{x} \sim Binary, \textit{z} \sim Normal$ & 300 & FALSE & 0.20 & 2.00 & 1.00 & 0.07 & 0.05 \\ 
   & $\textit{x} + \textit{z} + \textit{z} \times \textit{z}$ & $\textit{x} \sim Normal , \textit{z} \sim Normal$ & 200 & FALSE & 0.30 & 2.00 & 1.00 & 0.08 & 0.05 \\ 
   & $\textit{x} + \textit{z} + \textit{z} \times \textit{z}$ & $\textit{x} \sim Binary, \textit{z} \sim Binary$ & 200 & FALSE & 0.30 & 2.00 & 1.00 & 0.08 & 0.05 \\ 
   & $\textit{x} + \textit{z} + \textit{z} \times \textit{z}$ & $\textit{x} \sim Normal, \textit{z} \sim Binary$ & 200 & FALSE & 0.30 & 2.00 & 1.00 & 0.08 & 0.05 \\ 
   & $\textit{x} + \textit{z} + \textit{z} \times \textit{z}$ & $\textit{x} \sim Binary, \textit{z} \sim Normal$ & 200 & FALSE & 0.30 & 2.00 & 1.00 & 0.07 & 0.05 \\ 
   & $\textit{x} + \textit{z} + \textit{z} \times \textit{z}$ & $\textit{x} \sim Normal , \textit{z} \sim Normal$ & 200 & FALSE & 0.40 & 2.00 & 1.00 & 0.09 & 0.05 \\ 
   & $\textit{x} + \textit{z} + \textit{z} \times \textit{z}$ & $\textit{x} \sim Binary, \textit{z} \sim Binary$ & 200 & FALSE & 0.40 & 2.00 & 1.00 & 0.09 & 0.05 \\ 
   & $\textit{x} + \textit{z} + \textit{z} \times \textit{z}$ & $\textit{x} \sim Normal, \textit{z} \sim Binary$ & 200 & FALSE & 0.40 & 2.00 & 1.00 & 0.09 & 0.05 \\ 
   & $\textit{x} + \textit{z} + \textit{z} \times \textit{z}$ & $\textit{x} \sim Binary, \textit{z} \sim Normal$ & 200 & FALSE & 0.40 & 2.00 & 1.00 & 0.09 & 0.05 \\ 
   & $\textit{x} + \textit{z} + \textit{z} \times \textit{z}$ & $\textit{x} \sim Normal , \textit{z} \sim Normal$ & 200 & FALSE & 0.30 & 2.00 & 1.00 & 0.08 & 0.05 \\ 
   & $\textit{x} + \textit{z} + \textit{z} \times \textit{z}$ & $\textit{x} \sim Binary, \textit{z} \sim Binary$ & 200 & FALSE & 0.30 & 2.00 & 1.00 & 0.09 & 0.05 \\ 
   & $\textit{x} + \textit{z} + \textit{z} \times \textit{z}$ & $\textit{x} \sim Normal, \textit{z} \sim Binary$ & 200 & FALSE & 0.30 & 2.00 & 1.00 & 0.08 & 0.05 \\ 
   & $\textit{x} + \textit{z} + \textit{z} \times \textit{z}$ & $\textit{x} \sim Binary, \textit{z} \sim Normal$ & 200 & FALSE & 0.30 & 2.00 & 1.00 & 0.08 & 0.05 \\ 
   & $\textit{x} + \textit{z} + \textit{z} \times \textit{z}$ & $\textit{x} \sim Normal , \textit{z} \sim Normal$ & 200 & FALSE & 0.40 & 2.00 & 1.00 & 0.10 & 0.05 \\ 
   & $\textit{x} + \textit{z} + \textit{z} \times \textit{z}$ & $\textit{x} \sim Binary, \textit{z} \sim Binary$ & 200 & FALSE & 0.40 & 2.00 & 1.00 & 0.10 & 0.05 \\ 
   & $\textit{x} + \textit{z} + \textit{z} \times \textit{z}$ & $\textit{x} \sim Normal, \textit{z} \sim Binary$ & 200 & FALSE & 0.40 & 2.00 & 1.00 & 0.10 & 0.05 \\ 
   & $\textit{x} + \textit{z} + \textit{z} \times \textit{z}$ & $\textit{x} \sim Binary, \textit{z} \sim Normal$ & 200 & FALSE & 0.40 & 2.00 & 1.00 & 0.10 & 0.05 \\ 
   & $\textit{x} + \textit{z} + \textit{z} \times \textit{z}$ & $\textit{x} \sim Normal , \textit{z} \sim Normal$ & 200 & TRUE & 0.20 & 2.00 & 1.00 & 0.14 & 0.05 \\ 
   & $\textit{x} + \textit{z} + \textit{z} \times \textit{z}$ & $\textit{x} \sim Binary, \textit{z} \sim Binary$ & 200 & TRUE & 0.20 & 2.00 & 1.00 & 0.11 & 0.05 \\ 
   & $\textit{x} + \textit{z} + \textit{z} \times \textit{z}$ & $\textit{x} \sim Normal, \textit{z} \sim Binary$ & 200 & TRUE & 0.20 & 2.00 & 1.00 & 0.12 & 0.05 \\ 
   & $\textit{x} + \textit{z} + \textit{z} \times \textit{z}$ & $\textit{x} \sim Binary, \textit{z} \sim Normal$ & 200 & TRUE & 0.20 & 2.00 & 1.00 & 0.14 & 0.05 \\ 
   & $\textit{x} + \textit{z} + \textit{z} \times \textit{z}$ & $\textit{x} \sim Normal , \textit{z} \sim Normal$ & 200 & TRUE & 0.20 & 2.00 & 1.00 & 0.14 & 0.05 \\ 
   & $\textit{x} + \textit{z} + \textit{z} \times \textit{z}$ & $\textit{x} \sim Binary, \textit{z} \sim Binary$ & 200 & TRUE & 0.20 & 2.00 & 1.00 & 0.11 & 0.05 \\ 
   & $\textit{x} + \textit{z} + \textit{z} \times \textit{z}$ & $\textit{x} \sim Normal, \textit{z} \sim Binary$ & 200 & TRUE & 0.20 & 2.00 & 1.00 & 0.12 & 0.05 \\ 
   & $\textit{x} + \textit{z} + \textit{z} \times \textit{z}$ & $\textit{x} \sim Binary, \textit{z} \sim Normal$ & 200 & TRUE & 0.20 & 2.00 & 1.00 & 0.13 & 0.05 \\ 
   & $\textit{x} + \textit{z} + \textit{z} \times \textit{z}$ & $\textit{x} \sim Normal , \textit{z} \sim Normal$ & 200 & FALSE & 0.20 & 2.00 & 3.00 & 0.13 & 0.05 \\ 
   & $\textit{x} + \textit{z} + \textit{z} \times \textit{z}$ & $\textit{x} \sim Binary, \textit{z} \sim Binary$ & 200 & FALSE & 0.20 & 2.00 & 3.00 & 0.14 & 0.05 \\ 
   & $\textit{x} + \textit{z} + \textit{z} \times \textit{z}$ & $\textit{x} \sim Normal, \textit{z} \sim Binary$ & 200 & FALSE & 0.20 & 2.00 & 3.00 & 0.13 & 0.05 \\ 
   & $\textit{x} + \textit{z} + \textit{z} \times \textit{z}$ & $\textit{x} \sim Binary, \textit{z} \sim Normal$ & 200 & FALSE & 0.20 & 2.00 & 3.00 & 0.13 & 0.05 \\ 
   & $\textit{x} + \textit{z} + \textit{z} \times \textit{z}$ & $\textit{x} \sim Normal , \textit{z} \sim Normal$ & 200 & FALSE & 0.20 & 2.00 & 3.00 & 0.14 & 0.05 \\ 
   & $\textit{x} + \textit{z} + \textit{z} \times \textit{z}$ & $\textit{x} \sim Binary, \textit{z} \sim Binary$ & 200 & FALSE & 0.20 & 2.00 & 3.00 & 0.14 & 0.05 \\ 
   & $\textit{x} + \textit{z} + \textit{z} \times \textit{z}$ & $\textit{x} \sim Normal, \textit{z} \sim Binary$ & 200 & FALSE & 0.20 & 2.00 & 3.00 & 0.14 & 0.05 \\ 
   & $\textit{x} + \textit{z} + \textit{z} \times \textit{z}$ & $\textit{x} \sim Binary, \textit{z} \sim Normal$ & 200 & FALSE & 0.20 & 2.00 & 3.00 & 0.14 & 0.05 \\ 
   & $\textit{x} \times \textit{z} + \textit{z} \times \textit{z}$ & $\textit{x} \sim Normal , \textit{z} \sim Normal$ & 200 & FALSE & 0.20 & 2.00 & 1.00 & 0.19 & 0.09 \\ 
   & $\textit{x} \times \textit{z} + \textit{z} \times \textit{z}$ & $\textit{x} \sim Normal , \textit{z} \sim Normal$ & 200 & FALSE & 0.20 & 3.00 & 1.00 & 0.29 & 0.13 \\ 
   & $\textit{x} \times \textit{z} + \textit{z} \times \textit{z}$ & $\textit{x} \sim Binary, \textit{z} \sim Binary$ & 200 & FALSE & 0.20 & 2.00 & 1.00 & 0.36 & 0.13 \\ 
   & $\textit{x} \times \textit{z} + \textit{z} \times \textit{z}$ & $\textit{x} \sim Binary, \textit{z} \sim Binary$ & 200 & FALSE & 0.20 & 3.00 & 1.00 & 0.93 & 0.30 \\ 
   & $\textit{x} \times \textit{z} + \textit{z} \times \textit{z}$ & $\textit{x} \sim Normal, \textit{z} \sim Binary$ & 200 & FALSE & 0.20 & 2.00 & 1.00 & 0.79 & 0.33 \\ 
   & $\textit{x} \times \textit{z} + \textit{z} \times \textit{z}$ & $\textit{x} \sim Normal, \textit{z} \sim Binary$ & 200 & FALSE & 0.20 & 3.00 & 1.00 & 0.94 & 0.61 \\ 
   & $\textit{x} \times \textit{z} + \textit{z} \times \textit{z}$ & $\textit{x} \sim Binary, \textit{z} \sim Normal$ & 200 & FALSE & 0.20 & 2.00 & 1.00 & 0.20 & 0.08 \\ 
   & $\textit{x} \times \textit{z} + \textit{z} \times \textit{z}$ & $\textit{x} \sim Binary, \textit{z} \sim Normal$ & 200 & FALSE & 0.20 & 3.00 & 1.00 & 0.35 & 0.10 \\ 
   & $\textit{x} \times \textit{z} + \textit{z} \times \textit{z}$ & $\textit{x} \sim Normal , \textit{z} \sim Normal$ & 100 & FALSE & 0.20 & 2.00 & 1.00 & 0.19 & 0.09 \\ 
   & $\textit{x} \times \textit{z} + \textit{z} \times \textit{z}$ & $\textit{x} \sim Binary, \textit{z} \sim Binary$ & 100 & FALSE & 0.20 & 2.00 & 1.00 & 0.29 & 0.10 \\ 
   & $\textit{x} \times \textit{z} + \textit{z} \times \textit{z}$ & $\textit{x} \sim Normal, \textit{z} \sim Binary$ & 100 & FALSE & 0.20 & 2.00 & 1.00 & 0.54 & 0.21 \\ 
   & $\textit{x} \times \textit{z} + \textit{z} \times \textit{z}$ & $\textit{x} \sim Binary, \textit{z} \sim Normal$ & 100 & FALSE & 0.20 & 2.00 & 1.00 & 0.20 & 0.07 \\ 
   & $\textit{x} \times \textit{z} + \textit{z} \times \textit{z}$ & $\textit{x} \sim Normal , \textit{z} \sim Normal$ & 150 & FALSE & 0.20 & 2.00 & 1.00 & 0.19 & 0.09 \\ 
   & $\textit{x} \times \textit{z} + \textit{z} \times \textit{z}$ & $\textit{x} \sim Binary, \textit{z} \sim Binary$ & 150 & FALSE & 0.20 & 2.00 & 1.00 & 0.32 & 0.12 \\ 
   & $\textit{x} \times \textit{z} + \textit{z} \times \textit{z}$ & $\textit{x} \sim Normal, \textit{z} \sim Binary$ & 150 & FALSE & 0.20 & 2.00 & 1.00 & 0.69 & 0.28 \\ 
   & $\textit{x} \times \textit{z} + \textit{z} \times \textit{z}$ & $\textit{x} \sim Binary, \textit{z} \sim Normal$ & 150 & FALSE & 0.20 & 2.00 & 1.00 & 0.20 & 0.07 \\ 
   & $\textit{x} \times \textit{z} + \textit{z} \times \textit{z}$ & $\textit{x} \sim Normal , \textit{z} \sim Normal$ & 250 & FALSE & 0.20 & 2.00 & 1.00 & 0.19 & 0.09 \\ 
   & $\textit{x} \times \textit{z} + \textit{z} \times \textit{z}$ & $\textit{x} \sim Binary, \textit{z} \sim Binary$ & 250 & FALSE & 0.20 & 2.00 & 1.00 & 0.40 & 0.14 \\ 
   & $\textit{x} \times \textit{z} + \textit{z} \times \textit{z}$ & $\textit{x} \sim Normal, \textit{z} \sim Binary$ & 250 & FALSE & 0.20 & 2.00 & 1.00 & 0.87 & 0.38 \\ 
   & $\textit{x} \times \textit{z} + \textit{z} \times \textit{z}$ & $\textit{x} \sim Binary, \textit{z} \sim Normal$ & 250 & FALSE & 0.20 & 2.00 & 1.00 & 0.20 & 0.07 \\ 
   & $\textit{x} \times \textit{z} + \textit{z} \times \textit{z}$ & $\textit{x} \sim Normal , \textit{z} \sim Normal$ & 300 & FALSE & 0.20 & 2.00 & 1.00 & 0.19 & 0.09 \\ 
   & $\textit{x} \times \textit{z} + \textit{z} \times \textit{z}$ & $\textit{x} \sim Binary, \textit{z} \sim Binary$ & 300 & FALSE & 0.20 & 2.00 & 1.00 & 0.44 & 0.15 \\ 
   & $\textit{x} \times \textit{z} + \textit{z} \times \textit{z}$ & $\textit{x} \sim Normal, \textit{z} \sim Binary$ & 300 & FALSE & 0.20 & 2.00 & 1.00 & 0.92 & 0.41 \\ 
   & $\textit{x} \times \textit{z} + \textit{z} \times \textit{z}$ & $\textit{x} \sim Binary, \textit{z} \sim Normal$ & 300 & FALSE & 0.20 & 2.00 & 1.00 & 0.20 & 0.07 \\ 
   & $\textit{x} \times \textit{z} + \textit{z} \times \textit{z}$ & $\textit{x} \sim Normal , \textit{z} \sim Normal$ & 200 & FALSE & 0.30 & 2.00 & 1.00 & 0.22 & 0.09 \\ 
   & $\textit{x} \times \textit{z} + \textit{z} \times \textit{z}$ & $\textit{x} \sim Binary, \textit{z} \sim Binary$ & 200 & FALSE & 0.30 & 2.00 & 1.00 & 0.57 & 0.20 \\ 
   & $\textit{x} \times \textit{z} + \textit{z} \times \textit{z}$ & $\textit{x} \sim Normal, \textit{z} \sim Binary$ & 200 & FALSE & 0.30 & 2.00 & 1.00 & 0.98 & 0.47 \\ 
   & $\textit{x} \times \textit{z} + \textit{z} \times \textit{z}$ & $\textit{x} \sim Binary, \textit{z} \sim Normal$ & 200 & FALSE & 0.30 & 2.00 & 1.00 & 0.21 & 0.07 \\ 
   & $\textit{x} \times \textit{z} + \textit{z} \times \textit{z}$ & $\textit{x} \sim Normal , \textit{z} \sim Normal$ & 200 & FALSE & 0.40 & 2.00 & 1.00 & 0.25 & 0.10 \\ 
   & $\textit{x} \times \textit{z} + \textit{z} \times \textit{z}$ & $\textit{x} \sim Binary, \textit{z} \sim Binary$ & 200 & FALSE & 0.40 & 2.00 & 1.00 & 0.79 & 0.29 \\ 
   & $\textit{x} \times \textit{z} + \textit{z} \times \textit{z}$ & $\textit{x} \sim Normal, \textit{z} \sim Binary$ & 200 & FALSE & 0.40 & 2.00 & 1.00 & 1.00 & 0.52 \\ 
   & $\textit{x} \times \textit{z} + \textit{z} \times \textit{z}$ & $\textit{x} \sim Binary, \textit{z} \sim Normal$ & 200 & FALSE & 0.40 & 2.00 & 1.00 & 0.22 & 0.08 \\ 
   & $\textit{x} \times \textit{z} + \textit{z} \times \textit{z}$ & $\textit{x} \sim Normal , \textit{z} \sim Normal$ & 200 & TRUE & 0.20 & 2.00 & 1.00 & 0.36 & 0.08 \\ 
   & $\textit{x} \times \textit{z} + \textit{z} \times \textit{z}$ & $\textit{x} \sim Binary, \textit{z} \sim Binary$ & 200 & TRUE & 0.20 & 2.00 & 1.00 & 0.48 & 0.13 \\ 
   & $\textit{x} \times \textit{z} + \textit{z} \times \textit{z}$ & $\textit{x} \sim Normal, \textit{z} \sim Binary$ & 200 & TRUE & 0.20 & 2.00 & 1.00 & 0.86 & 0.30 \\ 
   & $\textit{x} \times \textit{z} + \textit{z} \times \textit{z}$ & $\textit{x} \sim Binary, \textit{z} \sim Normal$ & 200 & TRUE & 0.20 & 2.00 & 1.00 & 0.35 & 0.07 \\ 
   & $\textit{x} \times \textit{z} + \textit{z} \times \textit{z}$ & $\textit{x} \sim Normal , \textit{z} \sim Normal$ & 200 & FALSE & 0.20 & 2.00 & 3.00 & 0.34 & 0.09 \\ 
   & $\textit{x} \times \textit{z} + \textit{z} \times \textit{z}$ & $\textit{x} \sim Binary, \textit{z} \sim Binary$ & 200 & FALSE & 0.20 & 2.00 & 3.00 & 0.56 & 0.13 \\ 
   & $\textit{x} \times \textit{z} + \textit{z} \times \textit{z}$ & $\textit{x} \sim Normal, \textit{z} \sim Binary$ & 200 & FALSE & 0.20 & 2.00 & 3.00 & 0.93 & 0.35 \\ 
   & $\textit{x} \times \textit{z} + \textit{z} \times \textit{z}$ & $\textit{x} \sim Binary, \textit{z} \sim Normal$ & 200 & FALSE & 0.20 & 2.00 & 3.00 & 0.35 & 0.07 \\ 
   & $\textit{x} + \textit{z} + \textit{x} \times \textit{z} + \textit{z} \times \textit{z}$ & $\textit{x} \sim Normal , \textit{z} \sim Normal$ & 200 & FALSE & 0.20 & 2.00 & 1.00 & 0.17 & 0.08 \\ 
   & $\textit{x} + \textit{z} + \textit{x} \times \textit{z} + \textit{z} \times \textit{z}$ & $\textit{x} \sim Normal , \textit{z} \sim Normal$ & 200 & FALSE & 0.20 & 3.00 & 1.00 & 0.30 & 0.12 \\ 
   & $\textit{x} + \textit{z} + \textit{x} \times \textit{z} + \textit{z} \times \textit{z}$ & $\textit{x} \sim Binary, \textit{z} \sim Binary$ & 200 & FALSE & 0.20 & 2.00 & 1.00 & 0.20 & 0.07 \\ 
   & $\textit{x} + \textit{z} + \textit{x} \times \textit{z} + \textit{z} \times \textit{z}$ & $\textit{x} \sim Binary, \textit{z} \sim Binary$ & 200 & FALSE & 0.20 & 3.00 & 1.00 & 0.34 & 0.11 \\ 
   & $\textit{x} + \textit{z} + \textit{x} \times \textit{z} + \textit{z} \times \textit{z}$ & $\textit{x} \sim Normal, \textit{z} \sim Binary$ & 200 & FALSE & 0.20 & 2.00 & 1.00 & 0.17 & 0.08 \\ 
   & $\textit{x} + \textit{z} + \textit{x} \times \textit{z} + \textit{z} \times \textit{z}$ & $\textit{x} \sim Normal, \textit{z} \sim Binary$ & 200 & FALSE & 0.20 & 3.00 & 1.00 & 0.29 & 0.12 \\ 
   & $\textit{x} + \textit{z} + \textit{x} \times \textit{z} + \textit{z} \times \textit{z}$ & $\textit{x} \sim Binary, \textit{z} \sim Normal$ & 200 & FALSE & 0.20 & 2.00 & 1.00 & 0.19 & 0.07 \\ 
   & $\textit{x} + \textit{z} + \textit{x} \times \textit{z} + \textit{z} \times \textit{z}$ & $\textit{x} \sim Binary, \textit{z} \sim Normal$ & 200 & FALSE & 0.20 & 3.00 & 1.00 & 0.34 & 0.10 \\ 
   & $\textit{x} + \textit{z} + \textit{x} \times \textit{z} + \textit{z} \times \textit{z}$ & $\textit{x} \sim Normal , \textit{z} \sim Normal$ & 200 & FALSE & 0.20 & 2.00 & 1.00 & 0.23 & 0.10 \\ 
   & $\textit{x} + \textit{z} + \textit{x} \times \textit{z} + \textit{z} \times \textit{z}$ & $\textit{x} \sim Normal , \textit{z} \sim Normal$ & 200 & FALSE & 0.20 & 3.00 & 1.00 & 0.42 & 0.13 \\ 
   & $\textit{x} + \textit{z} + \textit{x} \times \textit{z} + \textit{z} \times \textit{z}$ & $\textit{x} \sim Binary, \textit{z} \sim Binary$ & 200 & FALSE & 0.20 & 2.00 & 1.00 & 0.51 & 0.14 \\ 
   & $\textit{x} + \textit{z} + \textit{x} \times \textit{z} + \textit{z} \times \textit{z}$ & $\textit{x} \sim Binary, \textit{z} \sim Binary$ & 200 & FALSE & 0.20 & 3.00 & 1.00 & 0.96 & 0.23 \\ 
   & $\textit{x} + \textit{z} + \textit{x} \times \textit{z} + \textit{z} \times \textit{z}$ & $\textit{x} \sim Normal, \textit{z} \sim Binary$ & 200 & FALSE & 0.20 & 2.00 & 1.00 & 0.82 & 0.24 \\ 
   & $\textit{x} + \textit{z} + \textit{x} \times \textit{z} + \textit{z} \times \textit{z}$ & $\textit{x} \sim Normal, \textit{z} \sim Binary$ & 200 & FALSE & 0.20 & 3.00 & 1.00 & 0.96 & 0.41 \\ 
   & $\textit{x} + \textit{z} + \textit{x} \times \textit{z} + \textit{z} \times \textit{z}$ & $\textit{x} \sim Binary, \textit{z} \sim Normal$ & 200 & FALSE & 0.20 & 2.00 & 1.00 & 0.23 & 0.09 \\ 
   & $\textit{x} + \textit{z} + \textit{x} \times \textit{z} + \textit{z} \times \textit{z}$ & $\textit{x} \sim Binary, \textit{z} \sim Normal$ & 200 & FALSE & 0.20 & 3.00 & 1.00 & 0.41 & 0.11 \\ 
   & $\textit{x} + \textit{z} + \textit{x} \times \textit{z} + \textit{z} \times \textit{z}$ & $\textit{x} \sim Normal , \textit{z} \sim Normal$ & 100 & FALSE & 0.20 & 2.00 & 1.00 & 0.17 & 0.08 \\ 
   & $\textit{x} + \textit{z} + \textit{x} \times \textit{z} + \textit{z} \times \textit{z}$ & $\textit{x} \sim Binary, \textit{z} \sim Binary$ & 100 & FALSE & 0.20 & 2.00 & 1.00 & 0.20 & 0.07 \\ 
   & $\textit{x} + \textit{z} + \textit{x} \times \textit{z} + \textit{z} \times \textit{z}$ & $\textit{x} \sim Normal, \textit{z} \sim Binary$ & 100 & FALSE & 0.20 & 2.00 & 1.00 & 0.17 & 0.08 \\ 
   & $\textit{x} + \textit{z} + \textit{x} \times \textit{z} + \textit{z} \times \textit{z}$ & $\textit{x} \sim Binary, \textit{z} \sim Normal$ & 100 & FALSE & 0.20 & 2.00 & 1.00 & 0.21 & 0.08 \\ 
   & $\textit{x} + \textit{z} + \textit{x} \times \textit{z} + \textit{z} \times \textit{z}$ & $\textit{x} \sim Normal , \textit{z} \sim Normal$ & 150 & FALSE & 0.20 & 2.00 & 1.00 & 0.17 & 0.08 \\ 
   & $\textit{x} + \textit{z} + \textit{x} \times \textit{z} + \textit{z} \times \textit{z}$ & $\textit{x} \sim Binary, \textit{z} \sim Binary$ & 150 & FALSE & 0.20 & 2.00 & 1.00 & 0.20 & 0.08 \\ 
   & $\textit{x} + \textit{z} + \textit{x} \times \textit{z} + \textit{z} \times \textit{z}$ & $\textit{x} \sim Normal, \textit{z} \sim Binary$ & 150 & FALSE & 0.20 & 2.00 & 1.00 & 0.17 & 0.08 \\ 
   & $\textit{x} + \textit{z} + \textit{x} \times \textit{z} + \textit{z} \times \textit{z}$ & $\textit{x} \sim Binary, \textit{z} \sim Normal$ & 150 & FALSE & 0.20 & 2.00 & 1.00 & 0.20 & 0.07 \\ 
   & $\textit{x} + \textit{z} + \textit{x} \times \textit{z} + \textit{z} \times \textit{z}$ & $\textit{x} \sim Normal , \textit{z} \sim Normal$ & 250 & FALSE & 0.20 & 2.00 & 1.00 & 0.17 & 0.08 \\ 
   & $\textit{x} + \textit{z} + \textit{x} \times \textit{z} + \textit{z} \times \textit{z}$ & $\textit{x} \sim Binary, \textit{z} \sim Binary$ & 250 & FALSE & 0.20 & 2.00 & 1.00 & 0.19 & 0.07 \\ 
   & $\textit{x} + \textit{z} + \textit{x} \times \textit{z} + \textit{z} \times \textit{z}$ & $\textit{x} \sim Normal, \textit{z} \sim Binary$ & 250 & FALSE & 0.20 & 2.00 & 1.00 & 0.17 & 0.08 \\ 
   & $\textit{x} + \textit{z} + \textit{x} \times \textit{z} + \textit{z} \times \textit{z}$ & $\textit{x} \sim Binary, \textit{z} \sim Normal$ & 250 & FALSE & 0.20 & 2.00 & 1.00 & 0.20 & 0.08 \\ 
   & $\textit{x} + \textit{z} + \textit{x} \times \textit{z} + \textit{z} \times \textit{z}$ & $\textit{x} \sim Normal , \textit{z} \sim Normal$ & 300 & FALSE & 0.20 & 2.00 & 1.00 & 0.16 & 0.08 \\ 
   & $\textit{x} + \textit{z} + \textit{x} \times \textit{z} + \textit{z} \times \textit{z}$ & $\textit{x} \sim Binary, \textit{z} \sim Binary$ & 300 & FALSE & 0.20 & 2.00 & 1.00 & 0.20 & 0.07 \\ 
   & $\textit{x} + \textit{z} + \textit{x} \times \textit{z} + \textit{z} \times \textit{z}$ & $\textit{x} \sim Normal, \textit{z} \sim Binary$ & 300 & FALSE & 0.20 & 2.00 & 1.00 & 0.16 & 0.08 \\ 
   & $\textit{x} + \textit{z} + \textit{x} \times \textit{z} + \textit{z} \times \textit{z}$ & $\textit{x} \sim Binary, \textit{z} \sim Normal$ & 300 & FALSE & 0.20 & 2.00 & 1.00 & 0.20 & 0.07 \\ 
   & $\textit{x} + \textit{z} + \textit{x} \times \textit{z} + \textit{z} \times \textit{z}$ & $\textit{x} \sim Normal , \textit{z} \sim Normal$ & 100 & FALSE & 0.20 & 2.00 & 1.00 & 0.23 & 0.10 \\ 
   & $\textit{x} + \textit{z} + \textit{x} \times \textit{z} + \textit{z} \times \textit{z}$ & $\textit{x} \sim Binary, \textit{z} \sim Binary$ & 100 & FALSE & 0.20 & 2.00 & 1.00 & 0.39 & 0.11 \\ 
   & $\textit{x} + \textit{z} + \textit{x} \times \textit{z} + \textit{z} \times \textit{z}$ & $\textit{x} \sim Normal, \textit{z} \sim Binary$ & 100 & FALSE & 0.20 & 2.00 & 1.00 & 0.58 & 0.17 \\ 
   & $\textit{x} + \textit{z} + \textit{x} \times \textit{z} + \textit{z} \times \textit{z}$ & $\textit{x} \sim Binary, \textit{z} \sim Normal$ & 100 & FALSE & 0.20 & 2.00 & 1.00 & 0.23 & 0.09 \\ 
   & $\textit{x} + \textit{z} + \textit{x} \times \textit{z} + \textit{z} \times \textit{z}$ & $\textit{x} \sim Normal , \textit{z} \sim Normal$ & 150 & FALSE & 0.20 & 2.00 & 1.00 & 0.23 & 0.10 \\ 
   & $\textit{x} + \textit{z} + \textit{x} \times \textit{z} + \textit{z} \times \textit{z}$ & $\textit{x} \sim Binary, \textit{z} \sim Binary$ & 150 & FALSE & 0.20 & 2.00 & 1.00 & 0.46 & 0.13 \\ 
   & $\textit{x} + \textit{z} + \textit{x} \times \textit{z} + \textit{z} \times \textit{z}$ & $\textit{x} \sim Normal, \textit{z} \sim Binary$ & 150 & FALSE & 0.20 & 2.00 & 1.00 & 0.71 & 0.21 \\ 
   & $\textit{x} + \textit{z} + \textit{x} \times \textit{z} + \textit{z} \times \textit{z}$ & $\textit{x} \sim Binary, \textit{z} \sim Normal$ & 150 & FALSE & 0.20 & 2.00 & 1.00 & 0.23 & 0.09 \\ 
   & $\textit{x} + \textit{z} + \textit{x} \times \textit{z} + \textit{z} \times \textit{z}$ & $\textit{x} \sim Normal , \textit{z} \sim Normal$ & 250 & FALSE & 0.20 & 2.00 & 1.00 & 0.22 & 0.10 \\ 
   & $\textit{x} + \textit{z} + \textit{x} \times \textit{z} + \textit{z} \times \textit{z}$ & $\textit{x} \sim Binary, \textit{z} \sim Binary$ & 250 & FALSE & 0.20 & 2.00 & 1.00 & 0.56 & 0.16 \\ 
   & $\textit{x} + \textit{z} + \textit{x} \times \textit{z} + \textit{z} \times \textit{z}$ & $\textit{x} \sim Normal, \textit{z} \sim Binary$ & 250 & FALSE & 0.20 & 2.00 & 1.00 & 0.88 & 0.27 \\ 
   & $\textit{x} + \textit{z} + \textit{x} \times \textit{z} + \textit{z} \times \textit{z}$ & $\textit{x} \sim Binary, \textit{z} \sim Normal$ & 250 & FALSE & 0.20 & 2.00 & 1.00 & 0.23 & 0.09 \\ 
   & $\textit{x} + \textit{z} + \textit{x} \times \textit{z} + \textit{z} \times \textit{z}$ & $\textit{x} \sim Normal , \textit{z} \sim Normal$ & 300 & FALSE & 0.20 & 2.00 & 1.00 & 0.23 & 0.10 \\ 
   & $\textit{x} + \textit{z} + \textit{x} \times \textit{z} + \textit{z} \times \textit{z}$ & $\textit{x} \sim Binary, \textit{z} \sim Binary$ & 300 & FALSE & 0.20 & 2.00 & 1.00 & 0.61 & 0.17 \\ 
   & $\textit{x} + \textit{z} + \textit{x} \times \textit{z} + \textit{z} \times \textit{z}$ & $\textit{x} \sim Normal, \textit{z} \sim Binary$ & 300 & FALSE & 0.20 & 2.00 & 1.00 & 0.93 & 0.30 \\ 
   & $\textit{x} + \textit{z} + \textit{x} \times \textit{z} + \textit{z} \times \textit{z}$ & $\textit{x} \sim Binary, \textit{z} \sim Normal$ & 300 & FALSE & 0.20 & 2.00 & 1.00 & 0.22 & 0.09 \\ 
   & $\textit{x} + \textit{z} + \textit{x} \times \textit{z} + \textit{z} \times \textit{z}$ & $\textit{x} \sim Normal , \textit{z} \sim Normal$ & 200 & FALSE & 0.30 & 2.00 & 1.00 & 0.18 & 0.08 \\ 
   & $\textit{x} + \textit{z} + \textit{x} \times \textit{z} + \textit{z} \times \textit{z}$ & $\textit{x} \sim Binary, \textit{z} \sim Binary$ & 200 & FALSE & 0.30 & 2.00 & 1.00 & 0.20 & 0.07 \\ 
   & $\textit{x} + \textit{z} + \textit{x} \times \textit{z} + \textit{z} \times \textit{z}$ & $\textit{x} \sim Normal, \textit{z} \sim Binary$ & 200 & FALSE & 0.30 & 2.00 & 1.00 & 0.17 & 0.08 \\ 
   & $\textit{x} + \textit{z} + \textit{x} \times \textit{z} + \textit{z} \times \textit{z}$ & $\textit{x} \sim Binary, \textit{z} \sim Normal$ & 200 & FALSE & 0.30 & 2.00 & 1.00 & 0.21 & 0.07 \\ 
   & $\textit{x} + \textit{z} + \textit{x} \times \textit{z} + \textit{z} \times \textit{z}$ & $\textit{x} \sim Normal , \textit{z} \sim Normal$ & 200 & FALSE & 0.40 & 2.00 & 1.00 & 0.19 & 0.08 \\ 
   & $\textit{x} + \textit{z} + \textit{x} \times \textit{z} + \textit{z} \times \textit{z}$ & $\textit{x} \sim Binary, \textit{z} \sim Binary$ & 200 & FALSE & 0.40 & 2.00 & 1.00 & 0.21 & 0.07 \\ 
   & $\textit{x} + \textit{z} + \textit{x} \times \textit{z} + \textit{z} \times \textit{z}$ & $\textit{x} \sim Normal, \textit{z} \sim Binary$ & 200 & FALSE & 0.40 & 2.00 & 1.00 & 0.18 & 0.08 \\ 
   & $\textit{x} + \textit{z} + \textit{x} \times \textit{z} + \textit{z} \times \textit{z}$ & $\textit{x} \sim Binary, \textit{z} \sim Normal$ & 200 & FALSE & 0.40 & 2.00 & 1.00 & 0.21 & 0.07 \\ 
   & $\textit{x} + \textit{z} + \textit{x} \times \textit{z} + \textit{z} \times \textit{z}$ & $\textit{x} \sim Normal , \textit{z} \sim Normal$ & 200 & FALSE & 0.30 & 2.00 & 1.00 & 0.27 & 0.10 \\ 
   & $\textit{x} + \textit{z} + \textit{x} \times \textit{z} + \textit{z} \times \textit{z}$ & $\textit{x} \sim Binary, \textit{z} \sim Binary$ & 200 & FALSE & 0.30 & 2.00 & 1.00 & 0.77 & 0.22 \\ 
   & $\textit{x} + \textit{z} + \textit{x} \times \textit{z} + \textit{z} \times \textit{z}$ & $\textit{x} \sim Normal, \textit{z} \sim Binary$ & 200 & FALSE & 0.30 & 2.00 & 1.00 & 0.99 & 0.35 \\ 
   & $\textit{x} + \textit{z} + \textit{x} \times \textit{z} + \textit{z} \times \textit{z}$ & $\textit{x} \sim Binary, \textit{z} \sim Normal$ & 200 & FALSE & 0.30 & 2.00 & 1.00 & 0.25 & 0.09 \\ 
   & $\textit{x} + \textit{z} + \textit{x} \times \textit{z} + \textit{z} \times \textit{z}$ & $\textit{x} \sim Normal , \textit{z} \sim Normal$ & 200 & FALSE & 0.40 & 2.00 & 1.00 & 0.33 & 0.11 \\ 
   & $\textit{x} + \textit{z} + \textit{x} \times \textit{z} + \textit{z} \times \textit{z}$ & $\textit{x} \sim Binary, \textit{z} \sim Binary$ & 200 & FALSE & 0.40 & 2.00 & 1.00 & 0.96 & 0.31 \\ 
   & $\textit{x} + \textit{z} + \textit{x} \times \textit{z} + \textit{z} \times \textit{z}$ & $\textit{x} \sim Normal, \textit{z} \sim Binary$ & 200 & FALSE & 0.40 & 2.00 & 1.00 & 1.00 & 0.39 \\ 
   & $\textit{x} + \textit{z} + \textit{x} \times \textit{z} + \textit{z} \times \textit{z}$ & $\textit{x} \sim Binary, \textit{z} \sim Normal$ & 200 & FALSE & 0.40 & 2.00 & 1.00 & 0.28 & 0.09 \\ 
   & $\textit{x} + \textit{z} + \textit{x} \times \textit{z} + \textit{z} \times \textit{z}$ & $\textit{x} \sim Normal , \textit{z} \sim Normal$ & 200 & TRUE & 0.20 & 2.00 & 1.00 & 0.41 & 0.10 \\ 
   & $\textit{x} + \textit{z} + \textit{x} \times \textit{z} + \textit{z} \times \textit{z}$ & $\textit{x} \sim Binary, \textit{z} \sim Binary$ & 200 & TRUE & 0.20 & 2.00 & 1.00 & 0.63 & 0.14 \\ 
   & $\textit{x} + \textit{z} + \textit{x} \times \textit{z} + \textit{z} \times \textit{z}$ & $\textit{x} \sim Normal, \textit{z} \sim Binary$ & 200 & TRUE & 0.20 & 2.00 & 1.00 & 0.89 & 0.22 \\ 
   & $\textit{x} + \textit{z} + \textit{x} \times \textit{z} + \textit{z} \times \textit{z}$ & $\textit{x} \sim Binary, \textit{z} \sim Normal$ & 200 & TRUE & 0.20 & 2.00 & 1.00 & 0.38 & 0.09 \\ 
   & $\textit{x} + \textit{z} + \textit{x} \times \textit{z} + \textit{z} \times \textit{z}$ & $\textit{x} \sim Normal , \textit{z} \sim Normal$ & 200 & TRUE & 0.20 & 2.00 & 1.00 & 0.33 & 0.08 \\ 
   & $\textit{x} + \textit{z} + \textit{x} \times \textit{z} + \textit{z} \times \textit{z}$ & $\textit{x} \sim Binary, \textit{z} \sim Binary$ & 200 & TRUE & 0.20 & 2.00 & 1.00 & 0.30 & 0.07 \\ 
   & $\textit{x} + \textit{z} + \textit{x} \times \textit{z} + \textit{z} \times \textit{z}$ & $\textit{x} \sim Normal, \textit{z} \sim Binary$ & 200 & TRUE & 0.20 & 2.00 & 1.00 & 0.30 & 0.08 \\ 
   & $\textit{x} + \textit{z} + \textit{x} \times \textit{z} + \textit{z} \times \textit{z}$ & $\textit{x} \sim Binary, \textit{z} \sim Normal$ & 200 & TRUE & 0.20 & 2.00 & 1.00 & 0.35 & 0.07 \\ 
   & $\textit{x} + \textit{z} + \textit{x} \times \textit{z} + \textit{z} \times \textit{z}$ & $\textit{x} \sim Normal , \textit{z} \sim Normal$ & 200 & FALSE & 0.20 & 2.00 & 3.00 & 0.31 & 0.08 \\ 
   & $\textit{x} + \textit{z} + \textit{x} \times \textit{z} + \textit{z} \times \textit{z}$ & $\textit{x} \sim Binary, \textit{z} \sim Binary$ & 200 & FALSE & 0.20 & 2.00 & 3.00 & 0.35 & 0.07 \\ 
   & $\textit{x} + \textit{z} + \textit{x} \times \textit{z} + \textit{z} \times \textit{z}$ & $\textit{x} \sim Normal, \textit{z} \sim Binary$ & 200 & FALSE & 0.20 & 2.00 & 3.00 & 0.31 & 0.08 \\ 
   & $\textit{x} + \textit{z} + \textit{x} \times \textit{z} + \textit{z} \times \textit{z}$ & $\textit{x} \sim Binary, \textit{z} \sim Normal$ & 200 & FALSE & 0.20 & 2.00 & 3.00 & 0.35 & 0.07 \\ 
   & $\textit{x} + \textit{z} + \textit{x} \times \textit{z} + \textit{z} \times \textit{z}$ & $\textit{x} \sim Normal , \textit{z} \sim Normal$ & 200 & FALSE & 0.20 & 2.00 & 3.00 & 0.40 & 0.10 \\ 
   & $\textit{x} + \textit{z} + \textit{x} \times \textit{z} + \textit{z} \times \textit{z}$ & $\textit{x} \sim Binary, \textit{z} \sim Binary$ & 200 & FALSE & 0.20 & 2.00 & 3.00 & 0.74 & 0.15 \\ 
   & $\textit{x} + \textit{z} + \textit{x} \times \textit{z} + \textit{z} \times \textit{z}$ & $\textit{x} \sim Normal, \textit{z} \sim Binary$ & 200 & FALSE & 0.20 & 2.00 & 3.00 & 0.95 & 0.26 \\ 
   & $\textit{x} + \textit{z} + \textit{x} \times \textit{z} + \textit{z} \times \textit{z}$ & $\textit{x} \sim Binary, \textit{z} \sim Normal$ & 200 & FALSE & 0.20 & 2.00 & 3.00 & 0.39 & 0.09 \\ 
   \hline
\hline
\label{tab:apptabFull}
\end{longtable}

\end{landscape}

\begin{landscape}
\scriptsize
% latex table generated in R 4.0.0 by xtable 1.8-4 package
% Wed Feb 10 10:37:14 2021
\begin{longtable}{lccccccccc}
\caption{False-positive probability (FPP) and false-positive ratio (FPR) for the different model sets when using Bonferroni correction} \\ 
  \hline
Restrictions & Set & Type & Sample Size & Outlier exclusion & Correlation & Covariates & Dependent variables & FPP & FPR \\ 
  \hline
$Without$ & $\textit{x} + \textit{z}$ & $\textit{x} \sim Normal , \textit{z} \sim Normal$ & 200 & FALSE & 0.20 & 2.00 & 1.00 & 0.07 & 0.05 \\ 
  $Without$ & $\textit{x} + \textit{z}$ & $\textit{x} \sim Normal , \textit{z} \sim Normal$ & 200 & FALSE & 0.20 & 3.00 & 1.00 & 0.08 & 0.05 \\ 
  $Without$ & $\textit{x} + \textit{z}$ & $\textit{x} \sim Binary, \textit{z} \sim Binary$ & 200 & FALSE & 0.20 & 2.00 & 1.00 & 0.07 & 0.05 \\ 
  $Without$ & $\textit{x} + \textit{z}$ & $\textit{x} \sim Binary, \textit{z} \sim Binary$ & 200 & FALSE & 0.20 & 3.00 & 1.00 & 0.08 & 0.05 \\ 
  $Without$ & $\textit{x} + \textit{z}$ & $\textit{x} \sim Normal, \textit{z} \sim Binary$ & 200 & FALSE & 0.20 & 2.00 & 1.00 & 0.06 & 0.05 \\ 
  $Without$ & $\textit{x} + \textit{z}$ & $\textit{x} \sim Normal, \textit{z} \sim Binary$ & 200 & FALSE & 0.20 & 3.00 & 1.00 & 0.08 & 0.05 \\ 
  $Without$ & $\textit{x} + \textit{z}$ & $\textit{x} \sim Binary, \textit{z} \sim Normal$ & 200 & FALSE & 0.20 & 2.00 & 1.00 & 0.07 & 0.05 \\ 
  $Without$ & $\textit{x} + \textit{z}$ & $\textit{x} \sim Binary, \textit{z} \sim Normal$ & 200 & FALSE & 0.20 & 3.00 & 1.00 & 0.08 & 0.05 \\ 
  $Without$ & $\textit{x} + \textit{z}$ & $\textit{x} \sim Normal , \textit{z} \sim Normal$ & 100 & FALSE & 0.20 & 2.00 & 1.00 & 0.07 & 0.05 \\ 
  $Without$ & $\textit{x} + \textit{z}$ & $\textit{x} \sim Binary, \textit{z} \sim Binary$ & 100 & FALSE & 0.20 & 2.00 & 1.00 & 0.07 & 0.05 \\ 
  $Without$ & $\textit{x} + \textit{z}$ & $\textit{x} \sim Normal, \textit{z} \sim Binary$ & 100 & FALSE & 0.20 & 2.00 & 1.00 & 0.07 & 0.05 \\ 
  $Without$ & $\textit{x} + \textit{z}$ & $\textit{x} \sim Binary, \textit{z} \sim Normal$ & 100 & FALSE & 0.20 & 2.00 & 1.00 & 0.07 & 0.05 \\ 
  $Without$ & $\textit{x} + \textit{z}$ & $\textit{x} \sim Normal , \textit{z} \sim Normal$ & 150 & FALSE & 0.20 & 2.00 & 1.00 & 0.07 & 0.05 \\ 
  $Without$ & $\textit{x} + \textit{z}$ & $\textit{x} \sim Binary, \textit{z} \sim Binary$ & 150 & FALSE & 0.20 & 2.00 & 1.00 & 0.07 & 0.05 \\ 
  $Without$ & $\textit{x} + \textit{z}$ & $\textit{x} \sim Normal, \textit{z} \sim Binary$ & 150 & FALSE & 0.20 & 2.00 & 1.00 & 0.07 & 0.05 \\ 
  $Without$ & $\textit{x} + \textit{z}$ & $\textit{x} \sim Binary, \textit{z} \sim Normal$ & 150 & FALSE & 0.20 & 2.00 & 1.00 & 0.07 & 0.05 \\ 
  $Without$ & $\textit{x} + \textit{z}$ & $\textit{x} \sim Normal , \textit{z} \sim Normal$ & 250 & FALSE & 0.20 & 2.00 & 1.00 & 0.07 & 0.05 \\ 
  $Without$ & $\textit{x} + \textit{z}$ & $\textit{x} \sim Binary, \textit{z} \sim Binary$ & 250 & FALSE & 0.20 & 2.00 & 1.00 & 0.06 & 0.05 \\ 
  $Without$ & $\textit{x} + \textit{z}$ & $\textit{x} \sim Normal, \textit{z} \sim Binary$ & 250 & FALSE & 0.20 & 2.00 & 1.00 & 0.07 & 0.05 \\ 
  $Without$ & $\textit{x} + \textit{z}$ & $\textit{x} \sim Binary, \textit{z} \sim Normal$ & 250 & FALSE & 0.20 & 2.00 & 1.00 & 0.07 & 0.05 \\ 
  $Without$ & $\textit{x} + \textit{z}$ & $\textit{x} \sim Normal , \textit{z} \sim Normal$ & 300 & FALSE & 0.20 & 2.00 & 1.00 & 0.07 & 0.05 \\ 
  $Without$ & $\textit{x} + \textit{z}$ & $\textit{x} \sim Binary, \textit{z} \sim Binary$ & 300 & FALSE & 0.20 & 2.00 & 1.00 & 0.07 & 0.05 \\ 
  $Without$ & $\textit{x} + \textit{z}$ & $\textit{x} \sim Normal, \textit{z} \sim Binary$ & 300 & FALSE & 0.20 & 2.00 & 1.00 & 0.06 & 0.05 \\ 
  $Without$ & $\textit{x} + \textit{z}$ & $\textit{x} \sim Binary, \textit{z} \sim Normal$ & 300 & FALSE & 0.20 & 2.00 & 1.00 & 0.07 & 0.05 \\ 
  $Without$ & $\textit{x} + \textit{z}$ & $\textit{x} \sim Normal , \textit{z} \sim Normal$ & 200 & FALSE & 0.30 & 2.00 & 1.00 & 0.08 & 0.05 \\ 
  $Without$ & $\textit{x} + \textit{z}$ & $\textit{x} \sim Binary, \textit{z} \sim Binary$ & 200 & FALSE & 0.30 & 2.00 & 1.00 & 0.07 & 0.05 \\ 
  $Without$ & $\textit{x} + \textit{z}$ & $\textit{x} \sim Normal, \textit{z} \sim Binary$ & 200 & FALSE & 0.30 & 2.00 & 1.00 & 0.08 & 0.05 \\ 
  $Without$ & $\textit{x} + \textit{z}$ & $\textit{x} \sim Binary, \textit{z} \sim Normal$ & 200 & FALSE & 0.30 & 2.00 & 1.00 & 0.08 & 0.05 \\ 
  $Without$ & $\textit{x} + \textit{z}$ & $\textit{x} \sim Normal , \textit{z} \sim Normal$ & 200 & FALSE & 0.40 & 2.00 & 1.00 & 0.08 & 0.05 \\ 
  $Without$ & $\textit{x} + \textit{z}$ & $\textit{x} \sim Binary, \textit{z} \sim Binary$ & 200 & FALSE & 0.40 & 2.00 & 1.00 & 0.08 & 0.05 \\ 
  $Without$ & $\textit{x} + \textit{z}$ & $\textit{x} \sim Normal, \textit{z} \sim Binary$ & 200 & FALSE & 0.40 & 2.00 & 1.00 & 0.09 & 0.05 \\ 
  $Without$ & $\textit{x} + \textit{z}$ & $\textit{x} \sim Binary, \textit{z} \sim Normal$ & 200 & FALSE & 0.40 & 2.00 & 1.00 & 0.09 & 0.05 \\ 
  $Without$ & $\textit{x} + \textit{z}$ & $\textit{x} \sim Normal , \textit{z} \sim Normal$ & 200 & TRUE & 0.20 & 2.00 & 1.00 & 0.13 & 0.05 \\ 
  $Without$ & $\textit{x} + \textit{z}$ & $\textit{x} \sim Binary, \textit{z} \sim Binary$ & 200 & TRUE & 0.20 & 2.00 & 1.00 & 0.11 & 0.05 \\ 
  $Without$ & $\textit{x} + \textit{z}$ & $\textit{x} \sim Normal, \textit{z} \sim Binary$ & 200 & TRUE & 0.20 & 2.00 & 1.00 & 0.12 & 0.05 \\ 
  $Without$ & $\textit{x} + \textit{z}$ & $\textit{x} \sim Binary, \textit{z} \sim Normal$ & 200 & TRUE & 0.20 & 2.00 & 1.00 & 0.12 & 0.05 \\ 
  $Without$ & $\textit{x} + \textit{z}$ & $\textit{x} \sim Normal , \textit{z} \sim Normal$ & 200 & FALSE & 0.20 & 2.00 & 3.00 & 0.13 & 0.05 \\ 
  $Without$ & $\textit{x} + \textit{z}$ & $\textit{x} \sim Binary, \textit{z} \sim Binary$ & 200 & FALSE & 0.20 & 2.00 & 3.00 & 0.13 & 0.05 \\ 
  $Without$ & $\textit{x} + \textit{z}$ & $\textit{x} \sim Normal, \textit{z} \sim Binary$ & 200 & FALSE & 0.20 & 2.00 & 3.00 & 0.13 & 0.05 \\ 
  $Without$ & $\textit{x} + \textit{z}$ & $\textit{x} \sim Binary, \textit{z} \sim Normal$ & 200 & FALSE & 0.20 & 2.00 & 3.00 & 0.13 & 0.05 \\ 
  $Without$ & $\textit{x} \times \textit{z}$ & $\textit{x} \sim Normal , \textit{z} \sim Normal$ & 200 & FALSE & 0.20 & 2.00 & 1.00 & 0.12 & 0.05 \\ 
  $Without$ & $\textit{x} \times \textit{z}$ & $\textit{x} \sim Normal , \textit{z} \sim Normal$ & 200 & FALSE & 0.20 & 3.00 & 1.00 & 0.17 & 0.05 \\ 
  $Without$ & $\textit{x} \times \textit{z}$ & $\textit{x} \sim Binary, \textit{z} \sim Binary$ & 200 & FALSE & 0.20 & 2.00 & 1.00 & 0.69 & 0.27 \\ 
  $Without$ & $\textit{x} \times \textit{z}$ & $\textit{x} \sim Binary, \textit{z} \sim Binary$ & 200 & FALSE & 0.20 & 3.00 & 1.00 & 0.85 & 0.30 \\ 
  $Without$ & $\textit{x} \times \textit{z}$ & $\textit{x} \sim Normal, \textit{z} \sim Binary$ & 200 & FALSE & 0.20 & 2.00 & 1.00 & 0.67 & 0.26 \\ 
  $Without$ & $\textit{x} \times \textit{z}$ & $\textit{x} \sim Normal, \textit{z} \sim Binary$ & 200 & FALSE & 0.20 & 3.00 & 1.00 & 0.81 & 0.28 \\ 
  $Without$ & $\textit{x} \times \textit{z}$ & $\textit{x} \sim Binary, \textit{z} \sim Normal$ & 200 & FALSE & 0.20 & 2.00 & 1.00 & 0.13 & 0.05 \\ 
  $Without$ & $\textit{x} \times \textit{z}$ & $\textit{x} \sim Binary, \textit{z} \sim Normal$ & 200 & FALSE & 0.20 & 3.00 & 1.00 & 0.18 & 0.05 \\ 
  $Without$ & $\textit{x} \times \textit{z}$ & $\textit{x} \sim Normal , \textit{z} \sim Normal$ & 100 & FALSE & 0.20 & 2.00 & 1.00 & 0.12 & 0.05 \\ 
  $Without$ & $\textit{x} \times \textit{z}$ & $\textit{x} \sim Binary, \textit{z} \sim Binary$ & 100 & FALSE & 0.20 & 2.00 & 1.00 & 0.43 & 0.16 \\ 
  $Without$ & $\textit{x} \times \textit{z}$ & $\textit{x} \sim Normal, \textit{z} \sim Binary$ & 100 & FALSE & 0.20 & 2.00 & 1.00 & 0.39 & 0.14 \\ 
  $Without$ & $\textit{x} \times \textit{z}$ & $\textit{x} \sim Binary, \textit{z} \sim Normal$ & 100 & FALSE & 0.20 & 2.00 & 1.00 & 0.13 & 0.05 \\ 
  $Without$ & $\textit{x} \times \textit{z}$ & $\textit{x} \sim Normal , \textit{z} \sim Normal$ & 150 & FALSE & 0.20 & 2.00 & 1.00 & 0.13 & 0.05 \\ 
  $Without$ & $\textit{x} \times \textit{z}$ & $\textit{x} \sim Binary, \textit{z} \sim Binary$ & 150 & FALSE & 0.20 & 2.00 & 1.00 & 0.58 & 0.22 \\ 
  $Without$ & $\textit{x} \times \textit{z}$ & $\textit{x} \sim Normal, \textit{z} \sim Binary$ & 150 & FALSE & 0.20 & 2.00 & 1.00 & 0.55 & 0.20 \\ 
  $Without$ & $\textit{x} \times \textit{z}$ & $\textit{x} \sim Binary, \textit{z} \sim Normal$ & 150 & FALSE & 0.20 & 2.00 & 1.00 & 0.13 & 0.05 \\ 
  $Without$ & $\textit{x} \times \textit{z}$ & $\textit{x} \sim Normal , \textit{z} \sim Normal$ & 250 & FALSE & 0.20 & 2.00 & 1.00 & 0.12 & 0.05 \\ 
  $Without$ & $\textit{x} \times \textit{z}$ & $\textit{x} \sim Binary, \textit{z} \sim Binary$ & 250 & FALSE & 0.20 & 2.00 & 1.00 & 0.79 & 0.32 \\ 
  $Without$ & $\textit{x} \times \textit{z}$ & $\textit{x} \sim Normal, \textit{z} \sim Binary$ & 250 & FALSE & 0.20 & 2.00 & 1.00 & 0.77 & 0.31 \\ 
  $Without$ & $\textit{x} \times \textit{z}$ & $\textit{x} \sim Binary, \textit{z} \sim Normal$ & 250 & FALSE & 0.20 & 2.00 & 1.00 & 0.13 & 0.05 \\ 
  $Without$ & $\textit{x} \times \textit{z}$ & $\textit{x} \sim Normal , \textit{z} \sim Normal$ & 300 & FALSE & 0.20 & 2.00 & 1.00 & 0.13 & 0.06 \\ 
  $Without$ & $\textit{x} \times \textit{z}$ & $\textit{x} \sim Binary, \textit{z} \sim Binary$ & 300 & FALSE & 0.20 & 2.00 & 1.00 & 0.86 & 0.36 \\ 
  $Without$ & $\textit{x} \times \textit{z}$ & $\textit{x} \sim Normal, \textit{z} \sim Binary$ & 300 & FALSE & 0.20 & 2.00 & 1.00 & 0.84 & 0.35 \\ 
  $Without$ & $\textit{x} \times \textit{z}$ & $\textit{x} \sim Binary, \textit{z} \sim Normal$ & 300 & FALSE & 0.20 & 2.00 & 1.00 & 0.13 & 0.05 \\ 
  $Without$ & $\textit{x} \times \textit{z}$ & $\textit{x} \sim Normal , \textit{z} \sim Normal$ & 200 & FALSE & 0.30 & 2.00 & 1.00 & 0.15 & 0.06 \\ 
  $Without$ & $\textit{x} \times \textit{z}$ & $\textit{x} \sim Binary, \textit{z} \sim Binary$ & 200 & FALSE & 0.30 & 2.00 & 1.00 & 0.97 & 0.45 \\ 
  $Without$ & $\textit{x} \times \textit{z}$ & $\textit{x} \sim Normal, \textit{z} \sim Binary$ & 200 & FALSE & 0.30 & 2.00 & 1.00 & 0.96 & 0.45 \\ 
  $Without$ & $\textit{x} \times \textit{z}$ & $\textit{x} \sim Binary, \textit{z} \sim Normal$ & 200 & FALSE & 0.30 & 2.00 & 1.00 & 0.14 & 0.05 \\ 
  $Without$ & $\textit{x} \times \textit{z}$ & $\textit{x} \sim Normal , \textit{z} \sim Normal$ & 200 & FALSE & 0.40 & 2.00 & 1.00 & 0.18 & 0.06 \\ 
  $Without$ & $\textit{x} \times \textit{z}$ & $\textit{x} \sim Binary, \textit{z} \sim Binary$ & 200 & FALSE & 0.40 & 2.00 & 1.00 & 1.00 & 0.52 \\ 
  $Without$ & $\textit{x} \times \textit{z}$ & $\textit{x} \sim Normal, \textit{z} \sim Binary$ & 200 & FALSE & 0.40 & 2.00 & 1.00 & 1.00 & 0.52 \\ 
  $Without$ & $\textit{x} \times \textit{z}$ & $\textit{x} \sim Binary, \textit{z} \sim Normal$ & 200 & FALSE & 0.40 & 2.00 & 1.00 & 0.15 & 0.05 \\ 
  $Without$ & $\textit{x} \times \textit{z}$ & $\textit{x} \sim Normal , \textit{z} \sim Normal$ & 200 & TRUE & 0.20 & 2.00 & 1.00 & 0.25 & 0.05 \\ 
  $Without$ & $\textit{x} \times \textit{z}$ & $\textit{x} \sim Binary, \textit{z} \sim Binary$ & 200 & TRUE & 0.20 & 2.00 & 1.00 & 0.77 & 0.26 \\ 
  $Without$ & $\textit{x} \times \textit{z}$ & $\textit{x} \sim Normal, \textit{z} \sim Binary$ & 200 & TRUE & 0.20 & 2.00 & 1.00 & 0.75 & 0.22 \\ 
  $Without$ & $\textit{x} \times \textit{z}$ & $\textit{x} \sim Binary, \textit{z} \sim Normal$ & 200 & TRUE & 0.20 & 2.00 & 1.00 & 0.24 & 0.05 \\ 
  $Without$ & $\textit{x} \times \textit{z}$ & $\textit{x} \sim Normal , \textit{z} \sim Normal$ & 200 & FALSE & 0.20 & 2.00 & 3.00 & 0.24 & 0.05 \\ 
  $Without$ & $\textit{x} \times \textit{z}$ & $\textit{x} \sim Binary, \textit{z} \sim Binary$ & 200 & FALSE & 0.20 & 2.00 & 3.00 & 0.88 & 0.29 \\ 
  $Without$ & $\textit{x} \times \textit{z}$ & $\textit{x} \sim Normal, \textit{z} \sim Binary$ & 200 & FALSE & 0.20 & 2.00 & 3.00 & 0.86 & 0.28 \\ 
  $Without$ & $\textit{x} \times \textit{z}$ & $\textit{x} \sim Binary, \textit{z} \sim Normal$ & 200 & FALSE & 0.20 & 2.00 & 3.00 & 0.24 & 0.05 \\ 
  $Without$ & $\textit{z} \times \textit{z}$ & $\textit{x} \sim Normal , \textit{z} \sim Normal$ & 200 & FALSE & 0.20 & 2.00 & 1.00 & 0.07 & 0.05 \\ 
  $Without$ & $\textit{z} \times \textit{z}$ & $\textit{x} \sim Normal , \textit{z} \sim Normal$ & 200 & FALSE & 0.20 & 3.00 & 1.00 & 0.08 & 0.05 \\ 
  $Without$ & $\textit{z} \times \textit{z}$ & $\textit{x} \sim Binary, \textit{z} \sim Binary$ & 200 & FALSE & 0.20 & 2.00 & 1.00 & 0.07 & 0.05 \\ 
  $Without$ & $\textit{z} \times \textit{z}$ & $\textit{x} \sim Binary, \textit{z} \sim Binary$ & 200 & FALSE & 0.20 & 3.00 & 1.00 & 0.09 & 0.05 \\ 
  $Without$ & $\textit{z} \times \textit{z}$ & $\textit{x} \sim Normal, \textit{z} \sim Binary$ & 200 & FALSE & 0.20 & 2.00 & 1.00 & 0.07 & 0.05 \\ 
  $Without$ & $\textit{z} \times \textit{z}$ & $\textit{x} \sim Normal, \textit{z} \sim Binary$ & 200 & FALSE & 0.20 & 3.00 & 1.00 & 0.09 & 0.05 \\ 
  $Without$ & $\textit{z} \times \textit{z}$ & $\textit{x} \sim Binary, \textit{z} \sim Normal$ & 200 & FALSE & 0.20 & 2.00 & 1.00 & 0.07 & 0.05 \\ 
  $Without$ & $\textit{z} \times \textit{z}$ & $\textit{x} \sim Binary, \textit{z} \sim Normal$ & 200 & FALSE & 0.20 & 3.00 & 1.00 & 0.09 & 0.05 \\ 
  $Without$ & $\textit{z} \times \textit{z}$ & $\textit{x} \sim Normal , \textit{z} \sim Normal$ & 100 & FALSE & 0.20 & 2.00 & 1.00 & 0.07 & 0.05 \\ 
  $Without$ & $\textit{z} \times \textit{z}$ & $\textit{x} \sim Binary, \textit{z} \sim Binary$ & 100 & FALSE & 0.20 & 2.00 & 1.00 & 0.07 & 0.05 \\ 
  $Without$ & $\textit{z} \times \textit{z}$ & $\textit{x} \sim Normal, \textit{z} \sim Binary$ & 100 & FALSE & 0.20 & 2.00 & 1.00 & 0.07 & 0.05 \\ 
  $Without$ & $\textit{z} \times \textit{z}$ & $\textit{x} \sim Binary, \textit{z} \sim Normal$ & 100 & FALSE & 0.20 & 2.00 & 1.00 & 0.07 & 0.05 \\ 
  $Without$ & $\textit{z} \times \textit{z}$ & $\textit{x} \sim Normal , \textit{z} \sim Normal$ & 150 & FALSE & 0.20 & 2.00 & 1.00 & 0.07 & 0.05 \\ 
  $Without$ & $\textit{z} \times \textit{z}$ & $\textit{x} \sim Binary, \textit{z} \sim Binary$ & 150 & FALSE & 0.20 & 2.00 & 1.00 & 0.07 & 0.05 \\ 
  $Without$ & $\textit{z} \times \textit{z}$ & $\textit{x} \sim Normal, \textit{z} \sim Binary$ & 150 & FALSE & 0.20 & 2.00 & 1.00 & 0.07 & 0.05 \\ 
  $Without$ & $\textit{z} \times \textit{z}$ & $\textit{x} \sim Binary, \textit{z} \sim Normal$ & 150 & FALSE & 0.20 & 2.00 & 1.00 & 0.07 & 0.05 \\ 
  $Without$ & $\textit{z} \times \textit{z}$ & $\textit{x} \sim Normal , \textit{z} \sim Normal$ & 250 & FALSE & 0.20 & 2.00 & 1.00 & 0.08 & 0.05 \\ 
  $Without$ & $\textit{z} \times \textit{z}$ & $\textit{x} \sim Binary, \textit{z} \sim Binary$ & 250 & FALSE & 0.20 & 2.00 & 1.00 & 0.07 & 0.05 \\ 
  $Without$ & $\textit{z} \times \textit{z}$ & $\textit{x} \sim Normal, \textit{z} \sim Binary$ & 250 & FALSE & 0.20 & 2.00 & 1.00 & 0.07 & 0.05 \\ 
  $Without$ & $\textit{z} \times \textit{z}$ & $\textit{x} \sim Binary, \textit{z} \sim Normal$ & 250 & FALSE & 0.20 & 2.00 & 1.00 & 0.07 & 0.05 \\ 
  $Without$ & $\textit{z} \times \textit{z}$ & $\textit{x} \sim Normal , \textit{z} \sim Normal$ & 300 & FALSE & 0.20 & 2.00 & 1.00 & 0.07 & 0.05 \\ 
  $Without$ & $\textit{z} \times \textit{z}$ & $\textit{x} \sim Binary, \textit{z} \sim Binary$ & 300 & FALSE & 0.20 & 2.00 & 1.00 & 0.07 & 0.05 \\ 
  $Without$ & $\textit{z} \times \textit{z}$ & $\textit{x} \sim Normal, \textit{z} \sim Binary$ & 300 & FALSE & 0.20 & 2.00 & 1.00 & 0.07 & 0.05 \\ 
  $Without$ & $\textit{z} \times \textit{z}$ & $\textit{x} \sim Binary, \textit{z} \sim Normal$ & 300 & FALSE & 0.20 & 2.00 & 1.00 & 0.07 & 0.05 \\ 
  $Without$ & $\textit{z} \times \textit{z}$ & $\textit{x} \sim Normal , \textit{z} \sim Normal$ & 200 & FALSE & 0.30 & 2.00 & 1.00 & 0.08 & 0.05 \\ 
  $Without$ & $\textit{z} \times \textit{z}$ & $\textit{x} \sim Binary, \textit{z} \sim Binary$ & 200 & FALSE & 0.30 & 2.00 & 1.00 & 0.08 & 0.05 \\ 
  $Without$ & $\textit{z} \times \textit{z}$ & $\textit{x} \sim Normal, \textit{z} \sim Binary$ & 200 & FALSE & 0.30 & 2.00 & 1.00 & 0.08 & 0.05 \\ 
  $Without$ & $\textit{z} \times \textit{z}$ & $\textit{x} \sim Binary, \textit{z} \sim Normal$ & 200 & FALSE & 0.30 & 2.00 & 1.00 & 0.08 & 0.05 \\ 
  $Without$ & $\textit{z} \times \textit{z}$ & $\textit{x} \sim Normal , \textit{z} \sim Normal$ & 200 & FALSE & 0.40 & 2.00 & 1.00 & 0.10 & 0.05 \\ 
  $Without$ & $\textit{z} \times \textit{z}$ & $\textit{x} \sim Binary, \textit{z} \sim Binary$ & 200 & FALSE & 0.40 & 2.00 & 1.00 & 0.10 & 0.05 \\ 
  $Without$ & $\textit{z} \times \textit{z}$ & $\textit{x} \sim Normal, \textit{z} \sim Binary$ & 200 & FALSE & 0.40 & 2.00 & 1.00 & 0.10 & 0.05 \\ 
  $Without$ & $\textit{z} \times \textit{z}$ & $\textit{x} \sim Binary, \textit{z} \sim Normal$ & 200 & FALSE & 0.40 & 2.00 & 1.00 & 0.09 & 0.05 \\ 
  $Without$ & $\textit{z} \times \textit{z}$ & $\textit{x} \sim Normal , \textit{z} \sim Normal$ & 200 & TRUE & 0.20 & 2.00 & 1.00 & 0.14 & 0.05 \\ 
  $Without$ & $\textit{z} \times \textit{z}$ & $\textit{x} \sim Binary, \textit{z} \sim Binary$ & 200 & TRUE & 0.20 & 2.00 & 1.00 & 0.11 & 0.05 \\ 
  $Without$ & $\textit{z} \times \textit{z}$ & $\textit{x} \sim Normal, \textit{z} \sim Binary$ & 200 & TRUE & 0.20 & 2.00 & 1.00 & 0.12 & 0.05 \\ 
  $Without$ & $\textit{z} \times \textit{z}$ & $\textit{x} \sim Binary, \textit{z} \sim Normal$ & 200 & TRUE & 0.20 & 2.00 & 1.00 & 0.14 & 0.05 \\ 
  $Without$ & $\textit{z} \times \textit{z}$ & $\textit{x} \sim Normal , \textit{z} \sim Normal$ & 200 & FALSE & 0.20 & 2.00 & 3.00 & 0.14 & 0.05 \\ 
  $Without$ & $\textit{z} \times \textit{z}$ & $\textit{x} \sim Binary, \textit{z} \sim Binary$ & 200 & FALSE & 0.20 & 2.00 & 3.00 & 0.14 & 0.05 \\ 
  $Without$ & $\textit{z} \times \textit{z}$ & $\textit{x} \sim Normal, \textit{z} \sim Binary$ & 200 & FALSE & 0.20 & 2.00 & 3.00 & 0.14 & 0.05 \\ 
  $Without$ & $\textit{z} \times \textit{z}$ & $\textit{x} \sim Binary, \textit{z} \sim Normal$ & 200 & FALSE & 0.20 & 2.00 & 3.00 & 0.13 & 0.05 \\ 
  $Without$ & $\textit{x} + \textit{z} + \textit{x} \times \textit{z}$ & $\textit{x} \sim Normal , \textit{z} \sim Normal$ & 200 & FALSE & 0.20 & 2.00 & 1.00 & 0.15 & 0.05 \\ 
  $Without$ & $\textit{x} + \textit{z} + \textit{x} \times \textit{z}$ & $\textit{x} \sim Normal , \textit{z} \sim Normal$ & 200 & FALSE & 0.20 & 3.00 & 1.00 & 0.23 & 0.05 \\ 
  $Without$ & $\textit{x} + \textit{z} + \textit{x} \times \textit{z}$ & $\textit{x} \sim Binary, \textit{z} \sim Binary$ & 200 & FALSE & 0.20 & 2.00 & 1.00 & 0.75 & 0.24 \\ 
  $Without$ & $\textit{x} + \textit{z} + \textit{x} \times \textit{z}$ & $\textit{x} \sim Binary, \textit{z} \sim Binary$ & 200 & FALSE & 0.20 & 3.00 & 1.00 & 0.90 & 0.30 \\ 
  $Without$ & $\textit{x} + \textit{z} + \textit{x} \times \textit{z}$ & $\textit{x} \sim Normal, \textit{z} \sim Binary$ & 200 & FALSE & 0.20 & 2.00 & 1.00 & 0.71 & 0.23 \\ 
  $Without$ & $\textit{x} + \textit{z} + \textit{x} \times \textit{z}$ & $\textit{x} \sim Normal, \textit{z} \sim Binary$ & 200 & FALSE & 0.20 & 3.00 & 1.00 & 0.87 & 0.29 \\ 
  $Without$ & $\textit{x} + \textit{z} + \textit{x} \times \textit{z}$ & $\textit{x} \sim Binary, \textit{z} \sim Normal$ & 200 & FALSE & 0.20 & 2.00 & 1.00 & 0.15 & 0.04 \\ 
  $Without$ & $\textit{x} + \textit{z} + \textit{x} \times \textit{z}$ & $\textit{x} \sim Binary, \textit{z} \sim Normal$ & 200 & FALSE & 0.20 & 3.00 & 1.00 & 0.24 & 0.05 \\ 
  $Without$ & $\textit{x} + \textit{z} + \textit{x} \times \textit{z}$ & $\textit{x} \sim Normal , \textit{z} \sim Normal$ & 100 & FALSE & 0.20 & 2.00 & 1.00 & 0.15 & 0.05 \\ 
  $Without$ & $\textit{x} + \textit{z} + \textit{x} \times \textit{z}$ & $\textit{x} \sim Binary, \textit{z} \sim Binary$ & 100 & FALSE & 0.20 & 2.00 & 1.00 & 0.49 & 0.13 \\ 
  $Without$ & $\textit{x} + \textit{z} + \textit{x} \times \textit{z}$ & $\textit{x} \sim Normal, \textit{z} \sim Binary$ & 100 & FALSE & 0.20 & 2.00 & 1.00 & 0.45 & 0.13 \\ 
  $Without$ & $\textit{x} + \textit{z} + \textit{x} \times \textit{z}$ & $\textit{x} \sim Binary, \textit{z} \sim Normal$ & 100 & FALSE & 0.20 & 2.00 & 1.00 & 0.16 & 0.05 \\ 
  $Without$ & $\textit{x} + \textit{z} + \textit{x} \times \textit{z}$ & $\textit{x} \sim Normal , \textit{z} \sim Normal$ & 150 & FALSE & 0.20 & 2.00 & 1.00 & 0.15 & 0.05 \\ 
  $Without$ & $\textit{x} + \textit{z} + \textit{x} \times \textit{z}$ & $\textit{x} \sim Binary, \textit{z} \sim Binary$ & 150 & FALSE & 0.20 & 2.00 & 1.00 & 0.63 & 0.18 \\ 
  $Without$ & $\textit{x} + \textit{z} + \textit{x} \times \textit{z}$ & $\textit{x} \sim Normal, \textit{z} \sim Binary$ & 150 & FALSE & 0.20 & 2.00 & 1.00 & 0.60 & 0.18 \\ 
  $Without$ & $\textit{x} + \textit{z} + \textit{x} \times \textit{z}$ & $\textit{x} \sim Binary, \textit{z} \sim Normal$ & 150 & FALSE & 0.20 & 2.00 & 1.00 & 0.15 & 0.04 \\ 
  $Without$ & $\textit{x} + \textit{z} + \textit{x} \times \textit{z}$ & $\textit{x} \sim Normal , \textit{z} \sim Normal$ & 250 & FALSE & 0.20 & 2.00 & 1.00 & 0.15 & 0.05 \\ 
  $Without$ & $\textit{x} + \textit{z} + \textit{x} \times \textit{z}$ & $\textit{x} \sim Binary, \textit{z} \sim Binary$ & 250 & FALSE & 0.20 & 2.00 & 1.00 & 0.83 & 0.28 \\ 
  $Without$ & $\textit{x} + \textit{z} + \textit{x} \times \textit{z}$ & $\textit{x} \sim Normal, \textit{z} \sim Binary$ & 250 & FALSE & 0.20 & 2.00 & 1.00 & 0.81 & 0.27 \\ 
  $Without$ & $\textit{x} + \textit{z} + \textit{x} \times \textit{z}$ & $\textit{x} \sim Binary, \textit{z} \sim Normal$ & 250 & FALSE & 0.20 & 2.00 & 1.00 & 0.15 & 0.04 \\ 
  $Without$ & $\textit{x} + \textit{z} + \textit{x} \times \textit{z}$ & $\textit{x} \sim Normal , \textit{z} \sim Normal$ & 300 & FALSE & 0.20 & 2.00 & 1.00 & 0.15 & 0.05 \\ 
  $Without$ & $\textit{x} + \textit{z} + \textit{x} \times \textit{z}$ & $\textit{x} \sim Binary, \textit{z} \sim Binary$ & 300 & FALSE & 0.20 & 2.00 & 1.00 & 0.89 & 0.32 \\ 
  $Without$ & $\textit{x} + \textit{z} + \textit{x} \times \textit{z}$ & $\textit{x} \sim Normal, \textit{z} \sim Binary$ & 300 & FALSE & 0.20 & 2.00 & 1.00 & 0.87 & 0.31 \\ 
  $Without$ & $\textit{x} + \textit{z} + \textit{x} \times \textit{z}$ & $\textit{x} \sim Binary, \textit{z} \sim Normal$ & 300 & FALSE & 0.20 & 2.00 & 1.00 & 0.15 & 0.04 \\ 
  $Without$ & $\textit{x} + \textit{z} + \textit{x} \times \textit{z}$ & $\textit{x} \sim Normal , \textit{z} \sim Normal$ & 200 & FALSE & 0.30 & 2.00 & 1.00 & 0.19 & 0.06 \\ 
  $Without$ & $\textit{x} + \textit{z} + \textit{x} \times \textit{z}$ & $\textit{x} \sim Binary, \textit{z} \sim Binary$ & 200 & FALSE & 0.30 & 2.00 & 1.00 & 0.98 & 0.41 \\ 
  $Without$ & $\textit{x} + \textit{z} + \textit{x} \times \textit{z}$ & $\textit{x} \sim Normal, \textit{z} \sim Binary$ & 200 & FALSE & 0.30 & 2.00 & 1.00 & 0.98 & 0.41 \\ 
  $Without$ & $\textit{x} + \textit{z} + \textit{x} \times \textit{z}$ & $\textit{x} \sim Binary, \textit{z} \sim Normal$ & 200 & FALSE & 0.30 & 2.00 & 1.00 & 0.18 & 0.04 \\ 
  $Without$ & $\textit{x} + \textit{z} + \textit{x} \times \textit{z}$ & $\textit{x} \sim Normal , \textit{z} \sim Normal$ & 200 & FALSE & 0.40 & 2.00 & 1.00 & 0.23 & 0.06 \\ 
  $Without$ & $\textit{x} + \textit{z} + \textit{x} \times \textit{z}$ & $\textit{x} \sim Binary, \textit{z} \sim Binary$ & 200 & FALSE & 0.40 & 2.00 & 1.00 & 1.00 & 0.49 \\ 
  $Without$ & $\textit{x} + \textit{z} + \textit{x} \times \textit{z}$ & $\textit{x} \sim Normal, \textit{z} \sim Binary$ & 200 & FALSE & 0.40 & 2.00 & 1.00 & 1.00 & 0.49 \\ 
  $Without$ & $\textit{x} + \textit{z} + \textit{x} \times \textit{z}$ & $\textit{x} \sim Binary, \textit{z} \sim Normal$ & 200 & FALSE & 0.40 & 2.00 & 1.00 & 0.20 & 0.04 \\ 
  $Without$ & $\textit{x} + \textit{z} + \textit{x} \times \textit{z}$ & $\textit{x} \sim Normal , \textit{z} \sim Normal$ & 200 & TRUE & 0.20 & 2.00 & 1.00 & 0.29 & 0.05 \\ 
  $Without$ & $\textit{x} + \textit{z} + \textit{x} \times \textit{z}$ & $\textit{x} \sim Binary, \textit{z} \sim Binary$ & 200 & TRUE & 0.20 & 2.00 & 1.00 & 0.82 & 0.22 \\ 
  $Without$ & $\textit{x} + \textit{z} + \textit{x} \times \textit{z}$ & $\textit{x} \sim Normal, \textit{z} \sim Binary$ & 200 & TRUE & 0.20 & 2.00 & 1.00 & 0.81 & 0.20 \\ 
  $Without$ & $\textit{x} + \textit{z} + \textit{x} \times \textit{z}$ & $\textit{x} \sim Binary, \textit{z} \sim Normal$ & 200 & TRUE & 0.20 & 2.00 & 1.00 & 0.28 & 0.04 \\ 
  $Without$ & $\textit{x} + \textit{z} + \textit{x} \times \textit{z}$ & $\textit{x} \sim Normal , \textit{z} \sim Normal$ & 200 & FALSE & 0.20 & 2.00 & 3.00 & 0.28 & 0.05 \\ 
  $Without$ & $\textit{x} + \textit{z} + \textit{x} \times \textit{z}$ & $\textit{x} \sim Binary, \textit{z} \sim Binary$ & 200 & FALSE & 0.20 & 2.00 & 3.00 & 0.92 & 0.26 \\ 
  $Without$ & $\textit{x} + \textit{z} + \textit{x} \times \textit{z}$ & $\textit{x} \sim Normal, \textit{z} \sim Binary$ & 200 & FALSE & 0.20 & 2.00 & 3.00 & 0.90 & 0.25 \\ 
  $Without$ & $\textit{x} + \textit{z} + \textit{x} \times \textit{z}$ & $\textit{x} \sim Binary, \textit{z} \sim Normal$ & 200 & FALSE & 0.20 & 2.00 & 3.00 & 0.28 & 0.04 \\ 
  $Without$ & $\textit{x} + \textit{z} + \textit{z} \times \textit{z}$ & $\textit{x} \sim Normal , \textit{z} \sim Normal$ & 200 & FALSE & 0.20 & 2.00 & 1.00 & 0.08 & 0.05 \\ 
  $Without$ & $\textit{x} + \textit{z} + \textit{z} \times \textit{z}$ & $\textit{x} \sim Normal , \textit{z} \sim Normal$ & 200 & FALSE & 0.20 & 3.00 & 1.00 & 0.10 & 0.05 \\ 
  $Without$ & $\textit{x} + \textit{z} + \textit{z} \times \textit{z}$ & $\textit{x} \sim Binary, \textit{z} \sim Binary$ & 200 & FALSE & 0.20 & 2.00 & 1.00 & 0.07 & 0.05 \\ 
  $Without$ & $\textit{x} + \textit{z} + \textit{z} \times \textit{z}$ & $\textit{x} \sim Binary, \textit{z} \sim Binary$ & 200 & FALSE & 0.20 & 3.00 & 1.00 & 0.10 & 0.05 \\ 
  $Without$ & $\textit{x} + \textit{z} + \textit{z} \times \textit{z}$ & $\textit{x} \sim Normal, \textit{z} \sim Binary$ & 200 & FALSE & 0.20 & 2.00 & 1.00 & 0.08 & 0.05 \\ 
  $Without$ & $\textit{x} + \textit{z} + \textit{z} \times \textit{z}$ & $\textit{x} \sim Normal, \textit{z} \sim Binary$ & 200 & FALSE & 0.20 & 3.00 & 1.00 & 0.09 & 0.05 \\ 
  $Without$ & $\textit{x} + \textit{z} + \textit{z} \times \textit{z}$ & $\textit{x} \sim Binary, \textit{z} \sim Normal$ & 200 & FALSE & 0.20 & 2.00 & 1.00 & 0.07 & 0.05 \\ 
  $Without$ & $\textit{x} + \textit{z} + \textit{z} \times \textit{z}$ & $\textit{x} \sim Binary, \textit{z} \sim Normal$ & 200 & FALSE & 0.20 & 3.00 & 1.00 & 0.10 & 0.05 \\ 
  $Without$ & $\textit{x} + \textit{z} + \textit{z} \times \textit{z}$ & $\textit{x} \sim Normal , \textit{z} \sim Normal$ & 100 & FALSE & 0.20 & 2.00 & 1.00 & 0.07 & 0.05 \\ 
  $Without$ & $\textit{x} + \textit{z} + \textit{z} \times \textit{z}$ & $\textit{x} \sim Binary, \textit{z} \sim Binary$ & 100 & FALSE & 0.20 & 2.00 & 1.00 & 0.07 & 0.05 \\ 
  $Without$ & $\textit{x} + \textit{z} + \textit{z} \times \textit{z}$ & $\textit{x} \sim Normal, \textit{z} \sim Binary$ & 100 & FALSE & 0.20 & 2.00 & 1.00 & 0.07 & 0.05 \\ 
  $Without$ & $\textit{x} + \textit{z} + \textit{z} \times \textit{z}$ & $\textit{x} \sim Binary, \textit{z} \sim Normal$ & 100 & FALSE & 0.20 & 2.00 & 1.00 & 0.07 & 0.05 \\ 
  $Without$ & $\textit{x} + \textit{z} + \textit{z} \times \textit{z}$ & $\textit{x} \sim Normal , \textit{z} \sim Normal$ & 150 & FALSE & 0.20 & 2.00 & 1.00 & 0.07 & 0.05 \\ 
  $Without$ & $\textit{x} + \textit{z} + \textit{z} \times \textit{z}$ & $\textit{x} \sim Binary, \textit{z} \sim Binary$ & 150 & FALSE & 0.20 & 2.00 & 1.00 & 0.07 & 0.05 \\ 
  $Without$ & $\textit{x} + \textit{z} + \textit{z} \times \textit{z}$ & $\textit{x} \sim Normal, \textit{z} \sim Binary$ & 150 & FALSE & 0.20 & 2.00 & 1.00 & 0.07 & 0.05 \\ 
  $Without$ & $\textit{x} + \textit{z} + \textit{z} \times \textit{z}$ & $\textit{x} \sim Binary, \textit{z} \sim Normal$ & 150 & FALSE & 0.20 & 2.00 & 1.00 & 0.07 & 0.05 \\ 
  $Without$ & $\textit{x} + \textit{z} + \textit{z} \times \textit{z}$ & $\textit{x} \sim Normal , \textit{z} \sim Normal$ & 250 & FALSE & 0.20 & 2.00 & 1.00 & 0.07 & 0.05 \\ 
  $Without$ & $\textit{x} + \textit{z} + \textit{z} \times \textit{z}$ & $\textit{x} \sim Binary, \textit{z} \sim Binary$ & 250 & FALSE & 0.20 & 2.00 & 1.00 & 0.07 & 0.05 \\ 
  $Without$ & $\textit{x} + \textit{z} + \textit{z} \times \textit{z}$ & $\textit{x} \sim Normal, \textit{z} \sim Binary$ & 250 & FALSE & 0.20 & 2.00 & 1.00 & 0.07 & 0.05 \\ 
  $Without$ & $\textit{x} + \textit{z} + \textit{z} \times \textit{z}$ & $\textit{x} \sim Binary, \textit{z} \sim Normal$ & 250 & FALSE & 0.20 & 2.00 & 1.00 & 0.07 & 0.05 \\ 
  $Without$ & $\textit{x} + \textit{z} + \textit{z} \times \textit{z}$ & $\textit{x} \sim Normal , \textit{z} \sim Normal$ & 300 & FALSE & 0.20 & 2.00 & 1.00 & 0.07 & 0.05 \\ 
  $Without$ & $\textit{x} + \textit{z} + \textit{z} \times \textit{z}$ & $\textit{x} \sim Binary, \textit{z} \sim Binary$ & 300 & FALSE & 0.20 & 2.00 & 1.00 & 0.07 & 0.05 \\ 
  $Without$ & $\textit{x} + \textit{z} + \textit{z} \times \textit{z}$ & $\textit{x} \sim Normal, \textit{z} \sim Binary$ & 300 & FALSE & 0.20 & 2.00 & 1.00 & 0.07 & 0.05 \\ 
  $Without$ & $\textit{x} + \textit{z} + \textit{z} \times \textit{z}$ & $\textit{x} \sim Binary, \textit{z} \sim Normal$ & 300 & FALSE & 0.20 & 2.00 & 1.00 & 0.07 & 0.05 \\ 
  $Without$ & $\textit{x} + \textit{z} + \textit{z} \times \textit{z}$ & $\textit{x} \sim Normal , \textit{z} \sim Normal$ & 200 & FALSE & 0.30 & 2.00 & 1.00 & 0.09 & 0.05 \\ 
  $Without$ & $\textit{x} + \textit{z} + \textit{z} \times \textit{z}$ & $\textit{x} \sim Binary, \textit{z} \sim Binary$ & 200 & FALSE & 0.30 & 2.00 & 1.00 & 0.08 & 0.05 \\ 
  $Without$ & $\textit{x} + \textit{z} + \textit{z} \times \textit{z}$ & $\textit{x} \sim Normal, \textit{z} \sim Binary$ & 200 & FALSE & 0.30 & 2.00 & 1.00 & 0.09 & 0.05 \\ 
  $Without$ & $\textit{x} + \textit{z} + \textit{z} \times \textit{z}$ & $\textit{x} \sim Binary, \textit{z} \sim Normal$ & 200 & FALSE & 0.30 & 2.00 & 1.00 & 0.08 & 0.05 \\ 
  $Without$ & $\textit{x} + \textit{z} + \textit{z} \times \textit{z}$ & $\textit{x} \sim Normal , \textit{z} \sim Normal$ & 200 & FALSE & 0.40 & 2.00 & 1.00 & 0.10 & 0.05 \\ 
  $Without$ & $\textit{x} + \textit{z} + \textit{z} \times \textit{z}$ & $\textit{x} \sim Binary, \textit{z} \sim Binary$ & 200 & FALSE & 0.40 & 2.00 & 1.00 & 0.10 & 0.05 \\ 
  $Without$ & $\textit{x} + \textit{z} + \textit{z} \times \textit{z}$ & $\textit{x} \sim Normal, \textit{z} \sim Binary$ & 200 & FALSE & 0.40 & 2.00 & 1.00 & 0.10 & 0.05 \\ 
  $Without$ & $\textit{x} + \textit{z} + \textit{z} \times \textit{z}$ & $\textit{x} \sim Binary, \textit{z} \sim Normal$ & 200 & FALSE & 0.40 & 2.00 & 1.00 & 0.10 & 0.05 \\ 
  $Without$ & $\textit{x} + \textit{z} + \textit{z} \times \textit{z}$ & $\textit{x} \sim Normal , \textit{z} \sim Normal$ & 200 & TRUE & 0.20 & 2.00 & 1.00 & 0.14 & 0.05 \\ 
  $Without$ & $\textit{x} + \textit{z} + \textit{z} \times \textit{z}$ & $\textit{x} \sim Binary, \textit{z} \sim Binary$ & 200 & TRUE & 0.20 & 2.00 & 1.00 & 0.11 & 0.05 \\ 
  $Without$ & $\textit{x} + \textit{z} + \textit{z} \times \textit{z}$ & $\textit{x} \sim Normal, \textit{z} \sim Binary$ & 200 & TRUE & 0.20 & 2.00 & 1.00 & 0.12 & 0.05 \\ 
  $Without$ & $\textit{x} + \textit{z} + \textit{z} \times \textit{z}$ & $\textit{x} \sim Binary, \textit{z} \sim Normal$ & 200 & TRUE & 0.20 & 2.00 & 1.00 & 0.13 & 0.05 \\ 
  $Without$ & $\textit{x} + \textit{z} + \textit{z} \times \textit{z}$ & $\textit{x} \sim Normal , \textit{z} \sim Normal$ & 200 & FALSE & 0.20 & 2.00 & 3.00 & 0.14 & 0.05 \\ 
  $Without$ & $\textit{x} + \textit{z} + \textit{z} \times \textit{z}$ & $\textit{x} \sim Binary, \textit{z} \sim Binary$ & 200 & FALSE & 0.20 & 2.00 & 3.00 & 0.14 & 0.05 \\ 
  $Without$ & $\textit{x} + \textit{z} + \textit{z} \times \textit{z}$ & $\textit{x} \sim Normal, \textit{z} \sim Binary$ & 200 & FALSE & 0.20 & 2.00 & 3.00 & 0.15 & 0.05 \\ 
  $Without$ & $\textit{x} + \textit{z} + \textit{z} \times \textit{z}$ & $\textit{x} \sim Binary, \textit{z} \sim Normal$ & 200 & FALSE & 0.20 & 2.00 & 3.00 & 0.14 & 0.05 \\ 
  $Without$ & $\textit{x} \times \textit{z} + \textit{z} \times \textit{z}$ & $\textit{x} \sim Normal , \textit{z} \sim Normal$ & 200 & FALSE & 0.20 & 2.00 & 1.00 & 0.12 & 0.05 \\ 
  $Without$ & $\textit{x} \times \textit{z} + \textit{z} \times \textit{z}$ & $\textit{x} \sim Normal , \textit{z} \sim Normal$ & 200 & FALSE & 0.20 & 3.00 & 1.00 & 0.19 & 0.06 \\ 
  $Without$ & $\textit{x} \times \textit{z} + \textit{z} \times \textit{z}$ & $\textit{x} \sim Binary, \textit{z} \sim Binary$ & 200 & FALSE & 0.20 & 2.00 & 1.00 & 0.24 & 0.08 \\ 
  $Without$ & $\textit{x} \times \textit{z} + \textit{z} \times \textit{z}$ & $\textit{x} \sim Binary, \textit{z} \sim Binary$ & 200 & FALSE & 0.20 & 3.00 & 1.00 & 0.84 & 0.19 \\ 
  $Without$ & $\textit{x} \times \textit{z} + \textit{z} \times \textit{z}$ & $\textit{x} \sim Normal, \textit{z} \sim Binary$ & 200 & FALSE & 0.20 & 2.00 & 1.00 & 0.66 & 0.25 \\ 
  $Without$ & $\textit{x} \times \textit{z} + \textit{z} \times \textit{z}$ & $\textit{x} \sim Normal, \textit{z} \sim Binary$ & 200 & FALSE & 0.20 & 3.00 & 1.00 & 0.84 & 0.45 \\ 
  $Without$ & $\textit{x} \times \textit{z} + \textit{z} \times \textit{z}$ & $\textit{x} \sim Binary, \textit{z} \sim Normal$ & 200 & FALSE & 0.20 & 2.00 & 1.00 & 0.13 & 0.05 \\ 
  $Without$ & $\textit{x} \times \textit{z} + \textit{z} \times \textit{z}$ & $\textit{x} \sim Binary, \textit{z} \sim Normal$ & 200 & FALSE & 0.20 & 3.00 & 1.00 & 0.21 & 0.04 \\ 
  $Without$ & $\textit{x} \times \textit{z} + \textit{z} \times \textit{z}$ & $\textit{x} \sim Normal , \textit{z} \sim Normal$ & 100 & FALSE & 0.20 & 2.00 & 1.00 & 0.13 & 0.05 \\ 
  $Without$ & $\textit{x} \times \textit{z} + \textit{z} \times \textit{z}$ & $\textit{x} \sim Binary, \textit{z} \sim Binary$ & 100 & FALSE & 0.20 & 2.00 & 1.00 & 0.18 & 0.06 \\ 
  $Without$ & $\textit{x} \times \textit{z} + \textit{z} \times \textit{z}$ & $\textit{x} \sim Normal, \textit{z} \sim Binary$ & 100 & FALSE & 0.20 & 2.00 & 1.00 & 0.39 & 0.14 \\ 
  $Without$ & $\textit{x} \times \textit{z} + \textit{z} \times \textit{z}$ & $\textit{x} \sim Binary, \textit{z} \sim Normal$ & 100 & FALSE & 0.20 & 2.00 & 1.00 & 0.13 & 0.05 \\ 
  $Without$ & $\textit{x} \times \textit{z} + \textit{z} \times \textit{z}$ & $\textit{x} \sim Normal , \textit{z} \sim Normal$ & 150 & FALSE & 0.20 & 2.00 & 1.00 & 0.12 & 0.05 \\ 
  $Without$ & $\textit{x} \times \textit{z} + \textit{z} \times \textit{z}$ & $\textit{x} \sim Binary, \textit{z} \sim Binary$ & 150 & FALSE & 0.20 & 2.00 & 1.00 & 0.20 & 0.07 \\ 
  $Without$ & $\textit{x} \times \textit{z} + \textit{z} \times \textit{z}$ & $\textit{x} \sim Normal, \textit{z} \sim Binary$ & 150 & FALSE & 0.20 & 2.00 & 1.00 & 0.54 & 0.20 \\ 
  $Without$ & $\textit{x} \times \textit{z} + \textit{z} \times \textit{z}$ & $\textit{x} \sim Binary, \textit{z} \sim Normal$ & 150 & FALSE & 0.20 & 2.00 & 1.00 & 0.13 & 0.05 \\ 
  $Without$ & $\textit{x} \times \textit{z} + \textit{z} \times \textit{z}$ & $\textit{x} \sim Normal , \textit{z} \sim Normal$ & 250 & FALSE & 0.20 & 2.00 & 1.00 & 0.12 & 0.05 \\ 
  $Without$ & $\textit{x} \times \textit{z} + \textit{z} \times \textit{z}$ & $\textit{x} \sim Binary, \textit{z} \sim Binary$ & 250 & FALSE & 0.20 & 2.00 & 1.00 & 0.26 & 0.09 \\ 
  $Without$ & $\textit{x} \times \textit{z} + \textit{z} \times \textit{z}$ & $\textit{x} \sim Normal, \textit{z} \sim Binary$ & 250 & FALSE & 0.20 & 2.00 & 1.00 & 0.76 & 0.31 \\ 
  $Without$ & $\textit{x} \times \textit{z} + \textit{z} \times \textit{z}$ & $\textit{x} \sim Binary, \textit{z} \sim Normal$ & 250 & FALSE & 0.20 & 2.00 & 1.00 & 0.13 & 0.05 \\ 
  $Without$ & $\textit{x} \times \textit{z} + \textit{z} \times \textit{z}$ & $\textit{x} \sim Normal , \textit{z} \sim Normal$ & 300 & FALSE & 0.20 & 2.00 & 1.00 & 0.12 & 0.05 \\ 
  $Without$ & $\textit{x} \times \textit{z} + \textit{z} \times \textit{z}$ & $\textit{x} \sim Binary, \textit{z} \sim Binary$ & 300 & FALSE & 0.20 & 2.00 & 1.00 & 0.29 & 0.10 \\ 
  $Without$ & $\textit{x} \times \textit{z} + \textit{z} \times \textit{z}$ & $\textit{x} \sim Normal, \textit{z} \sim Binary$ & 300 & FALSE & 0.20 & 2.00 & 1.00 & 0.83 & 0.35 \\ 
  $Without$ & $\textit{x} \times \textit{z} + \textit{z} \times \textit{z}$ & $\textit{x} \sim Binary, \textit{z} \sim Normal$ & 300 & FALSE & 0.20 & 2.00 & 1.00 & 0.13 & 0.05 \\ 
  $Without$ & $\textit{x} \times \textit{z} + \textit{z} \times \textit{z}$ & $\textit{x} \sim Normal , \textit{z} \sim Normal$ & 200 & FALSE & 0.30 & 2.00 & 1.00 & 0.15 & 0.06 \\ 
  $Without$ & $\textit{x} \times \textit{z} + \textit{z} \times \textit{z}$ & $\textit{x} \sim Binary, \textit{z} \sim Binary$ & 200 & FALSE & 0.30 & 2.00 & 1.00 & 0.40 & 0.14 \\ 
  $Without$ & $\textit{x} \times \textit{z} + \textit{z} \times \textit{z}$ & $\textit{x} \sim Normal, \textit{z} \sim Binary$ & 200 & FALSE & 0.30 & 2.00 & 1.00 & 0.96 & 0.44 \\ 
  $Without$ & $\textit{x} \times \textit{z} + \textit{z} \times \textit{z}$ & $\textit{x} \sim Binary, \textit{z} \sim Normal$ & 200 & FALSE & 0.30 & 2.00 & 1.00 & 0.14 & 0.05 \\ 
  $Without$ & $\textit{x} \times \textit{z} + \textit{z} \times \textit{z}$ & $\textit{x} \sim Normal , \textit{z} \sim Normal$ & 200 & FALSE & 0.40 & 2.00 & 1.00 & 0.18 & 0.06 \\ 
  $Without$ & $\textit{x} \times \textit{z} + \textit{z} \times \textit{z}$ & $\textit{x} \sim Binary, \textit{z} \sim Binary$ & 200 & FALSE & 0.40 & 2.00 & 1.00 & 0.64 & 0.23 \\ 
  $Without$ & $\textit{x} \times \textit{z} + \textit{z} \times \textit{z}$ & $\textit{x} \sim Normal, \textit{z} \sim Binary$ & 200 & FALSE & 0.40 & 2.00 & 1.00 & 1.00 & 0.51 \\ 
  $Without$ & $\textit{x} \times \textit{z} + \textit{z} \times \textit{z}$ & $\textit{x} \sim Binary, \textit{z} \sim Normal$ & 200 & FALSE & 0.40 & 2.00 & 1.00 & 0.15 & 0.05 \\ 
  $Without$ & $\textit{x} \times \textit{z} + \textit{z} \times \textit{z}$ & $\textit{x} \sim Normal , \textit{z} \sim Normal$ & 200 & TRUE & 0.20 & 2.00 & 1.00 & 0.25 & 0.05 \\ 
  $Without$ & $\textit{x} \times \textit{z} + \textit{z} \times \textit{z}$ & $\textit{x} \sim Binary, \textit{z} \sim Binary$ & 200 & TRUE & 0.20 & 2.00 & 1.00 & 0.33 & 0.08 \\ 
  $Without$ & $\textit{x} \times \textit{z} + \textit{z} \times \textit{z}$ & $\textit{x} \sim Normal, \textit{z} \sim Binary$ & 200 & TRUE & 0.20 & 2.00 & 1.00 & 0.75 & 0.22 \\ 
  $Without$ & $\textit{x} \times \textit{z} + \textit{z} \times \textit{z}$ & $\textit{x} \sim Binary, \textit{z} \sim Normal$ & 200 & TRUE & 0.20 & 2.00 & 1.00 & 0.24 & 0.05 \\ 
  $Without$ & $\textit{x} \times \textit{z} + \textit{z} \times \textit{z}$ & $\textit{x} \sim Normal , \textit{z} \sim Normal$ & 200 & FALSE & 0.20 & 2.00 & 3.00 & 0.23 & 0.05 \\ 
  $Without$ & $\textit{x} \times \textit{z} + \textit{z} \times \textit{z}$ & $\textit{x} \sim Binary, \textit{z} \sim Binary$ & 200 & FALSE & 0.20 & 2.00 & 3.00 & 0.41 & 0.09 \\ 
  $Without$ & $\textit{x} \times \textit{z} + \textit{z} \times \textit{z}$ & $\textit{x} \sim Normal, \textit{z} \sim Binary$ & 200 & FALSE & 0.20 & 2.00 & 3.00 & 0.85 & 0.28 \\ 
  $Without$ & $\textit{x} \times \textit{z} + \textit{z} \times \textit{z}$ & $\textit{x} \sim Binary, \textit{z} \sim Normal$ & 200 & FALSE & 0.20 & 2.00 & 3.00 & 0.24 & 0.05 \\ 
  $Without$ & $\textit{x} + \textit{z} + \textit{x} \times \textit{z} + \textit{z} \times \textit{z}$ & $\textit{x} \sim Normal , \textit{z} \sim Normal$ & 200 & FALSE & 0.20 & 2.00 & 1.00 & 0.14 & 0.05 \\ 
  $Without$ & $\textit{x} + \textit{z} + \textit{x} \times \textit{z} + \textit{z} \times \textit{z}$ & $\textit{x} \sim Normal , \textit{z} \sim Normal$ & 200 & FALSE & 0.20 & 3.00 & 1.00 & 0.25 & 0.05 \\ 
  $Without$ & $\textit{x} + \textit{z} + \textit{x} \times \textit{z} + \textit{z} \times \textit{z}$ & $\textit{x} \sim Binary, \textit{z} \sim Binary$ & 200 & FALSE & 0.20 & 2.00 & 1.00 & 0.36 & 0.08 \\ 
  $Without$ & $\textit{x} + \textit{z} + \textit{x} \times \textit{z} + \textit{z} \times \textit{z}$ & $\textit{x} \sim Binary, \textit{z} \sim Binary$ & 200 & FALSE & 0.20 & 3.00 & 1.00 & 0.89 & 0.13 \\ 
  $Without$ & $\textit{x} + \textit{z} + \textit{x} \times \textit{z} + \textit{z} \times \textit{z}$ & $\textit{x} \sim Normal, \textit{z} \sim Binary$ & 200 & FALSE & 0.20 & 2.00 & 1.00 & 0.69 & 0.16 \\ 
  $Without$ & $\textit{x} + \textit{z} + \textit{x} \times \textit{z} + \textit{z} \times \textit{z}$ & $\textit{x} \sim Normal, \textit{z} \sim Binary$ & 200 & FALSE & 0.20 & 3.00 & 1.00 & 0.90 & 0.27 \\ 
  $Without$ & $\textit{x} + \textit{z} + \textit{x} \times \textit{z} + \textit{z} \times \textit{z}$ & $\textit{x} \sim Binary, \textit{z} \sim Normal$ & 200 & FALSE & 0.20 & 2.00 & 1.00 & 0.14 & 0.05 \\ 
  $Without$ & $\textit{x} + \textit{z} + \textit{x} \times \textit{z} + \textit{z} \times \textit{z}$ & $\textit{x} \sim Binary, \textit{z} \sim Normal$ & 200 & FALSE & 0.20 & 3.00 & 1.00 & 0.25 & 0.04 \\ 
  $Without$ & $\textit{x} + \textit{z} + \textit{x} \times \textit{z} + \textit{z} \times \textit{z}$ & $\textit{x} \sim Normal , \textit{z} \sim Normal$ & 100 & FALSE & 0.20 & 2.00 & 1.00 & 0.14 & 0.05 \\ 
  $Without$ & $\textit{x} + \textit{z} + \textit{x} \times \textit{z} + \textit{z} \times \textit{z}$ & $\textit{x} \sim Binary, \textit{z} \sim Binary$ & 100 & FALSE & 0.20 & 2.00 & 1.00 & 0.25 & 0.06 \\ 
  $Without$ & $\textit{x} + \textit{z} + \textit{x} \times \textit{z} + \textit{z} \times \textit{z}$ & $\textit{x} \sim Normal, \textit{z} \sim Binary$ & 100 & FALSE & 0.20 & 2.00 & 1.00 & 0.42 & 0.10 \\ 
  $Without$ & $\textit{x} + \textit{z} + \textit{x} \times \textit{z} + \textit{z} \times \textit{z}$ & $\textit{x} \sim Binary, \textit{z} \sim Normal$ & 100 & FALSE & 0.20 & 2.00 & 1.00 & 0.15 & 0.04 \\ 
  $Without$ & $\textit{x} + \textit{z} + \textit{x} \times \textit{z} + \textit{z} \times \textit{z}$ & $\textit{x} \sim Normal , \textit{z} \sim Normal$ & 150 & FALSE & 0.20 & 2.00 & 1.00 & 0.14 & 0.05 \\ 
  $Without$ & $\textit{x} + \textit{z} + \textit{x} \times \textit{z} + \textit{z} \times \textit{z}$ & $\textit{x} \sim Binary, \textit{z} \sim Binary$ & 150 & FALSE & 0.20 & 2.00 & 1.00 & 0.31 & 0.07 \\ 
  $Without$ & $\textit{x} + \textit{z} + \textit{x} \times \textit{z} + \textit{z} \times \textit{z}$ & $\textit{x} \sim Normal, \textit{z} \sim Binary$ & 150 & FALSE & 0.20 & 2.00 & 1.00 & 0.56 & 0.13 \\ 
  $Without$ & $\textit{x} + \textit{z} + \textit{x} \times \textit{z} + \textit{z} \times \textit{z}$ & $\textit{x} \sim Binary, \textit{z} \sim Normal$ & 150 & FALSE & 0.20 & 2.00 & 1.00 & 0.15 & 0.05 \\ 
  $Without$ & $\textit{x} + \textit{z} + \textit{x} \times \textit{z} + \textit{z} \times \textit{z}$ & $\textit{x} \sim Normal , \textit{z} \sim Normal$ & 250 & FALSE & 0.20 & 2.00 & 1.00 & 0.15 & 0.05 \\ 
  $Without$ & $\textit{x} + \textit{z} + \textit{x} \times \textit{z} + \textit{z} \times \textit{z}$ & $\textit{x} \sim Binary, \textit{z} \sim Binary$ & 250 & FALSE & 0.20 & 2.00 & 1.00 & 0.41 & 0.09 \\ 
  $Without$ & $\textit{x} + \textit{z} + \textit{x} \times \textit{z} + \textit{z} \times \textit{z}$ & $\textit{x} \sim Normal, \textit{z} \sim Binary$ & 250 & FALSE & 0.20 & 2.00 & 1.00 & 0.78 & 0.20 \\ 
  $Without$ & $\textit{x} + \textit{z} + \textit{x} \times \textit{z} + \textit{z} \times \textit{z}$ & $\textit{x} \sim Binary, \textit{z} \sim Normal$ & 250 & FALSE & 0.20 & 2.00 & 1.00 & 0.14 & 0.04 \\ 
  $Without$ & $\textit{x} + \textit{z} + \textit{x} \times \textit{z} + \textit{z} \times \textit{z}$ & $\textit{x} \sim Normal , \textit{z} \sim Normal$ & 300 & FALSE & 0.20 & 2.00 & 1.00 & 0.14 & 0.05 \\ 
  $Without$ & $\textit{x} + \textit{z} + \textit{x} \times \textit{z} + \textit{z} \times \textit{z}$ & $\textit{x} \sim Binary, \textit{z} \sim Binary$ & 300 & FALSE & 0.20 & 2.00 & 1.00 & 0.46 & 0.11 \\ 
  $Without$ & $\textit{x} + \textit{z} + \textit{x} \times \textit{z} + \textit{z} \times \textit{z}$ & $\textit{x} \sim Normal, \textit{z} \sim Binary$ & 300 & FALSE & 0.20 & 2.00 & 1.00 & 0.86 & 0.22 \\ 
  $Without$ & $\textit{x} + \textit{z} + \textit{x} \times \textit{z} + \textit{z} \times \textit{z}$ & $\textit{x} \sim Binary, \textit{z} \sim Normal$ & 300 & FALSE & 0.20 & 2.00 & 1.00 & 0.15 & 0.05 \\ 
  $Without$ & $\textit{x} + \textit{z} + \textit{x} \times \textit{z} + \textit{z} \times \textit{z}$ & $\textit{x} \sim Normal , \textit{z} \sim Normal$ & 200 & FALSE & 0.30 & 2.00 & 1.00 & 0.18 & 0.05 \\ 
  $Without$ & $\textit{x} + \textit{z} + \textit{x} \times \textit{z} + \textit{z} \times \textit{z}$ & $\textit{x} \sim Binary, \textit{z} \sim Binary$ & 200 & FALSE & 0.30 & 2.00 & 1.00 & 0.63 & 0.15 \\ 
  $Without$ & $\textit{x} + \textit{z} + \textit{x} \times \textit{z} + \textit{z} \times \textit{z}$ & $\textit{x} \sim Normal, \textit{z} \sim Binary$ & 200 & FALSE & 0.30 & 2.00 & 1.00 & 0.97 & 0.30 \\ 
  $Without$ & $\textit{x} + \textit{z} + \textit{x} \times \textit{z} + \textit{z} \times \textit{z}$ & $\textit{x} \sim Binary, \textit{z} \sim Normal$ & 200 & FALSE & 0.30 & 2.00 & 1.00 & 0.16 & 0.05 \\ 
  $Without$ & $\textit{x} + \textit{z} + \textit{x} \times \textit{z} + \textit{z} \times \textit{z}$ & $\textit{x} \sim Normal , \textit{z} \sim Normal$ & 200 & FALSE & 0.40 & 2.00 & 1.00 & 0.23 & 0.06 \\ 
  $Without$ & $\textit{x} + \textit{z} + \textit{x} \times \textit{z} + \textit{z} \times \textit{z}$ & $\textit{x} \sim Binary, \textit{z} \sim Binary$ & 200 & FALSE & 0.40 & 2.00 & 1.00 & 0.90 & 0.25 \\ 
  $Without$ & $\textit{x} + \textit{z} + \textit{x} \times \textit{z} + \textit{z} \times \textit{z}$ & $\textit{x} \sim Normal, \textit{z} \sim Binary$ & 200 & FALSE & 0.40 & 2.00 & 1.00 & 1.00 & 0.36 \\ 
  $Without$ & $\textit{x} + \textit{z} + \textit{x} \times \textit{z} + \textit{z} \times \textit{z}$ & $\textit{x} \sim Binary, \textit{z} \sim Normal$ & 200 & FALSE & 0.40 & 2.00 & 1.00 & 0.18 & 0.05 \\ 
  $Without$ & $\textit{x} + \textit{z} + \textit{x} \times \textit{z} + \textit{z} \times \textit{z}$ & $\textit{x} \sim Normal , \textit{z} \sim Normal$ & 200 & TRUE & 0.20 & 2.00 & 1.00 & 0.28 & 0.05 \\ 
  $Without$ & $\textit{x} + \textit{z} + \textit{x} \times \textit{z} + \textit{z} \times \textit{z}$ & $\textit{x} \sim Binary, \textit{z} \sim Binary$ & 200 & TRUE & 0.20 & 2.00 & 1.00 & 0.47 & 0.08 \\ 
  $Without$ & $\textit{x} + \textit{z} + \textit{x} \times \textit{z} + \textit{z} \times \textit{z}$ & $\textit{x} \sim Normal, \textit{z} \sim Binary$ & 200 & TRUE & 0.20 & 2.00 & 1.00 & 0.79 & 0.15 \\ 
  $Without$ & $\textit{x} + \textit{z} + \textit{x} \times \textit{z} + \textit{z} \times \textit{z}$ & $\textit{x} \sim Binary, \textit{z} \sim Normal$ & 200 & TRUE & 0.20 & 2.00 & 1.00 & 0.27 & 0.05 \\ 
  $Without$ & $\textit{x} + \textit{z} + \textit{x} \times \textit{z} + \textit{z} \times \textit{z}$ & $\textit{x} \sim Normal , \textit{z} \sim Normal$ & 200 & FALSE & 0.20 & 2.00 & 3.00 & 0.27 & 0.05 \\ 
  $Without$ & $\textit{x} + \textit{z} + \textit{x} \times \textit{z} + \textit{z} \times \textit{z}$ & $\textit{x} \sim Binary, \textit{z} \sim Binary$ & 200 & FALSE & 0.20 & 2.00 & 3.00 & 0.58 & 0.09 \\ 
  $Without$ & $\textit{x} + \textit{z} + \textit{x} \times \textit{z} + \textit{z} \times \textit{z}$ & $\textit{x} \sim Normal, \textit{z} \sim Binary$ & 200 & FALSE & 0.20 & 2.00 & 3.00 & 0.88 & 0.18 \\ 
  $Without$ & $\textit{x} + \textit{z} + \textit{x} \times \textit{z} + \textit{z} \times \textit{z}$ & $\textit{x} \sim Binary, \textit{z} \sim Normal$ & 200 & FALSE & 0.20 & 2.00 & 3.00 & 0.27 & 0.04 \\ 
  $With$ & $\textit{x} + \textit{z}$ & $\textit{x} \sim Normal , \textit{z} \sim Normal$ & 200 & FALSE & 0.20 & 2.00 & 1.00 & 0.07 & 0.05 \\ 
  $With$ & $\textit{x} + \textit{z}$ & $\textit{x} \sim Normal , \textit{z} \sim Normal$ & 200 & FALSE & 0.20 & 3.00 & 1.00 & 0.08 & 0.05 \\ 
  $With$ & $\textit{x} + \textit{z}$ & $\textit{x} \sim Binary, \textit{z} \sim Binary$ & 200 & FALSE & 0.20 & 2.00 & 1.00 & 0.07 & 0.05 \\ 
  $With$ & $\textit{x} + \textit{z}$ & $\textit{x} \sim Binary, \textit{z} \sim Binary$ & 200 & FALSE & 0.20 & 3.00 & 1.00 & 0.08 & 0.05 \\ 
  $With$ & $\textit{x} + \textit{z}$ & $\textit{x} \sim Normal, \textit{z} \sim Binary$ & 200 & FALSE & 0.20 & 2.00 & 1.00 & 0.07 & 0.05 \\ 
  $With$ & $\textit{x} + \textit{z}$ & $\textit{x} \sim Normal, \textit{z} \sim Binary$ & 200 & FALSE & 0.20 & 3.00 & 1.00 & 0.08 & 0.05 \\ 
  $With$ & $\textit{x} + \textit{z}$ & $\textit{x} \sim Binary, \textit{z} \sim Normal$ & 200 & FALSE & 0.20 & 2.00 & 1.00 & 0.07 & 0.05 \\ 
  $With$ & $\textit{x} + \textit{z}$ & $\textit{x} \sim Binary, \textit{z} \sim Normal$ & 200 & FALSE & 0.20 & 3.00 & 1.00 & 0.08 & 0.05 \\ 
  $With$ & $\textit{x} + \textit{z}$ & $\textit{x} \sim Normal , \textit{z} \sim Normal$ & 100 & FALSE & 0.20 & 2.00 & 1.00 & 0.07 & 0.05 \\ 
  $With$ & $\textit{x} + \textit{z}$ & $\textit{x} \sim Binary, \textit{z} \sim Binary$ & 100 & FALSE & 0.20 & 2.00 & 1.00 & 0.07 & 0.05 \\ 
  $With$ & $\textit{x} + \textit{z}$ & $\textit{x} \sim Normal, \textit{z} \sim Binary$ & 100 & FALSE & 0.20 & 2.00 & 1.00 & 0.07 & 0.05 \\ 
  $With$ & $\textit{x} + \textit{z}$ & $\textit{x} \sim Binary, \textit{z} \sim Normal$ & 100 & FALSE & 0.20 & 2.00 & 1.00 & 0.07 & 0.05 \\ 
  $With$ & $\textit{x} + \textit{z}$ & $\textit{x} \sim Normal , \textit{z} \sim Normal$ & 150 & FALSE & 0.20 & 2.00 & 1.00 & 0.07 & 0.05 \\ 
  $With$ & $\textit{x} + \textit{z}$ & $\textit{x} \sim Binary, \textit{z} \sim Binary$ & 150 & FALSE & 0.20 & 2.00 & 1.00 & 0.07 & 0.05 \\ 
  $With$ & $\textit{x} + \textit{z}$ & $\textit{x} \sim Normal, \textit{z} \sim Binary$ & 150 & FALSE & 0.20 & 2.00 & 1.00 & 0.07 & 0.05 \\ 
  $With$ & $\textit{x} + \textit{z}$ & $\textit{x} \sim Binary, \textit{z} \sim Normal$ & 150 & FALSE & 0.20 & 2.00 & 1.00 & 0.07 & 0.05 \\ 
  $With$ & $\textit{x} + \textit{z}$ & $\textit{x} \sim Normal , \textit{z} \sim Normal$ & 250 & FALSE & 0.20 & 2.00 & 1.00 & 0.07 & 0.05 \\ 
  $With$ & $\textit{x} + \textit{z}$ & $\textit{x} \sim Binary, \textit{z} \sim Binary$ & 250 & FALSE & 0.20 & 2.00 & 1.00 & 0.07 & 0.05 \\ 
  $With$ & $\textit{x} + \textit{z}$ & $\textit{x} \sim Normal, \textit{z} \sim Binary$ & 250 & FALSE & 0.20 & 2.00 & 1.00 & 0.07 & 0.05 \\ 
  $With$ & $\textit{x} + \textit{z}$ & $\textit{x} \sim Binary, \textit{z} \sim Normal$ & 250 & FALSE & 0.20 & 2.00 & 1.00 & 0.07 & 0.05 \\ 
  $With$ & $\textit{x} + \textit{z}$ & $\textit{x} \sim Normal , \textit{z} \sim Normal$ & 300 & FALSE & 0.20 & 2.00 & 1.00 & 0.07 & 0.05 \\ 
  $With$ & $\textit{x} + \textit{z}$ & $\textit{x} \sim Binary, \textit{z} \sim Binary$ & 300 & FALSE & 0.20 & 2.00 & 1.00 & 0.06 & 0.05 \\ 
  $With$ & $\textit{x} + \textit{z}$ & $\textit{x} \sim Normal, \textit{z} \sim Binary$ & 300 & FALSE & 0.20 & 2.00 & 1.00 & 0.07 & 0.05 \\ 
  $With$ & $\textit{x} + \textit{z}$ & $\textit{x} \sim Binary, \textit{z} \sim Normal$ & 300 & FALSE & 0.20 & 2.00 & 1.00 & 0.07 & 0.05 \\ 
  $With$ & $\textit{x} + \textit{z}$ & $\textit{x} \sim Normal , \textit{z} \sim Normal$ & 200 & FALSE & 0.30 & 2.00 & 1.00 & 0.08 & 0.05 \\ 
  $With$ & $\textit{x} + \textit{z}$ & $\textit{x} \sim Binary, \textit{z} \sim Binary$ & 200 & FALSE & 0.30 & 2.00 & 1.00 & 0.08 & 0.05 \\ 
  $With$ & $\textit{x} + \textit{z}$ & $\textit{x} \sim Normal, \textit{z} \sim Binary$ & 200 & FALSE & 0.30 & 2.00 & 1.00 & 0.08 & 0.05 \\ 
  $With$ & $\textit{x} + \textit{z}$ & $\textit{x} \sim Binary, \textit{z} \sim Normal$ & 200 & FALSE & 0.30 & 2.00 & 1.00 & 0.08 & 0.05 \\ 
  $With$ & $\textit{x} + \textit{z}$ & $\textit{x} \sim Normal , \textit{z} \sim Normal$ & 200 & FALSE & 0.40 & 2.00 & 1.00 & 0.08 & 0.05 \\ 
  $With$ & $\textit{x} + \textit{z}$ & $\textit{x} \sim Binary, \textit{z} \sim Binary$ & 200 & FALSE & 0.40 & 2.00 & 1.00 & 0.09 & 0.05 \\ 
  $With$ & $\textit{x} + \textit{z}$ & $\textit{x} \sim Normal, \textit{z} \sim Binary$ & 200 & FALSE & 0.40 & 2.00 & 1.00 & 0.09 & 0.05 \\ 
  $With$ & $\textit{x} + \textit{z}$ & $\textit{x} \sim Binary, \textit{z} \sim Normal$ & 200 & FALSE & 0.40 & 2.00 & 1.00 & 0.09 & 0.05 \\ 
  $With$ & $\textit{x} + \textit{z}$ & $\textit{x} \sim Normal , \textit{z} \sim Normal$ & 200 & TRUE & 0.20 & 2.00 & 1.00 & 0.13 & 0.05 \\ 
  $With$ & $\textit{x} + \textit{z}$ & $\textit{x} \sim Binary, \textit{z} \sim Binary$ & 200 & TRUE & 0.20 & 2.00 & 1.00 & 0.11 & 0.05 \\ 
  $With$ & $\textit{x} + \textit{z}$ & $\textit{x} \sim Normal, \textit{z} \sim Binary$ & 200 & TRUE & 0.20 & 2.00 & 1.00 & 0.12 & 0.05 \\ 
  $With$ & $\textit{x} + \textit{z}$ & $\textit{x} \sim Binary, \textit{z} \sim Normal$ & 200 & TRUE & 0.20 & 2.00 & 1.00 & 0.13 & 0.05 \\ 
  $With$ & $\textit{x} + \textit{z}$ & $\textit{x} \sim Normal , \textit{z} \sim Normal$ & 200 & FALSE & 0.20 & 2.00 & 3.00 & 0.13 & 0.05 \\ 
  $With$ & $\textit{x} + \textit{z}$ & $\textit{x} \sim Binary, \textit{z} \sim Binary$ & 200 & FALSE & 0.20 & 2.00 & 3.00 & 0.13 & 0.05 \\ 
  $With$ & $\textit{x} + \textit{z}$ & $\textit{x} \sim Normal, \textit{z} \sim Binary$ & 200 & FALSE & 0.20 & 2.00 & 3.00 & 0.13 & 0.05 \\ 
  $With$ & $\textit{x} + \textit{z}$ & $\textit{x} \sim Binary, \textit{z} \sim Normal$ & 200 & FALSE & 0.20 & 2.00 & 3.00 & 0.13 & 0.05 \\ 
  $With$ & $\textit{x} + \textit{z} + \textit{x} \times \textit{z}$ & $\textit{x} \sim Normal , \textit{z} \sim Normal$ & 200 & FALSE & 0.20 & 2.00 & 1.00 & 0.12 & 0.05 \\ 
  $With$ & $\textit{x} + \textit{z} + \textit{x} \times \textit{z}$ & $\textit{x} \sim Normal , \textit{z} \sim Normal$ & 200 & FALSE & 0.20 & 3.00 & 1.00 & 0.16 & 0.05 \\ 
  $With$ & $\textit{x} + \textit{z} + \textit{x} \times \textit{z}$ & $\textit{x} \sim Binary, \textit{z} \sim Binary$ & 200 & FALSE & 0.20 & 2.00 & 1.00 & 0.13 & 0.05 \\ 
  $With$ & $\textit{x} + \textit{z} + \textit{x} \times \textit{z}$ & $\textit{x} \sim Binary, \textit{z} \sim Binary$ & 200 & FALSE & 0.20 & 3.00 & 1.00 & 0.18 & 0.05 \\ 
  $With$ & $\textit{x} + \textit{z} + \textit{x} \times \textit{z}$ & $\textit{x} \sim Normal, \textit{z} \sim Binary$ & 200 & FALSE & 0.20 & 2.00 & 1.00 & 0.12 & 0.05 \\ 
  $With$ & $\textit{x} + \textit{z} + \textit{x} \times \textit{z}$ & $\textit{x} \sim Normal, \textit{z} \sim Binary$ & 200 & FALSE & 0.20 & 3.00 & 1.00 & 0.16 & 0.05 \\ 
  $With$ & $\textit{x} + \textit{z} + \textit{x} \times \textit{z}$ & $\textit{x} \sim Binary, \textit{z} \sim Normal$ & 200 & FALSE & 0.20 & 2.00 & 1.00 & 0.13 & 0.05 \\ 
  $With$ & $\textit{x} + \textit{z} + \textit{x} \times \textit{z}$ & $\textit{x} \sim Binary, \textit{z} \sim Normal$ & 200 & FALSE & 0.20 & 3.00 & 1.00 & 0.18 & 0.05 \\ 
  $With$ & $\textit{x} + \textit{z} + \textit{x} \times \textit{z}$ & $\textit{x} \sim Normal , \textit{z} \sim Normal$ & 100 & FALSE & 0.20 & 2.00 & 1.00 & 0.12 & 0.05 \\ 
  $With$ & $\textit{x} + \textit{z} + \textit{x} \times \textit{z}$ & $\textit{x} \sim Binary, \textit{z} \sim Binary$ & 100 & FALSE & 0.20 & 2.00 & 1.00 & 0.13 & 0.05 \\ 
  $With$ & $\textit{x} + \textit{z} + \textit{x} \times \textit{z}$ & $\textit{x} \sim Normal, \textit{z} \sim Binary$ & 100 & FALSE & 0.20 & 2.00 & 1.00 & 0.12 & 0.05 \\ 
  $With$ & $\textit{x} + \textit{z} + \textit{x} \times \textit{z}$ & $\textit{x} \sim Binary, \textit{z} \sim Normal$ & 100 & FALSE & 0.20 & 2.00 & 1.00 & 0.13 & 0.05 \\ 
  $With$ & $\textit{x} + \textit{z} + \textit{x} \times \textit{z}$ & $\textit{x} \sim Normal , \textit{z} \sim Normal$ & 150 & FALSE & 0.20 & 2.00 & 1.00 & 0.11 & 0.05 \\ 
  $With$ & $\textit{x} + \textit{z} + \textit{x} \times \textit{z}$ & $\textit{x} \sim Binary, \textit{z} \sim Binary$ & 150 & FALSE & 0.20 & 2.00 & 1.00 & 0.13 & 0.05 \\ 
  $With$ & $\textit{x} + \textit{z} + \textit{x} \times \textit{z}$ & $\textit{x} \sim Normal, \textit{z} \sim Binary$ & 150 & FALSE & 0.20 & 2.00 & 1.00 & 0.11 & 0.05 \\ 
  $With$ & $\textit{x} + \textit{z} + \textit{x} \times \textit{z}$ & $\textit{x} \sim Binary, \textit{z} \sim Normal$ & 150 & FALSE & 0.20 & 2.00 & 1.00 & 0.13 & 0.05 \\ 
  $With$ & $\textit{x} + \textit{z} + \textit{x} \times \textit{z}$ & $\textit{x} \sim Normal , \textit{z} \sim Normal$ & 250 & FALSE & 0.20 & 2.00 & 1.00 & 0.12 & 0.05 \\ 
  $With$ & $\textit{x} + \textit{z} + \textit{x} \times \textit{z}$ & $\textit{x} \sim Binary, \textit{z} \sim Binary$ & 250 & FALSE & 0.20 & 2.00 & 1.00 & 0.13 & 0.05 \\ 
  $With$ & $\textit{x} + \textit{z} + \textit{x} \times \textit{z}$ & $\textit{x} \sim Normal, \textit{z} \sim Binary$ & 250 & FALSE & 0.20 & 2.00 & 1.00 & 0.12 & 0.05 \\ 
  $With$ & $\textit{x} + \textit{z} + \textit{x} \times \textit{z}$ & $\textit{x} \sim Binary, \textit{z} \sim Normal$ & 250 & FALSE & 0.20 & 2.00 & 1.00 & 0.13 & 0.05 \\ 
  $With$ & $\textit{x} + \textit{z} + \textit{x} \times \textit{z}$ & $\textit{x} \sim Normal , \textit{z} \sim Normal$ & 300 & FALSE & 0.20 & 2.00 & 1.00 & 0.12 & 0.05 \\ 
  $With$ & $\textit{x} + \textit{z} + \textit{x} \times \textit{z}$ & $\textit{x} \sim Binary, \textit{z} \sim Binary$ & 300 & FALSE & 0.20 & 2.00 & 1.00 & 0.13 & 0.05 \\ 
  $With$ & $\textit{x} + \textit{z} + \textit{x} \times \textit{z}$ & $\textit{x} \sim Normal, \textit{z} \sim Binary$ & 300 & FALSE & 0.20 & 2.00 & 1.00 & 0.12 & 0.05 \\ 
  $With$ & $\textit{x} + \textit{z} + \textit{x} \times \textit{z}$ & $\textit{x} \sim Binary, \textit{z} \sim Normal$ & 300 & FALSE & 0.20 & 2.00 & 1.00 & 0.13 & 0.05 \\ 
  $With$ & $\textit{x} + \textit{z} + \textit{x} \times \textit{z}$ & $\textit{x} \sim Normal , \textit{z} \sim Normal$ & 200 & FALSE & 0.30 & 2.00 & 1.00 & 0.13 & 0.05 \\ 
  $With$ & $\textit{x} + \textit{z} + \textit{x} \times \textit{z}$ & $\textit{x} \sim Binary, \textit{z} \sim Binary$ & 200 & FALSE & 0.30 & 2.00 & 1.00 & 0.14 & 0.05 \\ 
  $With$ & $\textit{x} + \textit{z} + \textit{x} \times \textit{z}$ & $\textit{x} \sim Normal, \textit{z} \sim Binary$ & 200 & FALSE & 0.30 & 2.00 & 1.00 & 0.13 & 0.05 \\ 
  $With$ & $\textit{x} + \textit{z} + \textit{x} \times \textit{z}$ & $\textit{x} \sim Binary, \textit{z} \sim Normal$ & 200 & FALSE & 0.30 & 2.00 & 1.00 & 0.14 & 0.05 \\ 
  $With$ & $\textit{x} + \textit{z} + \textit{x} \times \textit{z}$ & $\textit{x} \sim Normal , \textit{z} \sim Normal$ & 200 & FALSE & 0.40 & 2.00 & 1.00 & 0.14 & 0.05 \\ 
  $With$ & $\textit{x} + \textit{z} + \textit{x} \times \textit{z}$ & $\textit{x} \sim Binary, \textit{z} \sim Binary$ & 200 & FALSE & 0.40 & 2.00 & 1.00 & 0.16 & 0.05 \\ 
  $With$ & $\textit{x} + \textit{z} + \textit{x} \times \textit{z}$ & $\textit{x} \sim Normal, \textit{z} \sim Binary$ & 200 & FALSE & 0.40 & 2.00 & 1.00 & 0.14 & 0.05 \\ 
  $With$ & $\textit{x} + \textit{z} + \textit{x} \times \textit{z}$ & $\textit{x} \sim Binary, \textit{z} \sim Normal$ & 200 & FALSE & 0.40 & 2.00 & 1.00 & 0.16 & 0.05 \\ 
  $With$ & $\textit{x} + \textit{z} + \textit{x} \times \textit{z}$ & $\textit{x} \sim Normal , \textit{z} \sim Normal$ & 200 & TRUE & 0.20 & 2.00 & 1.00 & 0.24 & 0.05 \\ 
  $With$ & $\textit{x} + \textit{z} + \textit{x} \times \textit{z}$ & $\textit{x} \sim Binary, \textit{z} \sim Binary$ & 200 & TRUE & 0.20 & 2.00 & 1.00 & 0.21 & 0.05 \\ 
  $With$ & $\textit{x} + \textit{z} + \textit{x} \times \textit{z}$ & $\textit{x} \sim Normal, \textit{z} \sim Binary$ & 200 & TRUE & 0.20 & 2.00 & 1.00 & 0.22 & 0.05 \\ 
  $With$ & $\textit{x} + \textit{z} + \textit{x} \times \textit{z}$ & $\textit{x} \sim Binary, \textit{z} \sim Normal$ & 200 & TRUE & 0.20 & 2.00 & 1.00 & 0.24 & 0.05 \\ 
  $With$ & $\textit{x} + \textit{z} + \textit{x} \times \textit{z}$ & $\textit{x} \sim Normal , \textit{z} \sim Normal$ & 200 & FALSE & 0.20 & 2.00 & 3.00 & 0.22 & 0.05 \\ 
  $With$ & $\textit{x} + \textit{z} + \textit{x} \times \textit{z}$ & $\textit{x} \sim Binary, \textit{z} \sim Binary$ & 200 & FALSE & 0.20 & 2.00 & 3.00 & 0.24 & 0.05 \\ 
  $With$ & $\textit{x} + \textit{z} + \textit{x} \times \textit{z}$ & $\textit{x} \sim Normal, \textit{z} \sim Binary$ & 200 & FALSE & 0.20 & 2.00 & 3.00 & 0.22 & 0.05 \\ 
  $With$ & $\textit{x} + \textit{z} + \textit{x} \times \textit{z}$ & $\textit{x} \sim Binary, \textit{z} \sim Normal$ & 200 & FALSE & 0.20 & 2.00 & 3.00 & 0.25 & 0.05 \\ 
  $With$ & $\textit{x} + \textit{z} + \textit{z} \times \textit{z}$ & $\textit{x} \sim Normal , \textit{z} \sim Normal$ & 200 & FALSE & 0.20 & 2.00 & 1.00 & 0.07 & 0.05 \\ 
  $With$ & $\textit{x} + \textit{z} + \textit{z} \times \textit{z}$ & $\textit{x} \sim Normal , \textit{z} \sim Normal$ & 200 & FALSE & 0.20 & 3.00 & 1.00 & 0.09 & 0.05 \\ 
  $With$ & $\textit{x} + \textit{z} + \textit{z} \times \textit{z}$ & $\textit{x} \sim Binary, \textit{z} \sim Binary$ & 200 & FALSE & 0.20 & 2.00 & 1.00 & 0.07 & 0.05 \\ 
  $With$ & $\textit{x} + \textit{z} + \textit{z} \times \textit{z}$ & $\textit{x} \sim Binary, \textit{z} \sim Binary$ & 200 & FALSE & 0.20 & 3.00 & 1.00 & 0.08 & 0.05 \\ 
  $With$ & $\textit{x} + \textit{z} + \textit{z} \times \textit{z}$ & $\textit{x} \sim Normal, \textit{z} \sim Binary$ & 200 & FALSE & 0.20 & 2.00 & 1.00 & 0.07 & 0.05 \\ 
  $With$ & $\textit{x} + \textit{z} + \textit{z} \times \textit{z}$ & $\textit{x} \sim Normal, \textit{z} \sim Binary$ & 200 & FALSE & 0.20 & 3.00 & 1.00 & 0.09 & 0.05 \\ 
  $With$ & $\textit{x} + \textit{z} + \textit{z} \times \textit{z}$ & $\textit{x} \sim Binary, \textit{z} \sim Normal$ & 200 & FALSE & 0.20 & 2.00 & 1.00 & 0.07 & 0.05 \\ 
  $With$ & $\textit{x} + \textit{z} + \textit{z} \times \textit{z}$ & $\textit{x} \sim Binary, \textit{z} \sim Normal$ & 200 & FALSE & 0.20 & 3.00 & 1.00 & 0.09 & 0.05 \\ 
  $With$ & $\textit{x} + \textit{z} + \textit{z} \times \textit{z}$ & $\textit{x} \sim Normal , \textit{z} \sim Normal$ & 100 & FALSE & 0.20 & 2.00 & 1.00 & 0.07 & 0.05 \\ 
  $With$ & $\textit{x} + \textit{z} + \textit{z} \times \textit{z}$ & $\textit{x} \sim Binary, \textit{z} \sim Binary$ & 100 & FALSE & 0.20 & 2.00 & 1.00 & 0.07 & 0.05 \\ 
  $With$ & $\textit{x} + \textit{z} + \textit{z} \times \textit{z}$ & $\textit{x} \sim Normal, \textit{z} \sim Binary$ & 100 & FALSE & 0.20 & 2.00 & 1.00 & 0.07 & 0.05 \\ 
  $With$ & $\textit{x} + \textit{z} + \textit{z} \times \textit{z}$ & $\textit{x} \sim Binary, \textit{z} \sim Normal$ & 100 & FALSE & 0.20 & 2.00 & 1.00 & 0.07 & 0.05 \\ 
  $With$ & $\textit{x} + \textit{z} + \textit{z} \times \textit{z}$ & $\textit{x} \sim Normal , \textit{z} \sim Normal$ & 150 & FALSE & 0.20 & 2.00 & 1.00 & 0.07 & 0.05 \\ 
  $With$ & $\textit{x} + \textit{z} + \textit{z} \times \textit{z}$ & $\textit{x} \sim Binary, \textit{z} \sim Binary$ & 150 & FALSE & 0.20 & 2.00 & 1.00 & 0.07 & 0.05 \\ 
  $With$ & $\textit{x} + \textit{z} + \textit{z} \times \textit{z}$ & $\textit{x} \sim Normal, \textit{z} \sim Binary$ & 150 & FALSE & 0.20 & 2.00 & 1.00 & 0.07 & 0.05 \\ 
  $With$ & $\textit{x} + \textit{z} + \textit{z} \times \textit{z}$ & $\textit{x} \sim Binary, \textit{z} \sim Normal$ & 150 & FALSE & 0.20 & 2.00 & 1.00 & 0.07 & 0.05 \\ 
  $With$ & $\textit{x} + \textit{z} + \textit{z} \times \textit{z}$ & $\textit{x} \sim Normal , \textit{z} \sim Normal$ & 250 & FALSE & 0.20 & 2.00 & 1.00 & 0.07 & 0.05 \\ 
  $With$ & $\textit{x} + \textit{z} + \textit{z} \times \textit{z}$ & $\textit{x} \sim Binary, \textit{z} \sim Binary$ & 250 & FALSE & 0.20 & 2.00 & 1.00 & 0.07 & 0.05 \\ 
  $With$ & $\textit{x} + \textit{z} + \textit{z} \times \textit{z}$ & $\textit{x} \sim Normal, \textit{z} \sim Binary$ & 250 & FALSE & 0.20 & 2.00 & 1.00 & 0.07 & 0.05 \\ 
  $With$ & $\textit{x} + \textit{z} + \textit{z} \times \textit{z}$ & $\textit{x} \sim Binary, \textit{z} \sim Normal$ & 250 & FALSE & 0.20 & 2.00 & 1.00 & 0.07 & 0.05 \\ 
  $With$ & $\textit{x} + \textit{z} + \textit{z} \times \textit{z}$ & $\textit{x} \sim Normal , \textit{z} \sim Normal$ & 300 & FALSE & 0.20 & 2.00 & 1.00 & 0.07 & 0.05 \\ 
  $With$ & $\textit{x} + \textit{z} + \textit{z} \times \textit{z}$ & $\textit{x} \sim Binary, \textit{z} \sim Binary$ & 300 & FALSE & 0.20 & 2.00 & 1.00 & 0.07 & 0.05 \\ 
  $With$ & $\textit{x} + \textit{z} + \textit{z} \times \textit{z}$ & $\textit{x} \sim Normal, \textit{z} \sim Binary$ & 300 & FALSE & 0.20 & 2.00 & 1.00 & 0.07 & 0.05 \\ 
  $With$ & $\textit{x} + \textit{z} + \textit{z} \times \textit{z}$ & $\textit{x} \sim Binary, \textit{z} \sim Normal$ & 300 & FALSE & 0.20 & 2.00 & 1.00 & 0.07 & 0.05 \\ 
  $With$ & $\textit{x} + \textit{z} + \textit{z} \times \textit{z}$ & $\textit{x} \sim Normal , \textit{z} \sim Normal$ & 200 & FALSE & 0.30 & 2.00 & 1.00 & 0.08 & 0.05 \\ 
  $With$ & $\textit{x} + \textit{z} + \textit{z} \times \textit{z}$ & $\textit{x} \sim Binary, \textit{z} \sim Binary$ & 200 & FALSE & 0.30 & 2.00 & 1.00 & 0.08 & 0.05 \\ 
  $With$ & $\textit{x} + \textit{z} + \textit{z} \times \textit{z}$ & $\textit{x} \sim Normal, \textit{z} \sim Binary$ & 200 & FALSE & 0.30 & 2.00 & 1.00 & 0.08 & 0.05 \\ 
  $With$ & $\textit{x} + \textit{z} + \textit{z} \times \textit{z}$ & $\textit{x} \sim Binary, \textit{z} \sim Normal$ & 200 & FALSE & 0.30 & 2.00 & 1.00 & 0.08 & 0.05 \\ 
  $With$ & $\textit{x} + \textit{z} + \textit{z} \times \textit{z}$ & $\textit{x} \sim Normal , \textit{z} \sim Normal$ & 200 & FALSE & 0.40 & 2.00 & 1.00 & 0.09 & 0.05 \\ 
  $With$ & $\textit{x} + \textit{z} + \textit{z} \times \textit{z}$ & $\textit{x} \sim Binary, \textit{z} \sim Binary$ & 200 & FALSE & 0.40 & 2.00 & 1.00 & 0.09 & 0.05 \\ 
  $With$ & $\textit{x} + \textit{z} + \textit{z} \times \textit{z}$ & $\textit{x} \sim Normal, \textit{z} \sim Binary$ & 200 & FALSE & 0.40 & 2.00 & 1.00 & 0.09 & 0.05 \\ 
  $With$ & $\textit{x} + \textit{z} + \textit{z} \times \textit{z}$ & $\textit{x} \sim Binary, \textit{z} \sim Normal$ & 200 & FALSE & 0.40 & 2.00 & 1.00 & 0.09 & 0.05 \\ 
  $With$ & $\textit{x} + \textit{z} + \textit{z} \times \textit{z}$ & $\textit{x} \sim Normal , \textit{z} \sim Normal$ & 200 & TRUE & 0.20 & 2.00 & 1.00 & 0.13 & 0.05 \\ 
  $With$ & $\textit{x} + \textit{z} + \textit{z} \times \textit{z}$ & $\textit{x} \sim Binary, \textit{z} \sim Binary$ & 200 & TRUE & 0.20 & 2.00 & 1.00 & 0.11 & 0.05 \\ 
  $With$ & $\textit{x} + \textit{z} + \textit{z} \times \textit{z}$ & $\textit{x} \sim Normal, \textit{z} \sim Binary$ & 200 & TRUE & 0.20 & 2.00 & 1.00 & 0.12 & 0.05 \\ 
  $With$ & $\textit{x} + \textit{z} + \textit{z} \times \textit{z}$ & $\textit{x} \sim Binary, \textit{z} \sim Normal$ & 200 & TRUE & 0.20 & 2.00 & 1.00 & 0.12 & 0.05 \\ 
  $With$ & $\textit{x} + \textit{z} + \textit{z} \times \textit{z}$ & $\textit{x} \sim Normal , \textit{z} \sim Normal$ & 200 & FALSE & 0.20 & 2.00 & 3.00 & 0.13 & 0.05 \\ 
  $With$ & $\textit{x} + \textit{z} + \textit{z} \times \textit{z}$ & $\textit{x} \sim Binary, \textit{z} \sim Binary$ & 200 & FALSE & 0.20 & 2.00 & 3.00 & 0.14 & 0.05 \\ 
  $With$ & $\textit{x} + \textit{z} + \textit{z} \times \textit{z}$ & $\textit{x} \sim Normal, \textit{z} \sim Binary$ & 200 & FALSE & 0.20 & 2.00 & 3.00 & 0.14 & 0.05 \\ 
  $With$ & $\textit{x} + \textit{z} + \textit{z} \times \textit{z}$ & $\textit{x} \sim Binary, \textit{z} \sim Normal$ & 200 & FALSE & 0.20 & 2.00 & 3.00 & 0.13 & 0.05 \\ 
  $With$ & $\textit{x} + \textit{z} + \textit{x} \times \textit{z} + \textit{z} \times \textit{z}$ & $\textit{x} \sim Normal , \textit{z} \sim Normal$ & 200 & FALSE & 0.20 & 2.00 & 1.00 & 0.11 & 0.05 \\ 
  $With$ & $\textit{x} + \textit{z} + \textit{x} \times \textit{z} + \textit{z} \times \textit{z}$ & $\textit{x} \sim Normal , \textit{z} \sim Normal$ & 200 & FALSE & 0.20 & 3.00 & 1.00 & 0.19 & 0.05 \\ 
  $With$ & $\textit{x} + \textit{z} + \textit{x} \times \textit{z} + \textit{z} \times \textit{z}$ & $\textit{x} \sim Binary, \textit{z} \sim Binary$ & 200 & FALSE & 0.20 & 2.00 & 1.00 & 0.13 & 0.05 \\ 
  $With$ & $\textit{x} + \textit{z} + \textit{x} \times \textit{z} + \textit{z} \times \textit{z}$ & $\textit{x} \sim Binary, \textit{z} \sim Binary$ & 200 & FALSE & 0.20 & 3.00 & 1.00 & 0.21 & 0.05 \\ 
  $With$ & $\textit{x} + \textit{z} + \textit{x} \times \textit{z} + \textit{z} \times \textit{z}$ & $\textit{x} \sim Normal, \textit{z} \sim Binary$ & 200 & FALSE & 0.20 & 2.00 & 1.00 & 0.11 & 0.05 \\ 
  $With$ & $\textit{x} + \textit{z} + \textit{x} \times \textit{z} + \textit{z} \times \textit{z}$ & $\textit{x} \sim Normal, \textit{z} \sim Binary$ & 200 & FALSE & 0.20 & 3.00 & 1.00 & 0.18 & 0.05 \\ 
  $With$ & $\textit{x} + \textit{z} + \textit{x} \times \textit{z} + \textit{z} \times \textit{z}$ & $\textit{x} \sim Binary, \textit{z} \sim Normal$ & 200 & FALSE & 0.20 & 2.00 & 1.00 & 0.13 & 0.05 \\ 
  $With$ & $\textit{x} + \textit{z} + \textit{x} \times \textit{z} + \textit{z} \times \textit{z}$ & $\textit{x} \sim Binary, \textit{z} \sim Normal$ & 200 & FALSE & 0.20 & 3.00 & 1.00 & 0.21 & 0.05 \\ 
  $With$ & $\textit{x} + \textit{z} + \textit{x} \times \textit{z} + \textit{z} \times \textit{z}$ & $\textit{x} \sim Normal , \textit{z} \sim Normal$ & 100 & FALSE & 0.20 & 2.00 & 1.00 & 0.12 & 0.05 \\ 
  $With$ & $\textit{x} + \textit{z} + \textit{x} \times \textit{z} + \textit{z} \times \textit{z}$ & $\textit{x} \sim Binary, \textit{z} \sim Binary$ & 100 & FALSE & 0.20 & 2.00 & 1.00 & 0.12 & 0.05 \\ 
  $With$ & $\textit{x} + \textit{z} + \textit{x} \times \textit{z} + \textit{z} \times \textit{z}$ & $\textit{x} \sim Normal, \textit{z} \sim Binary$ & 100 & FALSE & 0.20 & 2.00 & 1.00 & 0.11 & 0.05 \\ 
  $With$ & $\textit{x} + \textit{z} + \textit{x} \times \textit{z} + \textit{z} \times \textit{z}$ & $\textit{x} \sim Binary, \textit{z} \sim Normal$ & 100 & FALSE & 0.20 & 2.00 & 1.00 & 0.13 & 0.05 \\ 
  $With$ & $\textit{x} + \textit{z} + \textit{x} \times \textit{z} + \textit{z} \times \textit{z}$ & $\textit{x} \sim Normal , \textit{z} \sim Normal$ & 150 & FALSE & 0.20 & 2.00 & 1.00 & 0.11 & 0.05 \\ 
  $With$ & $\textit{x} + \textit{z} + \textit{x} \times \textit{z} + \textit{z} \times \textit{z}$ & $\textit{x} \sim Binary, \textit{z} \sim Binary$ & 150 & FALSE & 0.20 & 2.00 & 1.00 & 0.13 & 0.05 \\ 
  $With$ & $\textit{x} + \textit{z} + \textit{x} \times \textit{z} + \textit{z} \times \textit{z}$ & $\textit{x} \sim Normal, \textit{z} \sim Binary$ & 150 & FALSE & 0.20 & 2.00 & 1.00 & 0.12 & 0.05 \\ 
  $With$ & $\textit{x} + \textit{z} + \textit{x} \times \textit{z} + \textit{z} \times \textit{z}$ & $\textit{x} \sim Binary, \textit{z} \sim Normal$ & 150 & FALSE & 0.20 & 2.00 & 1.00 & 0.13 & 0.05 \\ 
  $With$ & $\textit{x} + \textit{z} + \textit{x} \times \textit{z} + \textit{z} \times \textit{z}$ & $\textit{x} \sim Normal , \textit{z} \sim Normal$ & 250 & FALSE & 0.20 & 2.00 & 1.00 & 0.11 & 0.05 \\ 
  $With$ & $\textit{x} + \textit{z} + \textit{x} \times \textit{z} + \textit{z} \times \textit{z}$ & $\textit{x} \sim Binary, \textit{z} \sim Binary$ & 250 & FALSE & 0.20 & 2.00 & 1.00 & 0.12 & 0.05 \\ 
  $With$ & $\textit{x} + \textit{z} + \textit{x} \times \textit{z} + \textit{z} \times \textit{z}$ & $\textit{x} \sim Normal, \textit{z} \sim Binary$ & 250 & FALSE & 0.20 & 2.00 & 1.00 & 0.11 & 0.05 \\ 
  $With$ & $\textit{x} + \textit{z} + \textit{x} \times \textit{z} + \textit{z} \times \textit{z}$ & $\textit{x} \sim Binary, \textit{z} \sim Normal$ & 250 & FALSE & 0.20 & 2.00 & 1.00 & 0.13 & 0.05 \\ 
  $With$ & $\textit{x} + \textit{z} + \textit{x} \times \textit{z} + \textit{z} \times \textit{z}$ & $\textit{x} \sim Normal , \textit{z} \sim Normal$ & 300 & FALSE & 0.20 & 2.00 & 1.00 & 0.11 & 0.05 \\ 
  $With$ & $\textit{x} + \textit{z} + \textit{x} \times \textit{z} + \textit{z} \times \textit{z}$ & $\textit{x} \sim Binary, \textit{z} \sim Binary$ & 300 & FALSE & 0.20 & 2.00 & 1.00 & 0.13 & 0.05 \\ 
  $With$ & $\textit{x} + \textit{z} + \textit{x} \times \textit{z} + \textit{z} \times \textit{z}$ & $\textit{x} \sim Normal, \textit{z} \sim Binary$ & 300 & FALSE & 0.20 & 2.00 & 1.00 & 0.11 & 0.05 \\ 
  $With$ & $\textit{x} + \textit{z} + \textit{x} \times \textit{z} + \textit{z} \times \textit{z}$ & $\textit{x} \sim Binary, \textit{z} \sim Normal$ & 300 & FALSE & 0.20 & 2.00 & 1.00 & 0.13 & 0.05 \\ 
  $With$ & $\textit{x} + \textit{z} + \textit{x} \times \textit{z} + \textit{z} \times \textit{z}$ & $\textit{x} \sim Normal , \textit{z} \sim Normal$ & 200 & FALSE & 0.30 & 2.00 & 1.00 & 0.12 & 0.05 \\ 
  $With$ & $\textit{x} + \textit{z} + \textit{x} \times \textit{z} + \textit{z} \times \textit{z}$ & $\textit{x} \sim Binary, \textit{z} \sim Binary$ & 200 & FALSE & 0.30 & 2.00 & 1.00 & 0.13 & 0.05 \\ 
  $With$ & $\textit{x} + \textit{z} + \textit{x} \times \textit{z} + \textit{z} \times \textit{z}$ & $\textit{x} \sim Normal, \textit{z} \sim Binary$ & 200 & FALSE & 0.30 & 2.00 & 1.00 & 0.13 & 0.05 \\ 
  $With$ & $\textit{x} + \textit{z} + \textit{x} \times \textit{z} + \textit{z} \times \textit{z}$ & $\textit{x} \sim Binary, \textit{z} \sim Normal$ & 200 & FALSE & 0.30 & 2.00 & 1.00 & 0.13 & 0.05 \\ 
  $With$ & $\textit{x} + \textit{z} + \textit{x} \times \textit{z} + \textit{z} \times \textit{z}$ & $\textit{x} \sim Normal , \textit{z} \sim Normal$ & 200 & FALSE & 0.40 & 2.00 & 1.00 & 0.13 & 0.05 \\ 
  $With$ & $\textit{x} + \textit{z} + \textit{x} \times \textit{z} + \textit{z} \times \textit{z}$ & $\textit{x} \sim Binary, \textit{z} \sim Binary$ & 200 & FALSE & 0.40 & 2.00 & 1.00 & 0.15 & 0.05 \\ 
  $With$ & $\textit{x} + \textit{z} + \textit{x} \times \textit{z} + \textit{z} \times \textit{z}$ & $\textit{x} \sim Normal, \textit{z} \sim Binary$ & 200 & FALSE & 0.40 & 2.00 & 1.00 & 0.13 & 0.05 \\ 
  $With$ & $\textit{x} + \textit{z} + \textit{x} \times \textit{z} + \textit{z} \times \textit{z}$ & $\textit{x} \sim Binary, \textit{z} \sim Normal$ & 200 & FALSE & 0.40 & 2.00 & 1.00 & 0.15 & 0.05 \\ 
  $With$ & $\textit{x} + \textit{z} + \textit{x} \times \textit{z} + \textit{z} \times \textit{z}$ & $\textit{x} \sim Normal , \textit{z} \sim Normal$ & 200 & TRUE & 0.20 & 2.00 & 1.00 & 0.24 & 0.05 \\ 
  $With$ & $\textit{x} + \textit{z} + \textit{x} \times \textit{z} + \textit{z} \times \textit{z}$ & $\textit{x} \sim Binary, \textit{z} \sim Binary$ & 200 & TRUE & 0.20 & 2.00 & 1.00 & 0.21 & 0.05 \\ 
  $With$ & $\textit{x} + \textit{z} + \textit{x} \times \textit{z} + \textit{z} \times \textit{z}$ & $\textit{x} \sim Normal, \textit{z} \sim Binary$ & 200 & TRUE & 0.20 & 2.00 & 1.00 & 0.20 & 0.05 \\ 
  $With$ & $\textit{x} + \textit{z} + \textit{x} \times \textit{z} + \textit{z} \times \textit{z}$ & $\textit{x} \sim Binary, \textit{z} \sim Normal$ & 200 & TRUE & 0.20 & 2.00 & 1.00 & 0.24 & 0.05 \\ 
  $With$ & $\textit{x} + \textit{z} + \textit{x} \times \textit{z} + \textit{z} \times \textit{z}$ & $\textit{x} \sim Normal , \textit{z} \sim Normal$ & 200 & FALSE & 0.20 & 2.00 & 3.00 & 0.22 & 0.05 \\ 
  $With$ & $\textit{x} + \textit{z} + \textit{x} \times \textit{z} + \textit{z} \times \textit{z}$ & $\textit{x} \sim Binary, \textit{z} \sim Binary$ & 200 & FALSE & 0.20 & 2.00 & 3.00 & 0.24 & 0.05 \\ 
  $With$ & $\textit{x} + \textit{z} + \textit{x} \times \textit{z} + \textit{z} \times \textit{z}$ & $\textit{x} \sim Normal, \textit{z} \sim Binary$ & 200 & FALSE & 0.20 & 2.00 & 3.00 & 0.22 & 0.05 \\ 
  $With$ & $\textit{x} + \textit{z} + \textit{x} \times \textit{z} + \textit{z} \times \textit{z}$ & $\textit{x} \sim Binary, \textit{z} \sim Normal$ & 200 & FALSE & 0.20 & 2.00 & 3.00 & 0.24 & 0.05 \\ 
   \hline
\hline
\label{tab:apptabFull}
\end{longtable}

\end{landscape}




\end{document}