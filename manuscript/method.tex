\section{Method}
Focusing on different flexibilities in data analysis, we investigated the false-positive probability (FPP) and false-positive ratio (FPR) for a key independent variable that was uncorrelated with any other variables using a simulation. We used the previously discussed flexibilities i.e., different model specifications, outlier deletion present or absent, several dependent variables, and different sample sizes. The effect of flexibilities was tested using different correlations between the dependent variable and covariates, and different data structures. Even though in some simple cases it would be possible to calculate a closed-form solution for the FPP and FPR e.g., when looking at only one main effect, we used a simulation since it was not clear how different flexibilities would impact the FPP and FPR.\\

\subsection{Model Set}
When a researcher has to select which model(s) to use for testing a hypothesis, there are a lot of possibilities regarding what variables to collect and how they should enter the model. In a completely randomized experiment, it would only be necessary to focus on the key variable of interest, which we define as $x_1$, and the dependent variable, $y$ \citep{angrist2008mostly}. However, in many cases a researcher is either interested in how the variable of interest interacts with other variables or there is a need to control for other variables, which we call covariates and denote as $Z=(z_1,z_2,..,z_n)$. For instance, this could be the case in a quasi-experimental setting where no complete randomization could be obtained or for observational data where it is necessary to control for different variables to avoid bias \citep{angrist2008mostly}. \\ 
To better understand the construction of model sets, we divided the models into seven sets based on different interaction structures. The first three sets contained the key variable of interest and simple interactions, whereas the last four sets were the combinations of the first three sets. For clarity, the model equations exclude the constant and the coefficient in front of each variable. The first set only includes models of the main effects of the variable of interest and the covariates ($x + z$). The second set includes models with a main effect of the variable of interest and interactions between the variable of interest and covariates ($x \times z$)\footnote{For simplicity, we omit the main effect of the variable of interest, $x$, in the notation for the second, third, and sixth model set.}. The third set includes models with a main effect of the variable of interest and covariate-covariate interactions ($z \times z$).
To obtain the last four model sets, we combine the first three model sets. There are two ways of populating any the seven model sets: with restrictions on main effects and without restrictions on main effects. Restrictions on main effects mean that whenever there is an interaction effect, the corresponding main effects have to be included in the model. Models without restrictions, on the other hand, are allowed to include interaction effects without the corresponding main effects. In Table \ref{modelsets2.1}, we show examples of all model sets with one variable of interest, $x_{1}$, and two covariates, $z_{1}$ and $z_{2}$, depending on the main effect restriction. \\
 
\begin{table}[!htbp]
\centering
\caption{}
\caption*{\footnotesize An overview of all of the model sets when there are two covariates and one dependent variable with the restriction that main effects should be present when having interactions between variables (column two) and with no restrictions that main effects must follow interactions (column three). The ME set contains all of the main effects; the $ME + X \times Cov$  set includes the main effects and interactions between the variable of interest and the covariates; the $ME + Cov \times Cov$ set contains the main effects and  interactions between the covariates; the $ME + X \times Cov + Cov \times Cov$ set contains the main effects, the interactions between the variable of interest and covariates, and the interactions between the covariates; the $X \times Cov$, $Cov \times Cov$ and $X \times Cov + Cov \times Cov$ sets are empty sets when there is restrictions on our interactions since they only contain the interactions but not the required main effects.}
\label{modelsets2.1}
\begin{threeparttable}
\scalebox{0.85}{%
\begin{tabular}{ccc}
\hline
Model set & Models with restrictions & Models without restrictions \\ \hline
ME & \begin{tabular}[c]{@{}c@{}}$y=x_1$\\  $y=x_1+z_1$\\  $y=x_1+z_2$ \\  $y=x_1+z_1+z_2$\end{tabular} & \begin{tabular}[c]{@{}c@{}}$y=x_1$\\  $y=x_1+z_1$\\  $y=x_1+z_2$ \\  $y=x_1+z_1+z_2$\end{tabular} \\
$X \times Cov$ & Empty set & \begin{tabular}[c]{@{}c@{}}\rule{0pt}{5ex}$y=x_1+x_1z_1$ \\ $y=x_1+x_1z_2$\\ $y=x_1+x_1z_1+x_1z_2$\end{tabular} \\
$Cov \times Cov$ & Empty set & \rule{0pt}{5ex}$y=x_1+z_1z_2$ \\
$ME + X \times Cov$ & \begin{tabular}[c]{@{}c@{}}$y=x_1+z_1+x_1z_1$  \\ $y=x_1+z_2+x_1z_2$ \\ $y=x_1+z_1+z_2+x_1z_1$  \\ $y=x_1+z_1+z_2+x_1z_2$  \\ $y=x_1+z_1+z_2+x_1z_1+x_1z_2$\end{tabular} & \begin{tabular}[c]{@{}c@{}}\rule{0pt}{5ex}$y=x_1+z_1+x_1z_1$\\ $y=x_1+z_1+x_1z_2$\\ $y=x_1+z_1+x_1z_1+x_1z_2$\\ $y=x_1+z_2+x_1z_1$\\ $y=x_1+z_2+x_1z_2$\\ $y=x_1+z_2+x_1z_1+x_1z_2$\\ $y=x_1+z_1+z_2+x_1z_1$\\ $y=x_1+z_1+z_2+x_1z_2$\\ $y=x_1+z_1+z_2+x_1z_1+x_1z_2$\end{tabular} \\
$ME + Cov \times Cov$ & $y=x_1+z_1+z_2+z_1z_2$ & \begin{tabular}[c]{@{}c@{}}\rule{0pt}{5ex}$y=x_1+z_1+z_1z_2$\\ $y=x_1+z_2+z_1z_2$\\ $y=x_1+z_1+z_2+z_1z_2$\end{tabular} \\
$X \times Cov + Cov \times Cov$ & Empty set & \begin{tabular}[c]{@{}c@{}}\rule{0pt}{5ex}$y=x_1+x_1z_1+z_1z_2$\\ $y=x_1+x_1z_2+z_1z_2$\\ $y=x_1+x_1z_1+x_1z_2+z_1z_2$ \end{tabular} \\  
$ME + X \times Cov + Cov \times Cov$ & \begin{tabular}[c]{@{}c@{}}$y=x_1+z_1+z_2+x_1z_1+z_1z_2$ \\ $y=x_1+z_1+z_2+x_1z_2+z_1z_2$ \\ $y=x_1+z_1+z_2+x_1z_1+x_1z_2+z_1z_2$\end{tabular} & \begin{tabular}[c]{@{}c@{}}\rule{0pt}{5ex}$y=x_1+z_1+x_1z_1+z_1z_2$\\ $y=x_1+z_1+x_1z_2+z_1z_2$\\ $y=x_1+z_1+x_1z_1+x_1z_2+z_1z_2$\\ $y=x_1+z_2+x_1z_1+z_1z_2$\\ $y=x_1+z_2+x_1z_2+z_1z_2$\\ $y=x_1+z_2+x_1z_1+x_1z_2+z_1z_2$\\ $y=x_1+z_1+z_2+x_1z_1+z_1z_2$\\  $y=x_1+z_1+z_2+x_1z_2+z_1z_2$\\ $y=x_1+z_1+z_2+x_1z_1+x_1z_2+z_1z_2$\end{tabular} \\ \hline
\end{tabular}
}%
\end{threeparttable}

\end{table}

Equation (1) shows how to calculate the total number of models given one key variable of interest and a given number of covariates with the main effects restriction that models with interaction effects must include the corresponding main effects. Each underlined section in Equation (1) can also be used to calculate the size of the specified model set. The derivation of this equation can be found in the \textit{Appendix} in the section \nameref{with_res}. \\

\begin{equation} 
\begin{aligned}
\#\ models={} & \underbrace{\left(2^n\right)}_{x + z}+\underbrace{\sum^n_{j=1}{\left(2^j-1\right)\binom{n}{j}}}_{x + z + x \times z} + \\ 
& \underbrace{\sum^n_{j=2}{\binom{n}{j}\left(2^{\frac{j\left(j-1\right)}{2}}-1\right)}}_{x + z + z \times z} + \\
& \underbrace{\sum^n_{j=2}{\binom{n}{j}\left(2^j-1\right)\left(2^{\frac{j\left(j-1\right)}{2}}-1\right)}}_{x + z + x \times z + z \times z}\ \  
\end{aligned}
\end{equation} 
where $n$ denotes the number of covariates (all the $z$ variables) collected and $\binom{n}{j}$ is the binomial coefficient, the calculation of which is explained in the \textit{Appendix} in the section \nameref{binomial}.
Equation (1) does not include the ($x \times z$), ($z \times z$), and ($x \times z + z \times z$) sets as these include interaction effects without the corresponding main effects, and these are therefore empty sets. For the case with two covariates, the total number of models is calculated as follows:

\begin{equation*}
\centering
\left(2^2\right)+
\sum^2_{j=1}{\left(2^j-1\right) \binom{2}{j}}+
\sum^2_{j=2}{\binom{2}{j}\left(2^{\frac{j\left(j-1\right)}{2}}-1\right)}+  
\sum^2_{j=2}{\binom{2}{j}\left(2^j-1\right)\left(2^{\frac{j\left(j-1\right)}{2}}-1\right)}
\end{equation*}

which is equal to $4+5+1+3*1=13$. The number of models in each model set for different numbers of covariates depending on main effect restrictions can be seen in Table \ref{FullModel}. In the simulation, we focus on two to three covariates since this is enough to provide an impression of how the increase in the size of the model set affects the FPP and FPR.  \\

% latex table generated in R 4.0.0 by xtable 1.8-4 package
% Tue Jan 19 16:32:31 2021
\begin{table}[!h]
\centering
\caption{The total number of models for any given set considering the different number of covariates, with and without restriction that main effects should be present when having interaction effects.} 
\label{FullModel}
\begin{tabular}{lccccc}
  \hline
  & \multicolumn{5}{c}{Number of covariates} \\\cmidrule{2-6}
With restrictions & 2 & 3 & 4 & 5 & 6 \\ 
  \hline
$x + z$ & 4 & 8 & 16 & 32 & 64 \\ 
  $x + z + x \times z$ & 5 & 19 & 65 & 211 & 665 \\ 
  $x + z + z \times z$ & 1 & 10 & 97 & 1418 & 40005 \\ 
  $x + z+ x \times z + z \times z$ & 3 & 58 & 1159 & 36958 & 2269799 \\
  \hline 
  Number of total models & 13 & 95 & 1337 & 38619 & 2310533 \\ 
  \hline \\
  Without restrictions \\ 
  \hline
  $x + z$ & 4 & 8 & 16 & 32 & 64 \\ 
  $x \times z$ & 3 & 7 & 15 & 31 & 63 \\ 
  $z \times z$ & 1 & 4 & 32 & 512 & 16384 \\ 
  $x + z + x \times z$ & 9 & 49 & 225 & 961 & 3969 \\ 
  $x + z + z \times z$ & 3 & 49 & 945 & 31713 & 2064321 \\ 
  $x \times z + z \times z$ & 3 & 49 & 945 & 31713 & 2064321 \\ 
  $x + z + x \times z + z$ \times z & 9 & 343 & 14175 & 983103 & 130052223 \\ 
  \hline
  Number of total models & 32 & 509 & 16353 & 1048065 & 134201345 \\ 
   \hline 
\multicolumn{6}{p{13cm}}{\footnotesize{Note: $x + z$ = models with main effects only; $x \times z$ = models with interactions between the variable of interest and covariates; $z \times z$ = models with interactions between covariates;  $x + z + x \times z$ = models with main effects and interactions between the variable of interest and covariates; $x + z + z \times z$ = models with main effects and interactions between covariates; $x \times z + z \times z$ = models with interactions between covariates and variable of interest and interactions between covariates; $x + z + x \times z + z \times z$ = models with main effects and interactions between the variable of interest and covariates and the interactions between covariates.}} 

\end{tabular}
\end{table}



\subsection{Outlier Criteria}
Following the findings of \cite{Leyes2013}, we included four different outlier criteria. Three of the criteria were based on the standard deviation ($SD = 2$, $SD = 2.5$, and $SD = 3$), and the remaining criterion used the interquartile range \citep{Rousseeuw2011}. Outlier criteria were applied to all variables in a dataset by omitting all observations from the analysis that exceeded the outlier criteria. 

\subsection{Multiple Dependent Variables}
As in \cite{Simmons2011}, we tested the effect of using more than one dependent variable on the FPP and FPR. We simulated data with two correlated dependent variables, $r=0.5$, and computed the average of the two dependent variables to yield a third dependent variable. With these three dependent variables, the model set expanded three times compared to only one dependent variable. 

\subsection{Data Generating Process}
As the goal of the simulation was to understand what can lead to false-positive results, the variable of interest was simulated without any correlation with either the dependent variable or any of the covariates. The variable of interest could either have a normal distribution (continuous variable) or a binomial distribution (binary variable). The covariates were simulated using the same distribution so that all covariate distributions in a dataset were either normal or binomially distributed. As our target analysis was linear regression, the dependent variable was always simulated as a normal distribution. There were four different combinations of the variable of interest $x$ and covariate $z$ distributions: $x$ and $z$ are continuous, $x$ and $z$ are binary, $x$ is continuous and $z$ is binary, $x$ is binary and $z$ is continuous. The correlation between the dependent variable and covariates was simulated with three different levels (\textit{r} = 0.2, \textit{r} = 0.3, and \textit{r} = 0.4) corresponding to medium-strength correlations \citep{Cohen1989}. The correlation between the variable of interest and covariates as well as the correlation between the covariates themselves was always set to zero. The correlation matrix for the data structure when there is only one dependent variable is presented in Table \ref{tab:correlation} in the \textit{Appendix}. If a variable was generated as a normally distributed variable, it was generated with a mean of 0 and a standard deviation of 1, while binomial variables were generated with a 50\% chance of success in each trial. To test how the sample size affected the FPP and FPR, the simulation included samples with sizes ranging from 100 to 300 with increasing steps of 50. 

\subsection{Simulation}
In the simulation results presented in the main results section, we only included cases in which the variable of interest and covariates had the same distribution. The two other cases i.e., when the variable of interest was binary and the covariates were continuous and the other way around can be found in the online \textit{Supplementary Information}. For analyses not addressing the effect of sample size, the default sample size was kept constant at 200. When the effect of different correlation levels between covariates was not analyzed, the correlation between the dependent variable and covariates was set to a default of $r=0.2$. We did not apply any outlier deletion except when outlier deletion was the focus of the simulation.\\
We used R \citep{Team2018} to perform our simulation with 10,000 simulation iterations. When the variables had a binomial distribution, the variables were simulated using the package BinNor \citep{Demirtas2014}. The data were analyzed with linear regressions. We defined a false-positive result as any model where the variable of interest was significant ($p < .05$), either in a main effect or in an interaction with a covariate variable. The FPP was defined as \\

\[FPP_i=\left. \left\{\begin{array}{c}
1\ if\ any\ model\ in\ iteration\ i\ produces\ a\ false-positive\ result \\ 
0\ otherwise\  \end{array}
\right.\] 
\[FPP=\frac{\sum_{i}^{N}{FPP_i}}{N}\] 

Where  $N$ denotes the number of iterations in the simulation. The FPR is the ratio of the models with a false-positive result and was defined as \\

\[FPR_i=\frac{\#\ false-positive\ models\ in\ iteration\ i}{\#\ all\ models\ in\ the\ set\ in\ iteration\ i}\] 
\[FPR=\frac{\sum_{i}^{N}{FPR_i}}{N}\] 



